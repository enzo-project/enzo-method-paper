
\section{Primordial chemistry and cooling}\label{app:chemistry}

\red{This currently just describes the rates solver without any radiation
backgrounds.  Improve this to include terms for radiation backgrounds (including
photoionizing and lyman-werner) and, if necessary, explain how these
terms fit in there.}

The primordial chemistry network implemented in \enzo\ is discussed in 
Section~\ref{sec.ov.chem} and also in
much more detail by~\citet{abel97} and~\citet{anninos97}.  
These papers describe the chemistry and cooling behavior of low-density primordial gas
($n \simeq 10^4$ and below), as well as
the steps that are necessary to obtain fast and accurate numerical solutions of the 
nonequilibrium chemical reaction rate network.  This network is extended by
~\citet{ABN02} to include the 3-body molecular hydrogen creation process, 
which becomes important at higher densities, and extends the validity of the reaction
network several more orders of magnitude in density, essentially until the gas becomes
optically thick to cooling by $H_2$ line emission. 
For the sake of completeness, and because the properties the primordial gas are so 
crucial to the results that are discussed in this work, we describe the chemistry 
network in this appendix.

Tables~\ref{table.collisional} and~\ref{table.radiative} summarize the collisional
and radiative processes solved in the \enzo\ nonequilibrium chemistry routines.  
\citet{abel97} show that accurate results can be obtained if several unnecessary
reactions are eliminated and the reaction network is reduced to the following:


\begin{eqnarray}
\frac{d \nd{H} }{dt}    &=& \k{2} \nd{\mHp} \nd{e} - \k{1} \nd{H} \nd{e} + 2 \k{31} \nd{\mHH}  \\
\frac{d\nd{\mHp}}{dt}  &=& \k{1} \nd{H}   \nd{e} - \k{2} \nd{\mHp}  \nd{e}      \\
\frac{d\nd{He} }{dt}  &=& \k{4} \nd{\mHep}\nd{e} - \k{3} \nd{He}   \nd{e}       \\
\frac{d\nd{\mHep}}{dt} &=& \k{3} \nd{He} \nd{e} + \k{6} \nd{\mHepp}\nd{e} -
                       \k{4} \nd{\mHep}\nd{e}                           \\
\frac{d\nd{\mHepp}}{dt}&=& \k{5} \nd{\mHep}\nd{e} - \k{6} \nd{\mHepp} \nd{e}    \\
\frac{d\nd{\mHH} }{dt}  &=& \k{8} \nd{\mHm} \nd{H} + \k{22} \nd{H}^3 - \nd{\mHH}   \left( 
        \k{31} + \k{11} \nd{\mHp} + \k{12} \nd{e} \right), 
\end{eqnarray}

where the number density of \Hm is given by the equilibrium condition

\begin{eqnarray}
\nd{\mHm} = {\k{7} \nd{H} \nd{e}  \over \k{8} \nd{H}  + 
 \k{16} \nd{\mHp} + \k{14} \nd{e}}.
\end{eqnarray}

The \Hm number density can be calculated in equilibrium because the timescale at which the
reactions controlling its number density occur are much shorter than the rest of the 
reactions in this system.  The rate coefficients used in the equations above are defined as follows:

\begin{eqnarray}       
 k_1 & = & \exp[-32.71396786 + 13.536556  \ln(T) - 5.73932875   \ln(T)^2 \nonumber \\ 
     &   & + 1.56315498  \ln(T)^3  - 0.2877056 \ln(T)^4 + 3.48255977\times 10^{-2} \times\ln(T)^5  \nonumber \\
     &   & - 2.63197617\times 10^{-3}\times\ln(T)^6 +  1.11954395\times 10^{-4}   \ln(T)^7 \nonumber \\
     &   & -  2.03914985 \times 10^{-6}   \ln(T)^8]~{\rm cm^3~s^{-1}}
\end{eqnarray}

\begin{eqnarray}
 k_2 &=& \exp[-28.6130338 - 0.72411256  \ln(T) - 2.02604473\times10^{-2}  \ln(T)^2  \nonumber \\
     & &   - 2.38086188\times10^{-3}  \ln(T)^3 - 3.21260521\times10^{-4}  \ln(T)^4  \nonumber \\
     & &   - 1.42150291\times10^{-5}  \ln(T)^5 + 4.98910892  \times10^{-6} \ln(T)^6  \nonumber \\
     & &   + 5.75561414 \times10^{-7}   \ln(T)^7 - 1.85676704 \times10^{-8}   \ln(T)^8 \nonumber \\
     & &   - 3.07113524 \times10^{-9}   \ln(T)^9]~{\rm cm^3~s^{-1}}
\end{eqnarray}

\begin{eqnarray}
 k_3 &=& \exp[ (-44.09864886 + 23.91596563   \ln(T) -  10.7532302   \ln(T)^2 \nonumber \\
     & &   + 3.05803875   \ln(T)^3 - 0.56851189   \ln(T)^4 + 6.79539123 \times 10^{-2}  \ln(T)^5  \nonumber \\
     & &   - 5.00905610 \times 10^{-3}   \ln(T)^6 + 2.06723616 \times 10^{-4} \ln(T)^7 \nonumber \\
     & &   - 3.64916141 \times 10^{-6}   \ln(T)^8)~{\rm cm^3~s^{-1}}
\end{eqnarray}

\begin{eqnarray}
k_{4r} & = & 3.925 \times 10^{-13}T^{-0.6353}~{\rm cm^3 s^{-1} } \\
k_{4d} &=& 1.544 \times 10^{-9} T^{-\frac{3}{2}} 
     \exp\left( - \frac{48.596~eV}{T} \right) \times \nonumber \\
   & &     \left[0.3 + \exp\left(\frac{8.10~eV}{T}\right)\right]~{\rm cm^3~s^{-1}}
\end{eqnarray}

\begin{eqnarray}
k_5 & = & \exp[-68.71040990 + 43.93347633   \ln(T) - 18.4806699   \ln(T)^2 \nonumber \\
    & &   + 4.70162649   \ln(T)^3 - 0.76924663   \ln(T)^4 + 8.113042 \times 10^{-2} \ln(T)^5 \nonumber \\
    & &   - 5.32402063 \times 10^{-3}  \ln(T)^6 +  1.97570531\times 10^{-4}  \ln(T)^7 \nonumber \\
    & &   - 3.16558106 \times 10^{-6}  \ln(T)^8]~{\rm cm^3~s^{-1}}
\end{eqnarray}

\begin{eqnarray}
k_6 = 3.36 \times 10^{-10} T^{-\frac{1}{2}}
\left(\frac{T}{1000\K}\right)^{-0.2}
\left(1+\left(\frac{T}{10^6\K}\right)^{0.7}\right)^{-1}~{\rm cm^3~s^{-1}}
\end{eqnarray}

$k_7$ for T $\leq 6000$~K:

\begin{eqnarray} 
k_7 & = & 1.429 \times 10^{-18} T^{0.7620} T^{0.1523 \log_{10}(T)}  T^{-3.274 \times 10^{-2} \log_{10}^2(T)}~{\rm cm^3~s^{-1}} \nonumber \\
\end{eqnarray}

$k_7$ for T $ > 6000$~K:

\begin{eqnarray}  
k_7 & = & 3.802 \times 10^{-17} T^{0.1998 \log_{10}(T)}  \nonumber \\
    &   & {\rm dex} \left( { 4.0415 \times 10^{-5} \log_{10}^6(T)  - 5.447 \times 10^{-3} \log_{10}^4(T)}~{\rm cm^3~s^{-1}}\right)
\end{eqnarray}

\begin{eqnarray}
T&>&0.1~{\rm eV}:\ k_8 = \exp[-20.06913897 +0.22898   \ln(T) +  3.5998377  \nonumber \\
 &&       \times 10^{-2}   \ln(T)^2 - 4.55512 \times 10^{-3}  \ln(T)^3- 3.10511544\times 10^{-4}   \ln(T)^4   \nonumber \\
 &&       + 1.0732940 \times 10^{-4}  \ln(T)^5 -  8.36671960 \times 10^{-6}   \ln(T)^6 + 2.23830623 \nonumber \\
  &&      \times 10^{-7}   \ln(T)^7]~{\rm cm^3~s^{-1}} . \\
T &<& 0.1~{\rm eV}:  k_8  =  1.428 \times 10^{-9}~{\rm cm^3~s^{-1}}
\end{eqnarray}

\begin{eqnarray}
\ln(k_{11}) & = &   -24.24914687 + 3.40082444  \ln(T) - 3.89800396  \ln(T)^2  \nonumber \\
             & &       +  2.04558782  \ln(T)^3 -  0.541618285  \ln(T)^4   + 8.41077503 \times 10^{-2}  \ln(T)^5 \nonumber \\
             & &       - 7.87902615 \times 10^{-3} \ln(T)^6 \nonumber + 4.13839842 \times 10^{-4} \ln(T)^7 \nonumber \\
             & &       - 9.36345888\times 10^{-6}  \ln(T)^8 {\rm cm^3 s^{-1}}
\end{eqnarray}


\begin{eqnarray}
\ \ k_{12} = 5.6 \times 10^{-11} T^\frac{1}{2} \exp(-\frac{102,124 K}{T}) {\rm cm^3 s^{-1}}
\end{eqnarray}


\begin{eqnarray}
k_{14}&=& \exp[-18.01849334 + 2.3608522   \ln(T) - 0.28274430   \ln(T)^2   \nonumber \\
      & &    + 1.62331664 \times 10^{-2}  \ln(T)^3 - 3.36501203 \times 10^{-2} \ln(T)^4   \nonumber \\
      & &    + 1.17832978 \times 10^{-2}   \ln(T)^5 - 1.65619470 \times 10^{-3} \ln(T)^6   \nonumber \\
      & &    + 1.06827520 \times 10^{-4}   \ln(T)^7 - 2.63128581 \times 10^{-6} \ln(T)^8) {\rm cm^3 s^{-1}}
\end{eqnarray}


\begin{eqnarray}\ \ k_{16} = 7 \times 10^{-8} \left( \frac{T}{100{\rm
        K}} \right)^{-\frac{1}{2}} {\rm cm^3 s^{-1}} 
\end{eqnarray}

$k_{22}$ For T $< 300$~K:

\begin{eqnarray}
k_{22} & = & 1.3 \times 10^{-32} (T / 300 {\rm K} )^{-0.38} {\rm cm^6 s^{-1}}
\end{eqnarray}

$k_{22}$ for T $\geq 300$~K:

\begin{eqnarray}
k_{22} & = & 1.3 \times 10^{-32} (T / 300 {\rm K} )^{-1} {\rm cm^6 s^{-1}}
\end{eqnarray}

The photodissociation of molecular hydrogen by Lyman-Werner radiation is controlled by the $k_{31}$ parameter: 

\begin{eqnarray}
k_{31} & = & 1.13 \times 10^8~{\rm F_{LW} }~|t|
\end{eqnarray}

where $|t| \equiv ( 4 \pi G \Omega_m \rho_c (1+z_i)^3)^{-1/2}$
and has units of seconds, and F$_{LW}$ is the Lyman-Werner background flux in 
units of~erg~s$^{-1}$~cm$^{-2}$~Hz$^{-1}$.


% this is the end!
