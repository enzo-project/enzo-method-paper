%  If you are editing this file to add acknowledgments, please note that
%  some of the grant number formats have been edited a bit (from, say,
%  08-08184 to 0808184) to ensure consistency between different
%  grants.  I ask that you please adhere to the current formatting,
%  etc. as well.  Thanks!  --Brian

\acknowledgments

Development of \enzo\ has been ongoing since 1994 by a wide range of
agencies and institutions.  In all grants listed, we put the
initials of the PI (if an \enzo\ developer) or the \enzo\ developer
funded by the grant (if the PI is not a developer of \enzo).

This work has been supported by the National Science Foundation by
grants
AAG-0808184 (DRR),
AAG-1109008 (DRR),
ACI-9619019 (MLN),
ASC-9313135 (MLN),
AST-9803137 (MLN), 
AST-0307690 (MLN), 
AST-0407176 (RC),
AST-0407368 (SS, EJH),
AST-0507521 (RC), 
AST-0507717 (MLN), 
AST-0507768 (AK),
AST-0529734 (TA),
AST-0607675 (AK),
AST-0702923 (EJH),
AST-0707474 (BDS), 
AST-0708960 (MLN), 
AST-0807075 (TA),
AST-0807215 (JB),
AST-0808184 (MLN, AK),
AST-0808398 (TA),
AST-0908740 (AK, DCC),
AST-0908819 (BWO), 
AST-0955300 (NJG),
AST-1008134 (GB), 
AST-1009802 (JSO), 
AST-1102943 (MLN),
AST-1106437 (JB),
AST-1210890 (GB),
AST-1211626 (JHW),
OCI-0832662 (BWO, MLN),
OCI-0941373 (BWO),
PHY-1104819 (MLN, JB),
the CI TraCS fellowship (OCI-1048505; MJT),
and the Graduate Research Fellowship program (NJG; SWS).

This work has been supported by the National Aeronautics and Space
Administration through grants
NAGW-3152 (MLN),
NAG5-3923 (MLN),
NNX08AH26G (MLN, TA),
NNX09AD80G (BWO),
NNX12AH41G (GB),
NNX12AC98G (BWO),
NNZ07-AG77G (BDS),
NNG05GK10G (RC),
ATP09-0094 (SVL),
Chandra Theory grant \#TM9-0008X (BWO),
Hubble Space Telescope Theory Grant HST-AR-10978.01 (BDS),
the Fermi Guest Investigator Program (\#21077; BWO),
and the Hubble Postdoctoral Fellowship through the Space Telescope Science
Insititue, \#120-6370 (JHW).

This work has been supported by the Department of Energy via the
Los Alamos National Laboratory (LANL) Laboratory Directed Research and
Development Program (BWO, DCC, HX, SWS), 
the LANL Institute for Geophysics and Planetary Physics (BWO, DCC, CP,
BC),
the Los Alamos National Laboratory Director's Postdoctoral Fellowship
program (No. DE-AC52-06NA25396;
BWO and DCC), and the
DOE Computational Science Graduate Fellowship (DE-FG02-97ER25308; SWS)

Additional financial support for the \enzo\ code has come from
Canada's NSERC through the USRA and CGS programs (EL) and through a 
Japan MEXT grant for the Tenure Track System (EJT).

We acknowledge the  many academic institutions that have supported \enzo\
development, including (in alphabetical order)
Columbia University,
Georgia Institute of Technology,
Michigan State University and the MSU Institute for Cyber-Enabled
Research, 
the National Center for Supercomputing Applications, 
the Pennsylvania State University,
 the San Diego Supercomputer Center (through the Strategic Applications
Partner program and the Director’s office),
Princeton University,
SLAC National Accelerator Laboratory,
the SLAC/Stanford Kavli Institute for Particle
Astrophysics and Cosmology,  
Southern Methodist University,
Stanford University,
the University of Arizona,
the University of Califoria at San Diego, 
the University of Colorado at Boulder,
the University of Florida,
and the University of Illinois.
We acknowledge support from the Kavli Institute for Theoretical
Physics at Santa Barbara, the Aspen Center for Physics, and the UCLA
Institute for Pure and Applied Mathematics, which have
generously hosted \enzo\ developers through their conference and
workshop programs.

Computational resources for \enzo\ development have come from the NSF
XSEDE program, the NASA High Performance Computing program, the DOE INCITE
program, and the DOE Advanced Simulation and Computing (ASC) program.

The \enzo\ collaboration would like to acknowledge the following
scientists, who have made contributions to the \enzo\ codebase at some
point during their research career: Brian Crosby, Elizabeth
Harper-Clark, Daegene Koh, Eve Lee, Pascal Paschos, Carolyn Peruta,
Alex Razoumov, Munier Salem, and Rick Wagner.

The \enzo\ collaboration would also like to acknowledge the significant contributions to
\enzo\ development made by the late Dr. Robert P. Harkness. 

