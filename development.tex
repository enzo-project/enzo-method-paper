
%%%%%%%%%%%%%%%%%%%%%%%%%%%%%%%%%%%%%%%%%%%%%%%%%%%%%%%%%%%%%%%%%%%%%%%%%%%%%
\section{Code development methodology}
\label{sec.development}

Over time, \enzo's development has followed a trajectory toward
openness.  Started as the graduate research project of Greg Bryan at
the University of Illinois, it has been subsequently been stewarded by
the Laboratory for Computational Astrophysics (LCA) at the University
of California at San Diego, and has transitioned to a distributed,
completely open, and community-driven project.  Initially, \enzo\ was
versioned using a series of ``snapshots'' of the code base, usually
hand-created by the developing individuals.  These were distributed to
collaborators and colleagues, but the central ``trunk'' of development
was updated primarily by a single person: while patches and technology
were accepted from external developers, the relatively small number of
individuals using the code resulted in a strong centralization of
development.

As the stewardship of the code passed to the LCA, the code was
released first to ``friendly users'' and then as an open source
release.  However, while the code was made available with
documentation, technology developed in the broader community of users
was typically not re-integrated.  This led to a wide dispersal of
development, largely independent, by individuals who downloaded and
used the version of the code developed by the LCA.

Following the first public, open source release of \enzo, the code was
migrated to using the Subversion version control system.  This is a
centralized version control system, and the ``stable'' \enzo\ source
code was made globally readable following the \enzo\ 1.5. release.
Access to the primary development tree required a password and login
for each user, and providing upstream changes either required this
password and the granting of write access or a sequence of patches and
manually-created diffs (much like the original development system).
The technical friction of manually contributing patches and
modifications, combined with Subversion's difficulty with tracking
merges, resulted in further fragmentation of the code base.

The main development streams of \enzo\ were eventually merged using
the distributed version control system (DVCS) Mercurial
(\url{http://mercurial.selenic.com/}) into a branch of the code known
as \texttt{week-of-code}, so named after the in-person development
sprint at which it was created.  The fundamental, and transformative,
distinction between the previous \textit{centralized} version control
system and mercurial's \textit{distributed} version control system is
the elimination of gatekeepers.  While there still exists a canonical,
central location where stable and development versions of \enzo\ can
be obtained, changesets and versions can be exchanged between peers
without the intervention of designated gatekeepers.  This has the
direct effect of enabling local development to be versioned and its
provenance ensured, while still retaining the ability to benefit from
``upstream'' development.  An important, even crucial, side effect is
that the technology used for local versioning provides mechanisms for
easily submitting locally-developed modifications to the community
source location.  Mercurial internally represents all changes as nodes
in a directed, acyclic graph (DAG), which results in the natural
ability to more consistently and easily manage merging development
streams.

Currently, \enzo\ is developed using the hosted source control
platform Bitbucket (\url{http://bitbucket.org/}) at
\url{http://bitbucket.org/enzo/}.  There are two mailing lists, one
for usage-focused questions and discussion, and another for
development discussion.  Both of these lists are open and publicly
archived.  Bitbucket provides mechanisms for inspecting source,
hosting branches and forks of the primary source, and for code review.
All proposed source code changes for \enzo\ are subjected to a peer
review process, where experienced developers read, inspect, test, and
provide feedback on source code changes.  All developers, including
long-time \enzo\ contributors and developers, are subject to this
process before their code is included in the primary \enzo\
repository.  By using a remote, hosted system, \enzo\ is now
\textit{completely} open to contributions from the community.
Individuals, who may or may not consider themselves \enzo\ developers,
are free to ``fork'' the \enzo\ code base, develop changes (signed
with their own name), and submit them for review and inclusion.  In
contrast to the centralized, gatekeeper-focused technology used
previously, this enables anyone to contribute changes to be evaluated
for inclusion in the \enzo\ codebase.

While peer review is able to catch many bugs and problems with source
code changes, \enzo\ is also subject to ``answer'' testing.  We have
created a set of parameter files and problem types that exercise the
underlying machinery of \enzo.  These ``test problems'' have
affiliated ``tests,'' which consist of scripts that use \texttt{yt}
\citep{2011ApJS..192....9T} to produce results such as mass
distribution, projections, profiles and so on.  The testing
infrastructure then evaluates whether the variation in the new results
compared to a ``gold standard'' of results has exceeded an acceptable
threshold, typically set to roughly single precision.  Optionally, for
those test problems that are deemed unsafe to change to any precision,
the tests also produce hashes of the outputs; these hashes will only
remain unchanged in the event of bitwise identicality between results.
The results of the gold standard are versioned and stored in Amazon
S3, enabling remote testing to proceed.  While the testing process --
building, running, analyzing and comparing -- is not yet automated
against incoming pull requests, we hope to deploy that functionality
in the future.  The primary challenge is that of compute time; the
tests are organized into multiple categories, including by the
expected run time, but the full suite of tests can take several days
to run.
