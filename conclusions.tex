%%%%%%%%%%%%%%%%%%%%%%%%%%%%%%%%%%%%%%%%%%%%%%%%%%%%%%%%%%%%%%%%%%%%%%%%%%%%%
\section{Conclusions}
\label{sec.conclusions}

In this paper, we have presented the algorithms underlying \enzo, an
open-source adaptive mesh refinement code designed for
self-gravitating compressible fluid dynamics, including the effects of
magnetic fields, radiation transport, and a variety of microphysical
and subgrid processes.  In addition, we have described the \enzo\ code
development process, have shown the outputs of a set of representative
test problems, and have provided information about \enzo's performance
and parallel scaling on recent supercomputing platforms.  The \enzo\
code, its test suite, and all of the scripts used to generate plots
and figures for this paper are open source and are available at the
\enzo\ website, \url{http://enzo-project.org}.  Furthermore, the
\texttt{yt} toolkit, which is designed to analyze \enzo\ data (as well
as data from a wide variety of other simulation tools), can be found
at its website, \url{http://yt-project.org}.  Both of these codes have
active user and developer communities, extensive documentation and
user support, and strong mechanisms for users to contribute their
changes and fixes to the codebase.

The developers of the \enzo\ code are currently working on several
projects that will extend the functionality, scalability, or overall
performance of the code in the near future.  Projects that will appear
in forthcoming releases of the \enzo\ code include:

\begin{itemize}
\item The creation of a hybrid-parallel version of \enzo, combining
MPI for communication between nodes of a supercomputer and OpenMP for
thread-based parallelism within a node.  This will reduce on-node
memory usage and improve overall scaling behavior.
\item The restructuring of \enzo's treatment of particles to
accommodate a wider range of ``active'' particles that can easily
interact with each other and with multiple grids, and include sink,
source, and particle creation, destruction, splitting, and merging functionality.
\item A new HYPRE-based AMR gravity solver that is faster, more
accurate, and more scalable than the current multigrid solver.
\item New infrastructure for problem initialization, enabling
  users to more quickly and easily create new types of simulations.
\end{itemize}

With the continual rapid development of computer hardware, it makes
sense to not only review \enzo's current capabilities, but to look
toward its future development in view of predicted technological
trends. These trends in supercomputing hardware suggest that
substantial modifications to \enzo's core infrastructure, and very
possibly some of the core algorithms, will be required. More
specifically, the progression involves the usage of specialized
large-core-count, vectorized computing units such as graphics
processing units or chips like the Intel Xeon Phi, as well as
precipitously decreasing amounts of RAM per computing core.  The
former trend means that the amount of processing power per compute
node will continue to increase, likely much faster than the bandwidth
between nodes, and will require tremendous reduction in inter-node
(and possibly inter-CPU) communication in order to maintain code
scalability.  Also, much of the current code will need to be rewritten
to take advantage of the vector nature of these CPUs, making
assumptions that are quite unlike those made in much of the current
codebase.  The latter trend means that duplication of data -- for
example, the grid hierarchy -- must be effectively eliminated to save
memory, and all inter-core and inter-node communication must be
carefully thought through to minimize the amount of data moved.  An
additional challenge as one goes to core counts in the tens to
hundreds of millions (or more) is that the reliability of individual
computing elements will become much more of an issue, requiring
robustness to hardware failure to be built into the code at some
level.  Furthermore, we are nearing the physical limits of transistor
speed and interconnect latency~\citep{feynman1999feynman}, meaning that simple
hardware improvements will not make these challenges disappear, and
careful thought (and the rewriting of a great deal of code) must take
place!  These challenges are not unique to the \enzo\ code, and in
fact are faced by effectively all applications that wish to take
advantage of new computational architectures. We therefore anticipate
that \enzo\ (or a code that has the capabilities of \enzo, from a
user's point of view) will continue to be usable at the
largest scales on such machines.

