
%%%%%%%%%%%%%%%%%%%%%%%%%%%%%%%%%%%%%%%%%%%%%%%%%%%%%%%%%%%%%%%%%%%%%%%%%%%%%
\section{Conclusions}

In this paper, we have presented the algorithms underlying \enzo, an
open-source adaptive mesh refinement code designed for
self-gravitating compressible fluid dynamics, including the effects of
magnetic fields, radiation transport, and a variety of microphysical
and subgrid processes.  In addition, we have described the \enzo\ code
development process, have shown the outputs of some representative
test problems, and have provided information about its performance and
scaling on current-generation (circa 2013) supercomputing platforms.
The \enzo\ code, its test suite, and all of the scripts used to
generate plots and figures for this paper are open-source and are
available at the \enzo\ website, http://enzo-project.org.
Furthermore, the yt toolkit, which is designed to analyze \enzo\ data
(as well as data from a wide variety of other simulation tools), can
be found at its website, http://yt-project.org.  Both of these codes
have active user and developer communities, extensive documentation
and user suppport, and strong mechanisms for
users to contribute their changes and fixes to the codebase.

The developers of the \enzo\ code are currently working on several
projects that will extend the functionality, scalability, or overall
performance of the code in the near future.  Work is
underway to create a hybrid-parallel version of \enzo, combining MPI
for communication between nodes of a supercomputer and OpenMP for
thread-based parallelism within a node.  This will reduce on-node
memory usage and will in principle help with scaling.  \red{other stuff here!}.
This work is currently underway, and will be released in the stable
branch of \enzo\ in the near
future.

In the longer term, we plan to restructure \enzo's treatment of
particles to accommodate a wider range of ``active'' particles that
can easily interact with each other and with grids.  We also plan to
use the HYPRE library to create a new gravity solver for the code,
which will improve its scalability and correctness.  \red{other stuff
  here!}

\red{Do we want a separate paragraph for areas of improvement outside
  of short-term and longer-term?}

 