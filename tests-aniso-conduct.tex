\subsubsection{Anisotropic Thermal conduction}
\label{sec.tests.conduct}

Figure~\ref{fig.conduct} shows a test that demonstrates the correct
behavior of anisotropic thermal conduction in \enzo. We initialize a
two-dimensional, $256 \times 256$ cell simulation having a physical
scale of 1 kpc on a side, a uniform density of 1 proton cm$^{-3}$ and
a background temperature of $10^6$ Kelvin.  Magnetic fields with a
strength of B$_0 = 1$~$\mu$G are initialized such that the field lines
form circles around the center of the simulation volume, such that
B$_x = -B_0\sin(\theta)$ and B$_y = B_0\cos(\theta)$, where $\theta$
is the angle measured from the $+x$ direction in a counterclockwise
manner.  A Gaussian temperature pulse is injected at $(0.75, 0.5)$ (in
units of the box size), with a peak temperature of $10^8$ K and a FWHM
of $1/64$ of the box size.  This initial setup is shown in the left
panel of Figure~\ref{fig.conduct}.  The simulation is then allowed to
evolve with \textit{only} anisotropic conduction turned on (e.g., no
hydrodynamics, radiative cooling, or cosmological expansion), and with
a Spitzer fraction of f$_{\rm sp} = 1$.

The right panel of Figure~\ref{fig.conduct} shows the state of the
simulation after 300 Myr.  Heat has clearly been transported only
along field lines -- there has been no diffusion perpendicular to the
magnetic field setup, which is critical for many studies involving
anisotropic thermal conduction.  No oscillations are seen in the
temperature field in regions where the fields are not aligned with the
grid, suggesting that the flux-limiter is operating as expected (see
discussion in Section~\ref{sec.num.conductions}).

\begin{figure}
\begin{center}
\includegraphics[width=0.42\textwidth]{figures/aniso_conduction_initial_output.eps}
\includegraphics[width=0.42\textwidth]{figures/aniso_conduction_final_output.eps}
\caption{Two-dimensional anisotropic conduction test in a uniform,
constant temperature background with circular magnetic fields
(indicated by white streamlines) centered on (0.5, 0.5) in a $256
\times 256$ grid. The background medium has a density of 1 proton/cc
and a temperature of $10^6$~K.  At t$ = 0$ (left panel), a Gaussian
heat pulse is injected at (0.75, 0.5) with a FWHM of $1/64$ (with all
numbers given in units of the box size) and a peak temperature of
$10^8$~K, and allowed to evolve without hydrodynamical motion (i.e.,
static gas) and no radiative cooling for 300 Myr.  At t$ = 300$~Myr
(right panel), heat has been transported along magnetic field lines
with no significant diffusion perpendicular to field
lines. Furthermore, there are no detectable oscillations in the
temperature in regions where the magnetic field is not parallel with
the grids.}
\label{fig.conduct}
\end{center}
\end{figure}
