In this section, we describe the four solvers that we have implemented for
solving the fluid equations.  We describe the PPM method in considerably
more detail than the other methods in part
because its implementation in Enzo has not previously been described,
but mostly because it introduces many of the ideas and methods used for
the MUSCL-Dedner and MHDCT schemes (including expansion terms,
reconstruction, Riemann solvers, and dual energy formalism);

\subsection{Hydrodynamics: The PPM hydro method}
\label{sec.hydro.ppm}

One (purely) hydrodynamic method used in \enzo\ is closely based on the
piecewise parabolic method (PPM) of
\citet{1984JCoPh..54..174C}, which has been
modified for the study of cosmological fluid flows.  
\citet{1984JCoPh..54..174C} describe two variants on this method:
Lagrangian Remap and Direct Eulerian; here we use the Direct Eulerian
version which is better suited for AMR simulations.  The Lagrangian
Remap version has previously been adapted for cosmological use
as described in \citet{1995CoPhC..89..149B} and the implementation
we use here is closely based on that work.

The first term on the right hand side of equations~(\ref{eq:momentum}) and
(\ref{eq:total_energy}) come from our choice of comoving coordinates
(a similar term also appears in equation~(\ref{eq:dm_velocity}), the velocity 
relation for the dark matter particles).  A similar term does not
appear in the mass conservation equation (\ref{eq:mass}) because
of the comoving density definition.   
We note that these expansion terms could be
eliminated entirely by the proper choice of variables (including time),
although we have not done so here, as they do not constitute a major
source of error.  We solve these terms
using the same technique -- they are split from the rest of the terms and
solved using an (implicit) time-centered method, which is straightforward
as there are no spatial gradients.  Note that we use this method for
all four of the hydro solvers.

The remainder of the terms in the fluid equations
involve derivatives.  
Because we are interested in phenomena with no special geometry, we will
restrict discussion to Cartesian coordinates.
We can dimensionally split the equations and rewrite the one-dimensional (Eulerian) 
versions of equations~(\ref{eq:mass})--(\ref{eq:total_energy}) without expansion terms
in conservative form as,
\begin{eqnarray}
\frac{\partial \rho}{\partial t}  + \frac{1}{a} \frac{\partial \rho v }{\partial x}    & = &  
     0 \label{eq:mass1d} \\
\frac{\partial \rho v}{\partial t}  + \frac{1}{a} \frac{\partial \rho v^2}{\partial x}   + 
      \frac{1}{a} \frac{\partial p}{ \partial x} & = & 
      \rho \frac{1}{a} \frac{\partial \phi}{ \partial x}  \label{eq:momentum1d}  \\
\frac{\partial \rho E}{\partial t}  + \frac{1}{a} \frac{\partial \rho v E}{\partial x}  & =  &
      \rho v \frac{1}{a} \frac{\partial \phi}{\partial x}. \label{eq:total_energy1d}
\end{eqnarray}
%
Here $x$ and $v$ refer to the one dimensional comoving position and 
peculiar velocity of the baryonic gas, and $g$ is the acceleration at the cell center.
These equations are now in a form that can be solved by the split PPM scheme.

% ------------------------------------------------------------

%\subsubsection{The piecewise parabolic method in one dimension} % 2.2

We now restrict ourselves to the solution of equations
(\ref{eq:mass1d})--(\ref{eq:total_energy1d}), in one
dimension.  In the Direct Eulerian version of PPM, this is
accomplished by a three-step process.  First, we compute effective
left and right states at each grid boundary by constructing a
piecewise parabolic description of the primative variables ($\rho$, $u$
and $E$) and then averaging over the regions corresponding to the
distance each characteristic wave can travel ($u$, $u-c_s$ and
$u+c_s$, where $c_s$ is the sound speed).
Second, a Riemann problem is solved using these effective left and
right states, and finally the fluxes are computed based on the
solution to this Riemann problem and the conserved quantities are
updated.  This is described in detail in
\citet{1984JCoPh..54..174C}, but we will briefly outline the procedure
here both for completeness and to put the changes we will make in
context.


The Eulerian difference equations are:
%
\newcommand{\avp}[1]{\overline{#1}_{j+1/2}}
\newcommand{\avm}[1]{\overline{#1}_{j-1/2}}
\begin{equation}
\rho_j^{n+1} = \rho_j^{n} + \Delta t \left( 
                      \frac{ \avp{\rho}\avp{v} -  \avm{\rho}\avm{v} } {\Delta x_j}
            \right)
     \label{eq:mass_diff}
\end{equation}
\begin{equation}
\rho_j^{n+1} v_{j}^{n+1} =
       \rho_j^{n+1} v_j^n  + \Delta t \left(
          \frac{ \avp{\rho} \avp{v}^2 - \avm{\rho} \avm{v}^2 + \avp{p} - \avm{p}} {\Delta x_j}
             \right)
     + \frac{ \Delta t }{2} g_j^{n+1/2} (\rho_j^n + \rho_j^{n+1}), 
     \label{eq:momentum_diff}
\end{equation}
\begin{eqnarray}
\rho_j^{n+1} E_j^{n+1}  = 
       \rho_j^n E_j^n  & + & \Delta t  \left(
            \frac{  \avp{\rho} \avp{v} \avp{E}  - \avm{\rho} \avm{v} \avm{E}  +
                       \avp{v} \avp{p}    - \avm{v} \avm{p} } {\Delta x_j}
             \right) \nonumber \\
         & & + \frac{ \Delta t }{2} g_j^{n+1/2} (\rho_j^n v_j^n + \rho_j^{n+1} v_j^{n+1} ).
     \label{eq:energy_diff}
\end{eqnarray}

We have used subscripts to indicate zone-centered ($j$)
and face-centered ($j+1/2$) quantities, while superscripts refer to
position in time.  The cell width is $\Delta x_j$.
Although they have been discretized
in space, the accuracy of the updates depend on how well we can compute
the fluxes into and out of the cell during $\Delta t$.  
This in turn depends on our ability to compute the time-averaged 
values of $p$, $\rho$ and $v$ at the cell interfaces, denoted here
by $\overline{p}_{j\pm 1/2}$, $\overline{\rho}_{j\pm 1/2}$, and $\overline{v}_{j\pm 1/2}$.  We now
describe the steps required to compute these quantities.

We first construct monotonic piecewise 
parabolic (third order) interpolations in one dimension
for each of $p$, $\rho$, and $v$.  The pressure is determined from
equation~(\ref{eq:eq_of_state}), the equation of state.
The interpolation formula for some quantity $q$ is given by:
%
\begin{eqnarray}
q_j(x) & = &  q_{L,j} + \tilde{x}(\Delta q_j + q_{6,j}(1-\tilde{x})), \\
\tilde{x}      & \equiv & {x - x_{j-1/2} \over \Delta x_j}, \qquad
             x_{j-1/2} \leq x \leq x_{j+1/2}. \nonumber
\end{eqnarray}
%
This is equation~(1.4) of CW84 (in the spatial rather than mass coordinate as is appropriate for the direct Eulerian approach). 
The quantities $q_{L,j}$, $\Delta q_j$,
and $q_{6,j}$ can be viewed simply as interpolation constants; however,
they also have more intuitive meanings.  For example, $q_{L,j}$ is the
value of $q$ at the left edge of zone j, while $\Delta q_j$ and $q_{6,j}$ are analogous to the slope and first order correction to the slope of $q$ (see \citet{1984JCoPh..54..174C} for a complete discussion):
\begin{equation}
\Delta q_j \equiv q_{R,j} - q_{L,j} \qquad 
q_{6,j}    \equiv 6\left[q_j - 1/2\left(q_{L,j} + q_{R,j}\right)\right].
\end{equation}

We have reduced the problem to finding $q_{L,j}$ and $q_{R,j}$.  While this
is simple in principle, it is complicated somewhat by the requirement that
these values be of sufficient accuracy and that the resulting distribution
be monotonic.  That is, no new maxima or minima are introduced.
The resulting formulae are straightforward but complicated and are not
reproduced here, but see Equations 1.7 to 1.10 of \citet{1984JCoPh..54..174C}.
We also optionally allow steepening as described in that reference.

Once we have the reconstruction, the primary quantities ($p$, $\rho$, $v$ and $E$) are averaged over the domains corresponding to the three characteristics $C = u-c$, $u$ or $u+c$ (where $c$ is the sound speed in a cell):
\begin{equation}
      q^{C+}_{j+1/2} =
          {1 \over \Delta t C^n_j} \int^{x_{j+1/2}}_{x_{j+1/2}-\Delta t C^n_j} q_j(x) dx
\end{equation}
where $q^{C+}$ refers to the left characteristic $C$ (and a similar integral is done for the right characteristic $q^{C-}$).  The linearized gas dynamics equations are then used to compute second-order accurate left and right states that take into account the multiple wave families.  This process is described in \citet{1984JCoPh..54..174C} and we use their equations (3.6) and (3.7).

With these effective states, an approximation to the Riemann problem is found
(see below for more detail about the Riemann solvers used), producing estimates for 
$\overline{p}_{j\pm 1/2}$, $\overline{\rho}_{j\pm1/2}$, and $\overline{v}_{j\pm 1/2}$ that are third order
accurate in space and second order accurate in time.  These are then
used to solve the difference Equations (\ref{eq:mass_diff})--(\ref{eq:energy_diff}) for $\rho^{n+1}$, $v^{n+1}$, and $E^{n+1}$.

We include an optional diffusive flux (and flattening for the parabolic curves) that can improve the solution in some cases.  Our implementation follows that in the appendix of \citet{1984JCoPh..54..174C}.

In addition, as discussed earlier, the three dimensional scheme is achieved by operator splitting and repeating the above procedure in the other two orthogonal directions.  The transverse velocities and any additional passive quantities are naturally and easily added to this system (see Equation 3.6 of \citet{1984JCoPh..54..174C}).  

We note that the acceleration required in Equation~(\ref{eq:momentum_diff}) is actually the
acceleration felt by the entire zone and not just at the zone center.
Therefore, it is possible to find the mass weighted average acceleration over the zone
by expanding the density and acceleration distributions and
retaining all terms up to second order in $\Delta x$:
%
\begin{equation}
g_j^{n+1/2} = 
       \frac{1}{2 a^{n+1/2} \delta x_j} \left[ 
             \phi_{j+1}^{n+1/2} 
           - \phi_{j-1}^{n+1/2} 
           + \frac{1}{12} \left(    \phi^{n+1/2}_{j+1} 
                                 - 2\phi^{n+1/2}_j 
                                 + \phi^{n+1/2}_{j-1} \right) 
                                   {\delta d_j \over d_j}
       \right].
       \label{eq:accel}
\end{equation}
Generally we find that the potential is so slowly varying that this change is unnecessary.

% ------------------------------------------------------------

\subsubsection{The dual energy formulation for  very high Mach flows} % 2.4

The system described thus far works well for gravitating systems with
reasonable Mach numbers ($<100$) as long as the structures are well resolved.
This section and the next detail changes that are required
to correctly account for situations where one or both of these
requirements are not met.

Large, hypersonic bulk flows appear to be very common in cosmological
simulations and they present a problem because of the high ratio of
kinetic energy $E_k$ to gas internal energy $e$, which can reach as
high as $10^8$.  Inverted, we see that the internal energy consists of
an extremely small portion of the total energy.  The pressure then,
proportional to $E - E_{k}$, is the small difference between two large
numbers: a disastrous numerical situation.  This is not as large a
problem as it may at first appear because it only occurs when the
pressure is negligibly small.  Therefore, even if we suffer large
errors in the pressure distribution in these regions, the dynamics and
total energy budget of the flow will remain unaffected.  Nevertheless,
if the temperature distribution is required for other reasons
(e.g., for calculating radiative processes), a remedy is required.

To overcome this, we also solve the internal energy equation:
\begin{equation}
 \frac{\partial e}{\partial t} 
           + \frac{1}{a} \vecv \cdot \grad e
%         = - \frac{3(\gamma -1)\dot{a}}{a} e
         = - 3 \frac{\dot{a}}{a} \frac{p}{\rho}
           - \frac{p}{a\rho} \div \vecv
\end{equation}
in comoving coordinates.  The structure is similar to the total energy
equation; the second term on the left hand side represents transport, while
the first term on the right is due to expansion of the coordinate
system.  
It is differenced (again, in Eulerian form without the expansion term) as,
%
\begin{equation}
\rho_j^{n+1} e_j^{n+1}  = 
       \rho_j^n e_j^n   +  \Delta t  \left(
            \frac{  \avp{\rho} \avp{v} \avp{e}  - \avm{\rho} \avm{v} \avm{e}} {\Delta x_j} \right)
            - \Delta t \ p_j^{n} \left( \frac{ \avp{v} - \avm{v}} {\Delta x_j} 
                      \right)
                  \label{eq:gasenergy_diff}
\end{equation}
%
Note that because of the structure of this equation, it is not in
flux-conservative form.  In particular, the pressure is evaluated at
the cell center.  Unfortunately, time-centering of this pressure has
proved difficult to do without generating large errors in the internal
energy and so we leave the pressure at the old time in
this difference equation.  This leads to some spreading of shocks; however,
we note that this equation is only used in hypersonic flows.

It is necessary, however, to conserve the total energy so that the conversion
of kinetic to thermal energy is performed properly.   We must therefore,
combine the two formulations without allowing the separately advected
internal energy $e$ to play a role in the gas dynamics.  This is done by
carrying both terms through the simulation and using
the total energy $E$ for hydrodynamic routines and the internal energy
$e$ when the temperature profile is required.  One way to view this procedure
is to treat $e$ as enhanced precision (extra digits) for $E$ that automatically
`floats' to where it is needed.
We only require that they be kept synchronized when the two levels of precision
overlap.

When the pressure is required solely for dynamic purposes, the 
selection criterion 
operates on a cell by cell basis using,
\begin{equation}
p = \cases{ \rho(\gamma - 1)(E - \vecv^2/2),& 
                  $  (E - \vecv^2/2)/E > {\eta}_1 $; \cr
            \rho(\gamma -1)e,&
                  $  (E - \vecv^2/2)/E < {\eta}_1 $. \cr}
\end{equation}

It should be stressed that as long as the parameter ${\eta}_1$ is small enough
the dual energy method {\it will have no dynamical effect}.  We use
${\eta}_1 = 10^{-3}$, which is consistent with the truncation error of the
scheme for grid sizes that are typically used in our simulations.  We are now free to select the method by which the internal energy
field variable $e$ is updated so that it will not become contaminated with
errors advected by the total energy formulation but still give the
correct distribution in shocked regions. 
Since we are concerned with the advection of errors, the selection
criterion must look at each cell's local neighbourhood.
In one dimension, this is done with,
\begin{equation}
e = \cases{ (E - \vecv^2/2),& $\rho(E - \vecv^2/2)/
    \max ({\rho}_{j-1}E_{j-1},{\rho}_j E_j,{\rho}_{j+1}E_{j+1}) > {\eta}_2$,\cr
            e,& $\rho(E - \vecv^2/2)/
    \max ({\rho}_{j-1}E_{j-1},{\rho}_j E_j,{\rho}_{j+1}E_{j+1}) < {\eta}_2$.\cr}
    \label{eq:dualselect}
\end{equation}
Thus, ${\eta}_2$ determines when the synchronization (of $e$ with $E$) occurs.
%is a measure of how much error we will allow to be passed
%from neighboring cells.  
Too high a value may mask relatively weak shocks, 
while spurious heating (via contamination) may occur if it is set too low.
After some experimentation,
we have chosen ${\eta}_2 = 0.1$, a somewhat conservative value.
This scheme is optional and is generally only required in large-scale 
cosmological simulations where the gas cools due to the expansion of the universe
but large bulk flows develop due to the formation of structure.

We note that others have independently developed a similar but
distinct scheme for dealing with this problem, which is endemic to
methods adopting the total energy equation.  In \citet{TVD93} the 
two variables adopted are total energy and
entropy (rather than total energy and thermal energy), with an
analogous scheme for choosing which variable to employ.

\subsubsection{Riemann Solvers and Fallback}
\label{sec.riemann}

In this section, we describe the methods we adopt to solving the Riemann
problem, which is generally required to compute the fluxes in 
any Godunov.  This section therefore applies to all three of our 
Godunov-based schemes.  The Riemann problem we are solving
involves two constant states separated by a single discontinuity at $t=0$.
The subsequent evolution has an exact analytic solution.  This solution is
described in detail in many texts on computational fluid dynamics
\citep[e.g.,][]{toro-1997}.   In brief, there are three waves that propagate
away from the initial discontinuity.  The central wave, characterized by a
density jump but not a pressure jump, is called the contact discontinuity.  The
waves traveling to the left and right of the contact discontinuity can be either
shocks, if characteristics converge on the wave front, or rarefaction fans if
characteristics diverge.  

While there exists an exact solution to this problem, finding it is expensive.  There are four
possible combination of left- and right- traveling shocks and rarefactions,
only one of which is fully consistent with the initial conditions.  Once the
correct physical state is determined, the pressure in the central region can
only be found by finding the root to an algebraic equation, which is necessarily
an iterative process.  Thus a series of \emph{approximate} Riemann solvers are
typically used.  There are four approximate Riemann solvers in \enzo: two-shock
\citep{toro-1997},
Harten-Lax-van Leer \citep[HLL,][]{toro-1997}, HLL with a contact discontinuity
 \citep[HLLC,][]{toro-1997}, and HLL with
multiple discontinuities \citep[HLLD,][]{Miyoshi05}.  Two-shock is used only
with the PPM method.  HLL and HLLC are used with PPM, MUSCL (both with and
without MHD) and MHD-CT.  HLLD is exclusively an MHD solver, and works with both
MUSCL and MHD-CT methods.  

The only approximation that two-shock makes is that both left- and right- moving
waves are shocks.  This solution still requires an iterative method for finding
the pressure in between the two waves.  The HLL method alleviates this iteration
by assuming that there is no central contact discontinuity, and the signal speed
in the central region is approximated by an average over the left- and right-
moving waves.  This method is significantly faster than the two-shock method,
but also quite a bit more dissipative.  The HLLC is a three-wave method that improves upon the HLL
method by also including the third wave, the contact discontinuity.  

For the MHD equations, there are seven waves, instead of three.  This makes the
exact solution to the Riemann quite a bit more expensive.  Both the HLL and HLLC
approximations can be formulated for the MHD equations, and are employed in both
the Dedner and CT solvers in \enzo.  The HLLD
solver includes two of the additional waves, the rotational discontinuities,
making it a five-wave solver.

On rare occasions, high order solutions can cause negative densities or energies.  
Both our PPM and MUSCL solvers employ a Riemann
solver fallback mechanism \citep{Lemaster09}.  If a negative density is found at
a particular interface, the more diffusive HLL Riemann solver is used to compute the fluxes associated with that cell, and the flux update is repeated.



