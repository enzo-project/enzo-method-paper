\subsection{Hydrodynamics: The PPM hydro method}
\label{sec.hydro.ppm}

The primary hydrodynamic method used in \enzo\ is based on the
piecewise parabolic method (PPM) of
\citet{1984JCoPh..54..174C}, which has been
modified for the study of cosmological fluid flows.  
\citet{1984JCoPh..54..174C} describe two variants on this method: a
Lagrangian Remap variant and a Direct Eulerian variant.  Both have
been adapted for \enzo, and the Lagrangian Remap 
variant is described in \citet{1995CoPhC..89..149B}.  Because of its
conservative formulation, however, the Direct Eulerian variant is
preferred for \enzo\ AMR simulations.  We will describe the Direct
Eulerian (PPM-DE) variant here.

We can examine the structure of Equations~(\ref{eq:mass})-(\ref{eq:total_energy})
to obtain a better understanding of the effect of each term.
The derivatives are first order in time
and provide us with the dependent variables that will be evaluated each
time step: $\rhob$, $\vecvb$, and $E$.

The second term on the left hand side of each relation
is due to our choice of Eulerian coordinates and corresponds to
advection, the flow of conserved quantities 
($\rhob$, $\rhob \vecvb$, and $\rhob E$) 
across the mesh.  These will be termed the transport terms.

The first term on the right hand side of Equations~(\ref{eq:momentum}) and
(\ref{eq:total_energy}) comes from our choice of comoving coordinates
(a similar term also appears in Equation~(\ref{eq:dm_velocity}), the velocity 
relation for the dark matter particles).  It does not
appear in the mass conservation equation (\ref{eq:mass}) because
of the comoving density definition.   
We note that these expansion terms could be
eliminated entirely by the proper choice of variables (including time),
although we have not done so here, as they do not constitute a major
source of error.

The remainder of the terms (due to pressure and gravity) are denoted as
source terms.  They drive instabilities and discontinuities in the flow.
We make these distinctions because each of the three
groups of terms have different numerical behaviour and are therefore
treated differently.  The source and transport steps will be solved in
a self-consistent fashion by the PPM scheme while the expansion terms,
which are spatially localized, are split off and solved in a separate step
(as are radiative cooling and heating, if present).  

The remainder of the terms (transport and source) in the fluid equations
involve derivatives.  
Because we are interested in phenomena with no special geometry, we will
restrict discussion to Cartesian coordinates.
Once again we use the concept of operator splitting,
but now apply it spatially.   We can rewrite the one-dimensional (Eulerian) 
versions of Equations~(\ref{eq:mass})--(\ref{eq:total_energy}) without expansion terms
in conservative form as,
\begin{eqnarray}
\frac{\partial \rho}{\partial t}  + \frac{1}{a} \frac{\partial \rho v }{\partial x}    & = &  
     0 \label{eq:mass1d} \\
\frac{\partial \rho v}{\partial t}  + \frac{1}{a} \frac{\partial \rho v^2}{\partial x}   + 
      \frac{1}{a} \frac{\partial p}{ \partial x} & = & 
      \rho \frac{1}{a} \frac{\partial \phi}{ \partial x}  \label{eq:momentum1d}  \\
\frac{\partial \rho E}{\partial t}  + \frac{1}{a} \frac{\partial \rho v E}{\partial x}  & =  &
      \rho v \frac{1}{a} \frac{\partial \phi}{\partial x}. \label{eq:total_energy1d}
\end{eqnarray}

%
Note that $x$ and $v$ refer to the one dimensional comoving position and 
peculiar velocity of the baryonic gas, and $g$ is the acceleration at the cell center.
We write this including the cosmological scale factors, but it reduces to the non-cosmological case with $a = 1$.
These equations are now in a form that can be solved by the split PPM scheme.

% ------------------------------------------------------------

\subsubsection{The piecewise parabolic method in one dimension} % 2.2

We now restrict ourselves to the solution of Equations
(\ref{eq:mass1d})--(\ref{eq:total_energy1d}), plus transport, in one
dimension.  In the Direct Eulerian version of PPM, this is
accomplished by a three-step process.  First, we compute effective
left and right states at each grid boundary by constructing a
piecewise parabolic description of the primary variables ($\rho$, $u$
and $E$) and then averaging over the regions corresponding to the
distance each characteristic wave can travel ($u$, $u-c_s$ and
$u+c_s$, where $c_s$ is the sound speed).
Second, a Riemann problem is solved using these effective left and
right states, and finally the fluxes are computed based on the
solution to this Riemann problem.  This is described in detail in
\citet{1984JCoPh..54..174C}, but we will briefly outline the procedure
here both for completeness and to put the changes we will make in
context.


The Eulerian difference equations are:
%
\newcommand{\avp}[1]{\overline{#1}_{j+1/2}}
\newcommand{\avm}[1]{\overline{#1}_{j-1/2}}
\begin{equation}
\rho_j^{n+1} = \rho_j^{n} + \Delta t \left( 
                      \frac{ \avp{\rho}\avp{v} -  \avm{\rho}\avm{v} } {\Delta x_j}
            \right)
     \label{eq:mass_diff}
\end{equation}
\begin{equation}
\rho_j^{n+1} v_{j}^{n+1} =
       \rho_j^{n+1} v_j^n  + \Delta t \left(
          \frac{ \avp{\rho} \avp{v}^2 - \avm{\rho} \avm{v}^2 + \avp{p} - \avm{p}} {\Delta x_j}
             \right)
     + \frac{ \Delta }{2} g_j^{n+1/2} (\rho_j^n + \rho_j^{n+1}), 
     \label{eq:momentum_diff}
\end{equation}
\begin{eqnarray}
\rho_j^{n+1} E_j^{n+1}  = 
       \rho_j^n E_j^n  & + & \Delta t  \left(
            \frac{  \avp{\rho} \avp{v} \avp{E}  - \avm{\rho} \avm{v} \avm{E}  +
                       \avp{v} \avp{p}    - \avm{v} \avm{p} } {\Delta x_j}
             \right) \nonumber \\
         & & + \frac{ \Delta t }{2} g_j^{n+1/2} (\rho_j^n v_j^n + \rho_j^{n+1} v_j^{n+1} ).
     \label{eq:energy_diff}
\end{eqnarray}


We have used subscripts to indicate zone-centered ($j$)
and face-centered ($j+1/2$) quantities, while superscripts refer to
position in time.  The cell width is $\Delta x_j$.
Although they have been discretized
in space, the accuracy of the updates depend on how well we can compute
the fluxes into and out of the cell during $\Delta t$.  
This in turn depends on our ability to compute the time-averaged 
values of $p$, $\rho$ and $v$ at the cell interfaces, denoted here
by $\overline{p}_{j\pm 1/2}$, $\overline{\rho}_{j\pm 1/2}$, and $\overline{v}_{j\pm 1/2}$.  We now
describe the steps required to compute these quantities.
%Note that these difference equations are conservative.

We first construct monotonic piecewise 
parabolic (third order) interpolations in one dimension
for each of $p$, $\rho$, and $v$  The pressure is determined from
Equation~(\ref{eq:eq_of_state}), the equation of state.
%We have dropped the subscript b in this and the following two sections because we will be referring solely to baryonic quantities.  
The interpolation formula for some quantity $q$ is given by:
%
\begin{eqnarray}
q_j(x) & = &  q_{L,j} + \tilde{x}(\Delta q_j + q_{6,j}(1-\tilde{x})), \\
\tilde{x}      & \equiv & {x - x_{j-1/2} \over \Delta x_j}, \qquad
             x_{j-1/2} \leq x \leq x_{j+1/2}. \nonumber
\end{eqnarray}
%
This is Equation~(1.4) of CW84 (in the radial rather than mass coordinate as is appropriate for the direct Eulerian approach). 
%The zones edges in the mass coordinates are $m_{j-1/2}$ and $m_{j+1/2}$ for the left and right edges, respectively.  
The quantities $q_{L,j}$, $\Delta q_j$,
and $q_{6,j}$ can be viewed simply as interpolation constants; however,
they also have more intuitive meanings.  For example, $q_{L,j}$ is the
value of $q$ at the left edge of zone j, while $\Delta q_j$ and $q_{6,j}$ are analogous to the slope and first order correction to the slope of $q$ (see \citet{1984JCoPh..54..174C} for a complete discussion):
\begin{equation}
\Delta q_j \equiv q_{R,j} - q_{L,j} \qquad 
q_{6,j}    \equiv 6\left[q_j - 1/2\left(q_{L,j} + q_{R,j}\right)\right].
\end{equation}

We have reduced the problem to finding $q_{L,j}$ and $q_{R,j}$.  While this
is simple in principle, it is complicated somewhat by the requirement that
these values be of sufficient accuracy and that the resulting distribution
be monotonic.  That is, no new maxima or minima are introduced.
The resulting formulae are straightforward but complicated and are not
reproduced here, but see Equations 1.7 to 1.10 of \citet{1984JCoPh..54..174C}.
We also optionally allow steepening as described in this reference.

In the Lagrangian case, once we have the interpolation constants we compute the characteristic domain for each zone edge.  This is simply the farthest a sound wave could travel in order to reach the interface by the end of a timestep. The characteristic domain encompasses all of the information that can reach the (Lagrangian) zone edge during the current timestep.  This information would then be fed into the Riemann solver to find the flux at each Lagrangian cell boundary.

In the Eulerian approach, this is more complex because the Eulerian
zone boundary does not naturally correspond to one the three characteristic waves and so a more complicated approach is required.  First, the primary quantities ($p$, $\rho$, $v$ and $E$) are averaged over the domains corresponding to the three characteristics $C = u-c$, $u$ or $u+c$ (where $c$ is the sound speed in a cell):
\begin{eqnarray}
      q^{C+}_{j+1/2} & = & f_{j+1/2,L}(\Delta t C^n_j), \\
      q^{C-}_{j+1/2} & = & f_{j+1/2,R}(-\Delta t C^n_{j+1}), \nonumber
\end{eqnarray}
where $q^{C+}$ ($q^{C-}$) refers to the left (right) characteristic $C$.  The functions $f$ are just integrals
over the parabolic interpolation formula:
%
\begin{eqnarray}
f_{j+1/2,L}(y) & = &
       {1 \over y} \int^{x_{j+1/2}}_{x_{j+1/2}-y} q_j(x) dx, \\
f_{j+1/2,R}(y) & = &
       {1 \over y} \int^{x_{j+1/2}+y}_{x_{j+1/2}} q_{j+1}(x) dx. \nonumber
\end{eqnarray}
The linearized gas dynamics equations are then used to compute second-order correct left and right states that take into account the multiple wave families.  This process is described in \citet{1984JCoPh..54..174C} and we use their equations (3.6) and (3.7).

With these effective states, an approximation to the Riemann problem is found
with an iterative approach \citep[see][]{Woodward86}, producing estimates for 
$\overline{p}_{j\pm 1/2}$, $\overline{\rho}_{j\pm1/2}$, and $\overline{v}_{j\pm 1/2}$ that are third order
accurate in space and second order accurate in time.  These are then
used to solve the difference Equations (\ref{eq:mass_diff})--(\ref{eq:energy_diff}) for $\rho^{n+1}$, $v^{n+1}$, and $E^{n+1}$.

We include an optional diffusive flux (and flattening for the parabolic curves) that can improve the solution in some cases.  Our implementation follows that in the appendix of \citet{1984JCoPh..54..174C}.

In addition, as discussed earlier, the three dimensional scheme is achieved by operator splitting and repeating the above procedure in the other two orthogonal directions.  The transverse velocities and any additional passive quantities are naturally and easily added to this system (see Equation 3.6 of \citet{1984JCoPh..54..174C}).  

We note that the acceleration required in Equation~(\ref{eq:momentum_diff}) is actually the
acceleration felt by the entire zone and not just at the zone center.
Therefore, it is possible to find the mass weighted average acceleration over the zone
by expanding the density and acceleration distributions and
retaining all terms up to second order in $\Delta x$:
%
\begin{equation}
g_j^{n+1/2} = 
       \frac{1}{2 a^{n+1/2} \delta x_j} \left[ 
             \phi_{j+1}^{n+1/2} 
           - \phi_{j-1}^{n+1/2} 
           + \frac{1}{12} \left(    \phi^{n+1/2}_{j+1} 
                                 - 2\phi^{n+1/2}_j 
                                 + \phi^{n+1/2}_{j-1} \right) 
                                   {\delta d_j \over d_j}
       \right].
       \label{eq:accel}
\end{equation}
Generally we find that the potential is so slowly varying that this change is unnecessary.

% ------------------------------------------------------------

\subsubsection{The dual energy formulation for  very high Mach flows} % 2.4

The system described thus far works well for gravitating systems with
reasonable Mach numbers ($<100$) as long as the structures are well resolved.
This section and the next detail changes that are required
to correctly account for situations where one or both of these
requirements are not met.

Large, hypersonic bulk flows appear to be very common in cosmological
simulations and they present a problem because of the high ratio of
kinetic energy $E_k$ to gas internal energy $e$, which can reach as
high as $10^8$.  Inverted, we see that the internal energy consists of
an extremely small portion of the total energy.  The pressure then,
proportional to $E - E_{k}$, is the small difference between two large
numbers: a disastrous numerical situation.  This is not as large a
problem as it may at first appear because it only occurs when the
pressure is negligibly small.  Therefore, even if we suffer large
errors in the pressure distribution in these regions, the dynamics and
total energy budget of the flow will remain unaffected.  Nevertheless,
if the temperature distribution is required for other reasons
(e.g., for calculating radiative processes), a remedy is required.

To overcome this, we also solve the internal energy equation:
\begin{equation}
 \frac{\partial e}{\partial t} 
           + \frac{1}{a} \vecv \cdot \grad e
%         = - \frac{3(\gamma -1)\dot{a}}{a} e
         = - 3 \frac{\dot{a}}{a} \frac{p}{\rho}
           - \frac{p}{a\rho} \div \vecv
\end{equation}
in comoving coordinates.  The structure is similar to the total energy
equation; the second term on the left hand side represents transport, while
the first term on the right is due to expansion of the coordinate
system.  
It is differenced (again, in Eulerian form without the expansion term) as,
%
\begin{equation}
\rho_j^{n+1} e_j^{n+1}  = 
       \rho_j^n e_j^n   +  \Delta t  \left(
            \frac{  \avp{\rho} \avp{v} \avp{e}  - \avm{\rho} \avm{v} \avm{e}} {\Delta x_j} \right)
            - \Delta t \ p_j^{n} \left( \frac{ \avp{v} - \avm{v}} {\Delta x_j} 
                      \right)
                  \label{eq:gasenergy_diff}
\end{equation}
%
Note that because of the structure of this equation, it is not in
flux-conservative form.  In particular, the pressure is evaluated at
the cell center.  Unfortunately, time-centering of this pressure has
proved difficult to do without generating large errors in the internal
energy and so we conservatively leave the pressure at the old time in
this difference equation.  This leads to spreading of shocks; however,
we note that this equation is only used in hypersonic flows.

It is necessary, however, to conserve the total energy so that the conversion
of kinetic to thermal energy is performed properly.   We must therefore,
combine the two formulations without allowing the separately advected
internal energy $e$ to play a role in the gas dynamics.  This is done by
carrying both terms through the simulation and using
the total energy $E$ for hydrodynamic routines and the internal energy
$e$ when the temperature profile is required.  One way to view this procedure
is to treat $e$ as enhanced precision (extra digits) for $E$ that automatically
`floats' to where it is needed.
We only require that they be kept synchronized when the two levels of precision
overlap.

When the pressure is required solely for dynamic purposes, the 
selection criterion 
operates on a cell by cell basis using,
\begin{equation}
p = \cases{ \rho(\gamma - 1)(E - \vecv^2/2),& 
                  $  (E - \vecv^2/2)/E > {\eta}_1 $; \cr
            \rho(\gamma -1)e,&
                  $  (E - \vecv^2/2)/E < {\eta}_1 $. \cr}
\end{equation}

It should be stressed that as long as the parameter ${\eta}_1$ is small enough
the dual energy method {\it will have no dynamical effect}.  We use
${\eta}_1 = 10^{-3}$, which is consistent with the truncation error of the
scheme for grid sizes that are typically used in our simulations.  We are now free to select the method by which the internal energy
field variable $e$ is updated so that it will not become contaminated with
errors advected by the total energy formulation but still give the
correct distribution in shocked regions. 
Since we are concerned with the advection of errors, the selection
criterion must look at each cell's local neighbourhood.
In one dimension, this is done with,
\begin{equation}
e = \cases{ (E - \vecv^2/2),& $\rho(E - \vecv^2/2)/
    \max ({\rho}_{j-1}E_{j-1},{\rho}_j E_j,{\rho}_{j+1}E_{j+1}) > {\eta}_2$,\cr
            e,& $\rho(E - \vecv^2/2)/
    \max ({\rho}_{j-1}E_{j-1},{\rho}_j E_j,{\rho}_{j+1}E_{j+1}) < {\eta}_2$.\cr}
    \label{eq:dualselect}
\end{equation}
Thus, ${\eta}_2$ determines when the synchronization (of $e$ with $E$) occurs.
%is a measure of how much error we will allow to be passed
%from neighboring cells.  
Too high a value may mask relatively weak shocks, 
while spurious heating (via contamination) may occur if it is set too low.
After some experimentation,
we have chosen ${\eta}_2 = 0.08$, a somewhat conservative value.
This scheme is optional and is generally only required in large-scale 
cosmological simulations where the gas cools due to the expansion of the universe
but large bulk flows develop due to the formation of structure.

We note that others have independently developed a similar but
distinct scheme for dealing with this problem, which is endemic to
methods adopting the total energy equation.  In \citet{TVD93} the 
two variables adopted are total energy and
entropy (rather than total energy and thermal energy), with an
analogous scheme for choosing which variable to employ.

\subsubsection{Riemann Solvers and Fallback}

While most numerical hydrodynamics schemes center around finding approximate
solutions to the hydrodynamics equations, the core of philosophy of a Godunov's
method \citep{Godunov1959} is to find the exact solution to an
approximation of the physical setup.  The approximate problem in this case is
the Riemann problem, wherein two volumes are separated by a thin membrane that
is instantaneously removed at time $t=0$.  The subsequent evolution has an exact 
analytic solution.  This solution is
described in detail in many texts on computational fluid dynamics
\citep[e.g.,][]{toro-1997}.   In brief, there are three waves that propagate
away from the initial discontinuity.  The central wave, characterized by a
density jump but not a pressure jump, is called the contact discontinuity.  The
waves traveling to the left and right of the contact discontinuity can be either
shocks, if characteristics converge on the wave front, or  rarefaction fans if
characteristics diverge.  

While there exists an exact solution to this problem, finding it is expensive.  There are four
possible combination of left- and right- traveling shocks and rarefactions,
only one of which is fully consistent with the initial conditions.  Once the
correct physical state is determined, the pressure in the central region can
only be found by finding the root to an algebraic equation, which is necessarily
an iterative process.  Thus a series of \emph{approximate} Riemann solvers are
typically used.  There are four approximate Riemann solvers in \enzo: two-shock
\citep{toro-1997},
Harten-Lax-van Leer \citep[HLL,][]{toro-1997}, HLL with a contact discontinuity
 \citep[HLLC,][]{toro-1997}, and HLL with
multiple discontinuities \citep[HLLD,][]{Miyoshi05}.  Two-shock is used only
with the PPM method.  HLL and HLLC are used with PPM, MUSCL (both with and
without MHD) and MHD-CT.  HLLD is exclusively an MHD solver, and works with both
MUSCL and MHD-CT methods.  

The only approximation that two-shock makes is that both left- and right- moving
waves are shocks.  This solution still requires an iterative method for finding
the pressure in between the two waves.  The HLL method alleviates this iteration
by assuming that there is no central contact discontinuity, and the signal speed
in the central region is approximated by an average over the left- and right-
moving waves.  This method is significantly faster than the two-shock method,
but also quite a bit more dissipative.  The HLLC is a three-wave method that improves upon the HLL
method by also including the third wave, the contact discontinuity.  

For the MHD equations, there are seven waves, instead of three.  This makes the
exact solution to the Riemann quite a bit more expensive.  Both the HLL and HLLC
approximations can be formulated for the MHD equations, and are employed in both
Dedner and CT solvers in \enzo.  The HLLD
solver includes two of the additional waves, the rotational discontinuities,
making it a five-wave solver.

On rare occasions, high order
solutions can cause negative densities or energies.  Both PPM and MUSCL solvers employ a Riemann
solver fallback mechanism \citep{Lemaster09}.  If a negative density is found at
a particular interface, the more diffusive HLL Riemann solver is used instead.  

% ------------------------------------------------------------


% GB: I commented the following out because it has been incorporated into the above
%
%PPM is an explicit, higher order-accurate version of
%Godunov's method for ideal gas dynamics with third order-accurate piecewise parabolic
%monotonic interpolation and a nonlinear Riemann solver for shock
%capturing.  It advances the hydrodynamic equations in the following
%steps:
%\begin{enumerate}
% \item Construct monotonic parabolic interpolation of cell average
% data, for each fluid quantity.
% \item Compute interface states by averaging the parabola for over the domain of dependence for
% each interface
% \item Use interface data to solve the Riemann problem.
% \item Difference the interface fluxes to update the cell average quantities.
%\end{enumerate}
%It does an excellent job of capturing strong shocks in at
%most two cells.  Multidimensional schemes are built up by directional
%splitting and produce a method that is formally second order-accurate
%in space and time which explicitly conserves mass, linear momentum, and energy.  

%To use the above algorithm for cosmological simulations, the
%conservation laws for fluid mass, momentum and energy 
%density are written in comoving coordinates for a
%Friedman-Robertson-Walker space-time, as described previously 
%in Equations~\ref{enzoconserve} 
%through~\ref{enzoenergy}.  Both the conservation laws and
%the Riemann solver are modified to include gravity, which is solved
%using an adaptive particle-mesh (PM) technique (see Section~\ref{sec.ov.nbody}).  
%The terms due to cosmological expansion, as well as primordial chemistry and
%radiative heating and cooling, are solved in a separate step because they have
%different numerical behavior, and therefore must be treated differently to ensure
%stability.  Note that unlike the \zeus\ hydro scheme, PPM does not need to use 
%artificial viscosity to resolve shocks.

%The system of equations described above works well for systems with
%relatively low Mach numbers, as long as these systems are well resolved.  
%However, cosmology is replete with situations where there are bulk hypersonic
%flows.  In these situations, the ratio of kinetic to thermal energy can be
%very high -- in some situations up to $10^6 - 10^8$.  This implies that the
%thermal energy is an extremely tiny fraction of the kinetic energy, which can 
%cause numerical problems when one is interested in just the thermal energy of the gas, since
% Equation~\ref{enzoenergy} solves for the total energy.  In this system of
%equations, the
% thermal energy $E_{therm}$ is calculated as $E - E_{kin}$, 
%where E is the total specific energy as
%calculated in equation~\ref{enzoenergy} and $E_{kin}$ is the specific kinetic energy,
%$0.5 \vecv_b^2$.  
%In hypersonic flows $E$ and $E_{kin}$ are nearly the same, and any 
%number calculated as 
%the difference of these is going to be strongly affected by numerical error.  To 
%avoid this problem, \enzo\ also solves the internal energy equation in
%comoving coordinates:

%\begin{equation}
%\frac{\partial e}{\partial t} + \frac{1}{a} \vecv_b \cdot \grad e = 
%- \frac{3(\gamma-1)\dot{a}}{a}e - \frac{p}{a \rho} \div \vecv_b
%\label{enzointenergy}
%\end{equation}

%In this equation $e$ is the internal energy (defined as $P/\rho (\gamma-1)$ and the other terms are as described
%previously.  The code still conserves total energy ($E$) as well.  In order
%to maintain consistency, both equations are solved at all times in all cells,
%with the equation for the total energy (eqtn.~\ref{enzoenergy}) being used
%for hydrodynamics routines and the internal energy $e$ being used when 
%temperature is required.  When pressure is required for dynamic purposes,
%the total energy is used if the ratio of thermal energy to total energy
%is less than some threshold value $\eta$, and the internal energy is used
%for values of the ratio larger than $\eta$.  A typical value of this parameter
%is $10^{-3}$.  This \emph{dual energy formulation} ensures that the
%method produces the correct entropy jump at strong shocks and also
%yields accurate pressures and temperatures in cosmological hypersonic
%flows.


