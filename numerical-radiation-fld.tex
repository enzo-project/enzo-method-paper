\subsection{Radiation transport: Flux-limited diffusion}
\label{sec.num.rad-fld}

%% With regards to the numerical method section, we're really looking for
%% something that's no more than a page long: a few relevant equations,
%% maybe a finite difference equation or two if you feel that it's
%% relevant, but primarily a description of the algorithms used for the
%% FLD solve and a brief explanation of how it's coupled to the
%% chemistry, as well as references to the relevant method paper(s).
%% Working in a reference or two to science papers is also great! 

In addition to the ray-tracing approach for radiation transport
described in Section \ref{sec.num.raytracing}, Enzo currently includes
a field-based radiation transport solver for problems posed on uniform
(i.e.~non-SAMR grids).  Detailed explanations of the model and
solution approach may be found in
\cite{NBHBROW2007}, \cite{ReynoldsHayesPaschosNorman2009} and
\cite{NRS2009}, the salient features of which are reproduced here. 
In addition, comparisons of this solver with other astrophysical
radiation transport solvers may be found in \cite{IlievEtAl2009}.  

Enzo's field-based radiation solver focuses on
a flux-limited diffusion approximation for cosmological radiative
transfer, with couplings to both the gas energy and chemical number
densities.
%\begin{eqnarray}
%  \label{eq:fld_radiation}
%  \partial_t E_r + \frac1a \nabla\cdot\left(E_r \vec{v}_b\right) &=& 
%    \nabla\cdot\left(D\nabla E_r\right) -
%    \frac{\dot{a}}{a} E_r - c\kappa E_r + \eta, \\
%  \label{eq:fld_heating}
%  \partial_t e_c &=& -\frac{2\dot{a}}{a} e_c + G - \Lambda, \\
%  \label{eq:fld_chemistry}
%  \partial_t {\mathrm n}_i + 
%    \frac1a \nabla\cdot\left({\mathrm n}_i \vec{v}_b\right) &=& 
%    \alpha_{i,j} {\mathrm n}_e {\mathrm n}_j - {\mathrm n}_i
%    \Gamma_i^{ph}, \quad i=1,\ldots,N_s,
%\end{eqnarray}
%where $E_r$ is a grey radiation energy density, $e_c$ is the
%internal energy correction due to photo-heating and chemical cooling,
%and ${\mathrm n}_i$ correspond to the chemical number densities of
%HI, HII, HeI, HeII and HeIII, and ${\mathrm n}_e$ is the electron
%number density.  Here, we define the grey radiation energy density
%through first assuming a fixed frequency spectrum, i.e.
%$E_{\nu}(\nu,\vec{x},t) = \tilde{E}_r(\vec{x},t) \chi(\nu)$, and then
%defining the integrated quantity
%\begin{equation}
%\label{eq:grey_radiation_energy}
%   E_r(\vec{x},t) \equiv \int_{\nu_0}^{\infty}
%   E_{\nu}(\nu,\vec{x},t)\,\mathrm d\nu \  = \ 
%   \tilde{E}_r(\vec{x},t) \int_{\nu_0}^{\infty} \chi(\nu)\,\mathrm d\nu.
%\end{equation}
%In addition, $D$ in (\ref{eq:fld_radiation}) is the {\em Larsen
%square-root flux-limiter} (see \cite{Morel2000}), $\kappa$ is the total
%opacity, $\eta$ is the field of radiation sources, $G$ provides the
%radiation induced photo-heating, $\Lambda$ provides the chemical
%cooling, $\alpha_{i,j}$ are the reaction rate coefficients defining
%the interactions between species ${\mathrm n}_i$ and ${\mathrm n}_j$,
%and $\Gamma_i^{ph}$ are the radiation induced photo-ionization rates. 

The system of equations
(\ref{eq:fld_radiation}-\ref{eq:fld_heating}) along with the chemical network 
(eq.~\ref{eq:species_evolution} is solved
independently of Enzo's hydrodynamics, gravity and dark-matter solvers
(sections \ref{sec.hydro.ppm}-\ref{sec.ov.nbody}), thereby allowing
the advective portions of equations (\ref{eq:fld_radiation}) and
(\ref{eq:fld_chemistry}) to be taken care of by the fluid solvers.
Due to the disparate time scales between radiation transport and chemical
ionization and heating, the remainder of these equations is solved
using an operator-split algorithm.  Within a given time step to evolve
$(E_r^n, e_c^n, {\mathrm n}_i^n) \to (E_r^{n+1}, e_c^{n+1}, {\mathrm
n}_i^{n+1})$, we first evolve equation (\ref{eq:fld_radiation}):
$(E_r^n, e_c^n, {\mathrm n}_i^n) \to (E_r^{n+1}, e_c^{n}, {\mathrm
  n}_i^{n})$.  This uses an implicit Euler time discretization, and a
second-order centered finite-difference spatial discretization,
resulting in a large linear system of equations.  These are solved
using a preconditioned conjugate gradient iteration, where the
preconditioner consists of a geometric multigrid solver.  Both of
these linear solvers are provided by the HYPRE linear solver library
(see \cite{FalgoutYang2002} and \cite{hypre-manual}). 

We then evolve the system (\ref{eq:fld_heating}-\ref{eq:fld_chemistry}):  
$(E_r^{n+1}, e_c^n, {\mathrm n}_i^n) \to (E_r^{n+1}, e_c^{n+1},
{\mathrm n}_i^{n+1})$.  Due to the lack of spatial derivatives (since
advection is handled elsewhere), this system is a coupled system of
nonlinear ordinary differential equations.  This utilizes an implicit
quasi-steady-state approximation, formulated as follows.  We consider
the modified equations,
\begin{eqnarray}
  \label{eq:fld_heating_qss}
  \frac{\partial e_c}{\partial t} &=& -\frac{2\dot{a}}{a} e_c +
    \Gamma\left(\bar{E}_r,\bar{\mathrm n}_i\right) - 
    \Lambda\left(\bar{E}_r,\bar{\mathrm n}_i\right), \\
  \label{eq:fld_chemistry_qss}
  \frac{\partial {\mathrm n}_i}{\partial t} &=& k_{i,j}\left(\bar{e}_c\right)
    {\mathrm n}_e \bar{\mathrm n}_j - {\mathrm n}_i 
    \Gamma_i^{ph}\left(\bar{E}_r\right), \quad i=1,\ldots,N_s,
\end{eqnarray}
where we have defined the time-centered ``background'' states
$\bar{E}_r = \left(E_r^{n}+E_r^{n+1}\right)/2$, 
$\bar{\mathrm n}_i = \left({\mathrm n}_i^{n}+{\mathrm n}_i^{n+1}\right)/2$
and $\bar{e}_c = \left(e_c^{n}+e_c^{n+1}\right)/2$.  These equations
may each be solved {\em analytically} for their solution at the time
$t$, which we denote by
\begin{eqnarray}
  \label{eq:fld_heating_qss_sol}
  e_c(t) &=& \text{sol}_{e}\left(t,\bar{E}_r,\bar{\mathrm n}_i,e_c^n\right), \\
  \label{eq:fld_chemistry_qss_sol}
  {\mathrm n}_i(t) &=& \text{sol}_{\mathrm n_i}
  \left(t,\bar{E}_r,\bar{e_c},\mathrm n_i^n\right), \quad i=1,\ldots,N_s. 
\end{eqnarray}
We then define a nonlinear system of equations to compute the
time-evolved solutions $\left(E_r^{n+1}, e_c^{n+1}, 
{\mathrm n}_i^{n+1})\right)$ as
\begin{eqnarray}
  \label{eq:fld_heating_qss_fe}
  f_e(e_c^{n+1},{\mathrm n}_i^{n+1}) &\equiv& e_c^{n+1} -
    \text{sol}_{e}\left(t^{n+1},\bar{E}_r,\bar{\mathrm n}_i,e_c^n\right)
    = 0, \\
  \label{eq:fld_chemistry_qss_fn}
  f_{\mathrm n_i}(e_c^{n+1},{\mathrm n}_i^{n+1}) &\equiv& 
    {\mathrm n}_i(t) - \text{sol}_{\mathrm n_i}
    \left(t,\bar{E}_r,\bar{e_c},\mathrm n_i^n\right)=0, \quad
    i=1,\ldots,N_s.  
\end{eqnarray}
This system of $N_s+1$ nonlinear equations is solved using a damped
fixed point iteration, 
\[
   U_i = U_i - \lambda f_i(U), \quad i=1,\ldots,N_s+1,
\]
where $U$ is a vector containing the solutions to equations
(\ref{eq:fld_heating_qss_fe}-\ref{eq:fld_chemistry_qss_fn}).  In this
iteration, for the first 50 sweeps we use $\lambda=1$, and for more
challenging problems if that does not suffice we switch to a damping
parameter of $\lambda=0.1$.
