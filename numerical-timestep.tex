\subsection{Time Stepping}
\label{sec.timestepping}

%\dcc{Made this its own section (not really part of the AMR, it's more
%  general than that.)  Added stuff on harmonic average for 3d. See
%  note below on acceleration timestep criterion being crap. Moved
%  w-cycle to the previous section}

In \enzo, resolution of the equations being solved is adaptive in time 
as well as in space.  The timestep in \enzo\ is satisfied on a level-by-level
 basis by finding the largest timestep such that multiple criteria are
satisfied on each level.  The timestep criteria used by Enzo are 
(showing the one-dimensional case for clarity):

\begin{equation}
\Delta t_{hydro} = min \left( \kappa_{hydro} \frac{a \Delta x}{c_{s} + |v_x|} \right)_L ,
\label{eqn:dthydro}
\end{equation}

\begin{equation}
\Delta t_{MHD} = min \left( \kappa_{MHD} \frac{a \Delta x}{v_{f} + |v_x|} \right)_L ,
\label{eqn:dtMHD}
\end{equation}

\begin{equation}
\Delta t_{dm} = min \left(\kappa_{dm} \frac{a \Delta x}{v_{dm,x}} \right)_L ,
\label{eqn:dtdarkmatter}
\end{equation}

\begin{equation}
\Delta t_{accel} = min \left( \sqrt{\frac{\Delta x}{\vec{g}}} \right)_L ,
\label{eqn:dtaccel}
\end{equation}

\begin{equation}
\Delta t_{rad} = min \left(  \sqrt{\frac{\Delta x}{\vec{a}_{rad}} \right)_L,
\label{eqn:dtrad}
\end{equation}

\begin{equation}
\Delta t_{cond} = min \left(  \kappa_{cond} \frac{\Delta x^2
    n}{\kappa(T)} \right)_L,
\label{eqn:dtcond}
\end{equation}

\begin{equation}
\Delta t_{exp} = f_{exp} \left( \frac{a}{\dot{a}} \right) ,
\label{eqn:dtexpand}
\end{equation}

% \begin{equation}
% \Delta t_{} = 
% \label{eqn:dt}
% \end{equation}

% \begin{equation}
% \Delta t_{} = 
% \label{eqn:dt}
% \end{equation}

 In equations~\ref{eqn:dthydro}-\ref{eqn:dtcond}, the $min ( \ldots
)_L$ formalism means that this value is calculated for all cells on a
given level L and the minimum overall value is taken as the timestep.
Equation~\ref{eqn:dthydro} ensures that all cells satisfy the
Courant-Freidrichs-Levy (CFL) condition for accuracy and stability of
an explicit finite difference discretization of the Euler equations.
Effectively this condition forces the timestep to be small enough such
that any changes in the fluid propagate less than a single grid
spacing, $\Delta x$.  In this equation, $\kappa_{hydro}$ is a
numerical constant with a value of $0 < \kappa_{hydro} \leq 1$ (with a
typical value of $\kappa_{hydro} \sim 0.3-0.5$) that ensures that the
CFL condition is always met, and $c_s$ and $v_x$ are the sound speed
and peculiar baryon velocity in a given cell.

Equation ~\ref{eqn:dthydro} is valid for one dimension.  For 2 or 3
dimensions, it was shown by \cite{Godunov1959}  that using the
harmonic average of the timestep found along each of the coordinate
axes yields a maximum $\kappa_{hydro} = 0.8$.  So letting $\Delta
t_x$, $\Delta t_y$, and $\Delta t_z$ be the analogues of
eqn. \ref{eqn:dthydro} along the $x,y$ and $z$ axes, 
\begin{equation}
\Delta t_{hydro} = min ( \kappa_{hydro} /( \frac{1}{\Delta t_x}
+\frac{1}{\Delta t_y} + \frac{1}{\Delta t_z} ) )_L
\end{equation}

For all other criteria, multiple dimensions are accounted for by
repeating the one dimensional criterion along each axis, and taking
the minimum.

Equation~\ref{eqn:dtMHD} is only enforced when the equations of
magnetohydrodynamics are being solved, and is directly analogous to
Equation~\ref{eqn:dthydro} in that it ensures that the CLF condition
is being enforced at all times.  In this equation, $\kappa_{MHD}$ is a
numerical constant with a value of $0 < \kappa_{MHD} \leq 1$ (with a
typical value of $\kappa_{MHD} \sim 0.5$) that ensures that the CFL
condition is always met, and $v_f$ and $v_x$ are the ``fast wave
speed'' and peculiar baryon velocity in a given cell.  The fast wave
speed comes from a stability analysis of the MHD equations, and is
expressed as:

\begin{equation}
v_f = \sqrt{ \frac{1}{2} \left(  v_A^2 + c_s^2 + \sqrt{(v_A^2 +
      c_s^2)^2 - 4 v_A^2 c_s^2}  \right)  },
\label{eqn:vfastmhd}
\end{equation}

Where $c_s$ is the sound speed and $v_A$ is the Alfven speed, calculated
as $v_A = \sqrt{B^2/\rho_B}$ in units where $\mu_0 = 1$.

Equation~\ref{eqn:dtdarkmatter} is analogous to
Equation~\ref{eqn:dthydro} and ensures accuracy in the N-body solver
by requiring that no dark matter particle travels more than one cell
width.  The parameter $\kappa_{dm}$ is analogous to
$\kappa_{dthydro}$, with an identical range of values.
Equation~\ref{eqn:dtexpand} limits the timestep such that the
expansion parameter a can only change by a fractional amount of
$f_{exp} = \Delta a/a$, where $f_{exp}$ is a user-defined parameter
and has typical values of $f_{exp} = 0.01-0.02$.  This is required for
the stability of the PPM algorithm in comoving coordinates, and
typically limits the timestep only at high redshifts when densities
are relatively homogeneous.

Equation~\ref{eqn:dtaccel} and~\ref{eqn:dtrad} are supplementary to equation~\ref{eqn:dthydro} in that they
take into account the possibility of large accelerations due to either
gravity (Eqn.~\ref{eqn:dtaccel} or radiation pressure
(Eqn.~\ref{eqn:dtrad}), which may cause numerical 
instabilities by violating the Courant condition.  In Equation~\ref{eqn:dtaccel}, $\vec{g}$ is the
gravitational acceleration in each cell on level L.  In
Equation~\ref{eqn:dtrad}, $\vec{a}_{rad}$ is the estimated
acceleration due to radiation pressure in each cell on level L,
defined as

\begin{equation}
\vec{a}_{rad} = \frac{ \sum_i \frac{\dot{E_i}}{c} \hat{r_i} }{m_b} 
\end{equation}

Where the sum calculates the energy deposited in a cell during the
previous timestep due to \textit{all photon packets} that crossed that
cell, with $\hat{r_i}$ being a unit vector that accounting for the
packet direction.

Equation~\ref{}