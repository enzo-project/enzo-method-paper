\section{Timestepping}
\label{sec.timestepping}

In \enzo, the integration of the equations being solved is generally adaptive in time 
as well as in space.  The timestep $\Delta t$ is set on a level-by-level
basis by finding the largest timestep such that all of the criteria listed below are
satisfied.  The timestep criteria are given by the following expressions
(where we show the one-dimensional case for clarity):

\begin{equation}
\Delta t_{\rm hydro} = \min \left( \kappa_{\rm hydro} \frac{a \Delta x}{c_{s} + |v_x|} \right)_L ,
\label{eqn:dthydro}
\end{equation}

\begin{equation}
\Delta t_{\rm MHD} = \min \left( \kappa_{\rm MHD} \frac{a \Delta x}{v_{f} + |v_x|} \right)_L ,
\label{eqn:dtMHD}
\end{equation}

\begin{equation}
\Delta t_{\rm dm} = \min \left(\kappa_{\rm dm} \frac{a \Delta x}{v_{dm,x}} \right)_L ,
\label{eqn:dtdarkmatter}
\end{equation}

\begin{equation}
\Delta t_{\rm accel} = \min \left( \sqrt{\frac{\Delta x}{\vec{g}}} \right)_L ,
\label{eqn:dtaccel}
\end{equation}

\begin{equation}
\Delta t_{\rm rad} = \min \left(  \sqrt{\frac{\Delta x}{\vec{a}_{rad}}} \right)_L,
\label{eqn:dtrad}
\end{equation}

\begin{equation}
\Delta t_{\rm cond} = \min \left(  \frac{ \kappa_{\rm cond}}{f_{\rm sp}} \frac{\Delta x^2
    n_b}{\kappa(T)} \right)_L,
\label{eqn:dtcond}
\end{equation}

\begin{equation}
\Delta t_{\rm exp} = f_{\rm exp} \left( \frac{a}{\dot{a}} \right) ,
\label{eqn:dtexpand}
\end{equation}

% \begin{equation}
% \Delta t_{} = 
% \label{eqn:dt}
% \end{equation}

% \begin{equation}
% \Delta t_{} = 
% \label{eqn:dt}
% \end{equation}

In equations~(\ref{eqn:dthydro})-(\ref{eqn:dtcond}), the $\min ( \ldots)_L$ 
formalism means that this value is calculated for all cells or
particles on a given level l and the minimum overall value is taken as the timestep.

Equation~(\ref{eqn:dthydro}) ensures that all cells satisfy the
Courant-Freidrichs-Levy (CFL) condition for accuracy and stability of
an explicit finite difference discretization of the Euler equations.
%Effectively this condition forces the timestep to be small enough such
%that any changes in the fluid propagate less than a single grid
%spacing, $\Delta x$.  
In this equation, $\kappa_{\rm hydro}$ is a
numerical constant with a value of $0 < \kappa_{\rm hydro} \leq 1$
(with a typical value of $\kappa_{\rm hydro} \sim 0.3-0.5$) that
ensures that the CFL condition is always met, and $c_s$ and $v_x$ are
the sound speed and peculiar baryon velocity in a given cell.

Equation~(\ref{eqn:dthydro}) is valid for one dimension.  For 2 or 3
dimensions, it was shown by \cite{Godunov1959}  that using the
harmonic average of the timestep found along each of the coordinate
axes yields a maximum $\kappa_{\rm hydro} = 0.8$.  So letting $\Delta
t_x$, $\Delta t_y$, and $\Delta t_z$ be the analogues of
equation~(\ref{eqn:dthydro}) along the $x,y$ and $z$ axes, 
\begin{equation}
  \Delta t_{\rm hydro} = \min \left( \frac{\kappa_{\rm hydro}} {1/\Delta t_x
  +1/\Delta t_y + 1/\Delta t_z} \right)_L
\end{equation}

For all other criteria except for equation~(\ref{eqn:dtMHD}), multiple dimensions are accounted for by
repeating the one dimensional criterion along each axis, and taking the minimum.

Equation~(\ref{eqn:dtMHD}) is only enforced when the equations of
magnetohydrodynamics are being solved, and is directly analogous to
equation~(\ref{eqn:dthydro}) in that it ensures that the CFL condition
is being enforced at all times.  In this equation, $\kappa_{\rm MHD}$ is a
numerical constant with a value of $0 < \kappa_{\rm MHD} \leq 1$ (with a
typical value of $\kappa_{\rm MHD} \sim 0.5$) that ensures that the CFL
condition is always met, and $v_f$ and $v_x$ are the ``fast wave
speed'' and peculiar baryon velocity in a given cell.  The fast wave
speed comes from a stability analysis of the MHD equations, and is
given by:
%
\begin{equation}
v_f = \sqrt{ \frac{1}{2} \left(  v_A^2 + c_s^2 + \sqrt{(v_A^2 +
      c_s^2)^2 - 4 v_A^2 c_s^2}  \right)  },
\label{eqn:vfastmhd}
\end{equation}
%
where $c_s$ is the sound speed and $v_A$ is the Alfven speed, calculated
as $v_A = \sqrt{B^2/\rho_B}$ in units where $\mu_0 = 1$.

Equation~(\ref{eqn:dtdarkmatter}) is analogous to
equation~(\ref{eqn:dthydro}) and helps to ensure accuracy in the N-body solver
by requiring that no particle travels more than one cell
width.  The parameter $\kappa_{\rm dm}$ is like $\kappa_{\rm
hydro}$, with a similar range of values.

Equations~(\ref{eqn:dtaccel}) and~(\ref{eqn:dtrad}) are supplementary to equation~(\ref{eqn:dthydro}) in that they
take into account the possibility of large accelerations due to either
gravity (equation~\ref{eqn:dtaccel}) or radiation pressure
(equation~\ref{eqn:dtrad}).  In equation~(\ref{eqn:dtaccel}), $\vec{g}$ is the
gravitational acceleration in each cell on level l.  In
equation~(\ref{eqn:dtrad}), $\vec{a}_{\rm rad}$ is the estimated
acceleration due to radiation pressure in each cell on level l,
defined as
\begin{equation}
\vec{a}_{rad} = \frac{ \sum_i \frac{\dot{E_i}}{c} \hat{r_i} }{m_b} 
\end{equation}
where the sum calculates the energy deposited in a cell during the
previous timestep due to \textit{all photon packets} that crossed that
cell, with $\hat{r_i}$ being a unit vector that accounts for the
packet direction.

Equation~(\ref{eqn:dtcond}) is the stability condition for an explicit
solution to the equation of heat conduction.  In this expression,
$n_b$ is the baryon number density. 
%defined as $n_b = \rho_b / \mu
%m_p$, where $\mu$ is the mean molecular weight, 
$\kappa(T)$ is the Spitzer thermal conductivity, and f$_{\rm sp}$ is a user-defined
conduction suppression factor whose value must be f$_{\rm sp} \leq 1$.  
%Both are defined in Section~\ref{sec.num.conductions}.  
$\kappa_{\rm cond}$ is a prefactor whose
value must be $0 < \kappa_{\rm cond}  < 0.5$, and is exactly 0.5 for the
implemention in \enzo.
From a practical
perspective, it is useful to note that, unlike other timestep criteria
discussed above (which effectively scale as $\Delta x / \sqrt{T}$),
the timestep criterion due to thermal conduction scales as $\Delta x^2
/ T^{2.5}$, which can result in a rapid decrease in timestep in
regions of high resolution and/or temperature.

Finally, equation~(\ref{eqn:dtexpand}) is a cosmological constraint that
limits the timestep so that the simulated universe only expands by
some fractional amount, f$_{\rm exp}$, during a single step.  In this
equation, $a$ and $\dot{a}$ refer to the scale factor of the universe
and its rate of change, respectively.  This criteria is necessary
because the expansion of the universe and its first derivative with
respect to time both appear in the equations of cosmological
(magneto)hydrodynamics and particle motion, and some limit is required
for the stability of the PPM algorithm in comoving coordinates.  This
criterion typically limits the timestep
only at high redshifts, when densities are relatively homogeneous.

%%% Local Variables: 
%%% mode: latex
%%% TeX-master: "ms"
%%% End: 
