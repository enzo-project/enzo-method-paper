\subsection{Magnetohydrodynamics: Constrained transport}
\label{sec.num.mhd-ct}
\def\Bvec{{\bf B}}
\def\Bf{Bf}
\def\Bc{Bc}
\def\Evec{{\bf E}}
\def\Divb{\ensuremath{\div \Bvec}}

The third solver we describe is an MHD method developed by 
\citet{Collins10} and since a full description and suite of test problems can be
found in that paper, we only describe the method briefly here.

The divergence of the magnetic field, \Divb, is
identically zero due to the fact that the evolution of the magnetic field is the
curl of a vector, and the divergence of the curl of a vector is identically
zero.
%\begin{eqnarray}
%\partial \Bvec/\partial t & =& \frac{1}{a} \curl \Evec\\
%\Divb & =& 0.
%\end{eqnarray}
The Constrained Transport (CT) method \citep{Evans88, Balsara99} for
magnetohydrodynamic (MHD) evolution employs this same vector property
to evolve the magnetic field in a manner that preserves $\Divb=0$.  
The electric field is computed using
the fluxes from the Riemann solver.  The curl of that electric field
is then used to update the magnetic field.  The advantage of this
method is that it preserves \Divb\ to machine precision.  The primary
drawback is increased algorithm complexity.  Note also that since 
only the update of the magnetic field is divergence free, any monopoles 
created by other numerical sources persist.

The base Godunov method, described in \citet{Li08a}, uses spatially and
temporally second order reconstruction (both MUSCL-Hancock and
Piecewise Linear Method), and a selection of Riemann solvers including
HLLC and HLLD \citep{Mignone07}, as described earlier.  
The constrained transport methods
are the first-order method described by \citet{Balsara99} and the
second order methods described in \citet{Gardiner05}.  The AMR
machinery is described by \citet{Balsara01} and \citet{Collins10}.

The increased complexity of the constrained transport scheme comes in
the form of area-averaged face and length-averaged edge-centered
variables, while the rest of \enzo\ employs predominantly
volume-averaged cell centered variables.  The magnetic field is
represented by both a face centered field, B$_f$, and a cell centered
field, B$_c$.  The electric field is edge centered.  The magnetic
field is updated in four steps: first, the Riemann problem is solved
in the traditional manner, using the cell centered field; second, an
edge-centered electric field is computed using the fluxes from the
Riemann solver; third, the curl of that electric field is used to
update the face centered field; finally, the cell centered magnetic
field is updated with an average of the face centered field.

Divergence-free AMR is somewhat more complex than the AMR employed
elsewhere in \enzo.  First, the interpolation must be constrained to be
divergence free.  Thus, all three face-centered field components are
interpolated in concert.  Second, any magnetic field information from
the previous timestep must be included in the interpolation, making the
interpolation more complex than the simple parent-child relation used
for other fields.  Third, the flux correction involves more possible
grid relations than traditional AMR.  In order to circumvent this last
complexity, the electric field is projected from fine grids to parent
grids (rather than the magnetic field), and is then used to re-update
the parent magnetic field.  This is described in detail in
\citet{Balsara99} and \citet{Collins10}.

%Cosmological MHD can be treated in a number of ways.  As described in 
%the previous section, and equation~\ref{eq:induction} the MUSCL with Dedner
%method treats the cosmological dilution of
%$\vecB$ with a source term,  However, one can eliminate the right hand side of equation 
%\ref{eq:induction} by defining
%$\vecB^\prime = \vecB / \sqrt{a}.$
%Then equation~\ref{eq:induction} becomes
%$$ 
%\frac{\partial \vecB^\prime}{\partial t} - \frac{1}{a}  \curl (\vecv \times
%\vecB^\prime)  = 0, \label{eq:induction_cosmo_prime_nosource}
%$$
%and we can reuse each part of the non-cosmological algorithm (hyperbolic solver,
%electric field creation, interpolation, projection, and flux correction) without
%modification.  
%GB: I removed the above because the rest of the paper now uses the definition
% of B in MHD-CT.

The dual energy formalism has also been incorporated in two possible ways, one internal
energy, as described in Section \ref{sec.hydro.ppm}, and one that
uses entropy \cite{TVD93, Collins10}.
%The additional equations are
%\begin{eqnarray}
%\end{eqnarray}
%\begin{eqnarray}
%  \frac{\partial S}{\partial t} +\frac{1}{a} \div (S \vecv) & = &  -
%  \frac{ 3 (\gamma - 1) \dot{a}}{a} S \\
%  \frac{\partial \rho e}{\partial t} + \frac{1}{a} \div (\rho e \vecv) &
%  = & -\frac{ 3(\gamma -1 ) \dot{a}}{a} + \frac{p}{a} \div \vecv,
%\end{eqnarray}
%where $e$ is the internal energy and $S=p/\rho^{\gamma-1}$ is the entropy.  If
%the ratio of thermal to kinetic energy, $\eta$, is below a certain value, the
%alternative energy is used.  As in \citet{Li08a}, we use $\eta=0.008$.

%
%\enzo's MHD-CT has been used in a variety of contexts, from galaxy
%cluster evolution \citep{Xu11,Xu12}, synthetic observations of radio
%halos \citep{Skillman13}, primordial star formation \citep{Xu08}, and
%present-day star formation \citep{Collins11, Collins12a}.
% GB NOTE: incorporate these elsewhere

