

\section{Physical Equations and Overview of Numerical Methodology}
\label{sec.overview}

We begin this section by first writing down the complete set of
physical equations that be solved by \enzo, and then briefly describe
the numerical algorithms that we use to solve these equations.  This
section is intended to be an overview of the \enzo\ code's
capabilities: thus, we gather all of the equations solved into a
single place and provide a brief and high-level introduction to the
numerics of the code.  Detailed descriptions of the individual
components are then provided in Section~\ref{sec.methods}.

% ====================================================

\subsection{Physical Equations}
\label{sec.equations}

% ----------------------------------------------------------------------------------------

%\subsubsection{Magnetohydrodynamics and Gravity}

The Eulerian equations of ideal magnetohydrodynamics including gravity
in a coordinate systems comoving with the cosmological expansion are
given by:

\begin{eqnarray} 
\frac{\partial \rho}{\partial t} 
          + \frac{1}{a} \nabla \cdot (\rho \vecv) & = & 0
        \label{eq:mass} \\
%
\frac{\partial \rho \vecv}{\partial t}  
   + \frac{1}{a} \nabla \cdot \left(\rho \vecv \vecv + \myvec{I}p^* - \frac{\vecB \vecB}{a} \right) & = &
    - \frac{\dot{a}}{a} \rho \vecv - \frac{1}{a} \rho \nabla \phi
    %
%          + \frac{1}{a} ( \vecv \cdot \nabla ) \vecv
%    & = & - \frac{\dot{a}}{a} \vecv 
%          - \frac{1}{a \rho} \nabla p
%          - \frac{1}{a \rho} (\nalba \times \vecB) \times \vecB
%          - \frac{1}{a} \nabla \phi, 
        \label{eq:momentum} \\
%
\frac{\partial E}{\partial t} 
+ \frac{1}{a} \nabla \cdot \left[ (E + p^*) \vecv -  \frac{1}{a} \vecB (\vecB \cdot \vecv) \right] & = &
     - \frac{\dot{a}}{a} \left( 2E - \frac{B^2}{2a} \right)  - \frac{1}{a} \vecv \cdot \nabla \phi 
     - \Lambda + \Gamma + \frac{1}{a^2} \nabla \cdot \fcond
     %
%         + \frac{1}{a} \vecv \cdot \nabla E
 %  & = & - \frac{\dot{a}}{a} (3 \frac{p}{\rho} + {\vecv \cdot \vecv} )
  %       - \frac{1}{a \rho } \nabla \cdot (p \vecv) 
%%\nonumber   & &   %uncomment for 2-column
 %        - \frac{1}{a} \vecv \cdot \nabla \phi 
 %               + \Gamma - \Lambda.
       \label{eq:total_energy}  \\
%
\frac{\partial \vecB}{\partial t} - \frac{1}{a}  \nabla \times (\vecv \times \vecB) & = & 0 \label{eq:induction}
\end{eqnarray}

%\begin{eqnarray} 
%\frac{\partial \rho}{\partial t} 
%          + \vecv \cdot \nabla \rho
%    & = & - \rho \nabla \cdot \vecv,
%        \label{eq:mass} \\
%%
%\frac{\partial \vecv }{ \partial t} 
%          + ( \vecv \cdot \nabla ) \vecv 
%    & = & - \frac{1}{\rho} \nabla p
%          - \nabla \phi,
%        \label{eq:momentum} \\
%%
% \frac{\partial E}{\partial t} + \vecv \cdot \nabla E
%    & = & - \frac{1}{\rho} \nabla \cdot (p \vecv) 
%          - \vecv \cdot \nabla \phi
%%\nonumber          \\  & &     % uncomment for 2-column
%     + \Gamma - \Lambda
%        \label{eq:total_energy}
%\end{eqnarray}
%
In these equations, $E$, $\rho$, $\vecv$, and $\vecB$ are the comoving
total fluid energy, comoving baryonic density, peculiar velocity, and
comoving magnetic field strength, respectively.  $\myvec{I}$ is the
identity matrix, and $a$ is the cosmological expansion parameter
(which is discussed in more detail below).
The first equation represents conservation of mass, the second, conservation of momentum, and the third, conservation of total fluid (kinetic plus thermal) energy.  They are respectively, the first, second and third moments of the Boltzmann equation.  The fourth equation is the magnetic induction equation.   Terms representing radiative cooling ($\Lambda$) and heating ($\Gamma$) enter on the right-hand side of the energy equation (\ref{eq:total_energy}), as well as the flux due to thermal heat conduction ($\fcond$).

The total fluid energy $E$ is given by:
\begin{equation}
E =  e + \frac{\rho v^2}{2}  + \frac{B^2}{2a}
        \label{eq:total_energy_def}.
\end{equation}
where $e$ is the comoving thermal energy density, and the total isotropic pressure $p^*$ is given by
\begin{equation}
p^* = p + \frac{B^2}{2a}
\end{equation}
The equations are closed with the equation of state and Poisson's equation (we work in the Newtonian limit):
%
\begin{eqnarray}
%
e   & = & p / [(\gamma - 1) \rho ],
        \label{eq:eq_of_state} \\
%
\nabla^2 \phi & = & 4 \pi G \rho_{\rm total}
%                   + 3 p_{total}/c^2) - \Lambda. 
        \label{eq:potential}
\end{eqnarray}
%
The equation of state is shown here for an ideal gas with a ratio of specific heats $\gamma$.
Equation (\ref{eq:potential}) includes contributions to the
gravitational potential $\phi$ from the total mass density $\rho_{\rm
  total}$.  Although we write the equations including both MHD and
comoving coordinates, the code can operate both in the pure
hydrodynamic limit ($\vecB = 0$) and without cosmological expansion
($a = 1$, $\dot{a} = 0$).

% ----------------------------------------------------------------------------------------

%\subsubsection{Comoving Coordinates}

For completeness, we note here that as part of the usual transformation to comoving coordinates, we have defined a set of comoving quantities.  See, for example, \citet{Peebles93} for more details.  Specifically, we define:
\begin{eqnarray}
\vecx & = & \vecr/a, \\
\vecv & = & a \hspace{0.5mm} d\vecx/dt = 
              \vecvp - \dot{a}\vecx, \\
\rho    & = & a^3 \rhop,   \\
p       & = & a^3 \pp, \\
E       & = & a^3 (\Ep - 
              \dot{a} \vecx \cdot \vecvp - 
              \frac{1}{2} \dot{a}^2 \vecx^2), \\
\phi    & = & \phip + \frac{1}{2} a \ddot{a} \vecx^2. \\
\vecB & = & a^2 \vecB^\prime
\end{eqnarray}
where primes indicate the usual quantities defined in a fixed frame frame, $\vecr$ is the proper distance and $\vecx$ is the comoving distance.

%These definitions result in the following comoving equivalents to eqs. 
%(\ref{eq:mass})-(\ref{eq:potential}).
%%
%
%\begin{eqnarray} 
%\frac{\partial \rhop}{\partial t} 
%          + \frac{1}{a} \vecvp \cdot \nabla \rhop 
%    & = & - \frac{1}{a} \rhop \nabla \cdot \vecvp, 
%        \label{eq:mass_prime} \\
%%
%\frac{\partial \vecvp}{\partial t}  
%          + \frac{1}{a} ( \vecvp \cdot \nabla ) \vecvp 
%    & = & - \frac{\dot{a}}{a} \vecvp 
%          - \frac{1}{a \rhop} \nabla \pp 
%          - \frac{1}{a} \nabla \phip, 
%        \label{eq:momentum_prime} \\
%%
%\frac{\partial \Ep}{\partial t} 
%         + \frac{1}{a} \vecvp \cdot \nabla \Ep 
%   & = & - \frac{\dot{a}}{a} (3 \frac{\pp}{\rhop} + {\vecvp \cdot \vecvp} )
%         - \frac{1}{a \rhop } \nabla \cdot (p \vecvp) 
%%\nonumber   & &   %uncomment for 2-column
%         - \frac{1}{a} \vecvp \cdot \nabla \phip 
%                + \Gamma - \Lambda.
%       \label{eq:total_energy_prime} \\
%%
%\Ep  & = & \ep + \frac{1}{2} {\vecvp \cdot \vecvp},
%        \label{eq:total_energy_def_prime} \\
%%
%\ep                  & = & 
%        \pp/\left[ \left( \gamma - 1 \right) \rhop \right],
%      \label{eq:state_prime} \\ 
%{\nabla}^2 \phip    & = & 
%        \frac{4\pi G}{a} ( \rhop_{total} - \rhop_0 ).
%      \label{eq:poisson}
%\end{eqnarray}
%%
%%We use the proper peculiar baryonic velocity $\vecv_b \equiv a(t) d\vecx / dt$, proper pressure $p$, and modified gravitational potential $ \phi $ which is related to the potential in proper coordinates $\Phi$ by 
%%$$\phi \equiv \Phi + \frac{1}{2} a \ddot{a} \vecx^2. $$
%%The density, however, is comoving:
%%$$\rho_b \equiv \rho_{b,proper}a(t)^3, $$
The expansion parameter $a \equiv 1/(1 + z)$ follows the expansion of a smooth, homogeneous background, where $z$, the redshift, is a function only of $t$.  All derivatives are determined with respect to the comoving position $\vecx$, which is defined simply to remove the universal expansion from the coordinate system.

The evolution of $a(t)$ is governed by the formula for the expansion of an isotropic, homogeneous universe:
%
\begin{eqnarray}
\frac{\ddot{a}}{a} & = & 
      - \frac{4 \pi G }{3 a^3 } (\rho_0 
      + 3p_0/c^2) 
      + \Lambda_c /3 .
      \label{eq:expansion} 
\end{eqnarray}
%
Here, $\rho_0$ is the mean comoving mass density (including both baryonic and dark matter), $p_0$ is the comoving background pressure contribution, and $\Lambda_c$ is the cosmological constant.
%$\gamma$ is the ratio of specific heats, and $\Lambda$ is the cosmological constant.
%We note that if $\gamma$ is not $5/3$ then $E$ in the first term on
%the right side of eq.~(ref{eq:total_energy}) must be replaced with 
%eq.~(\ref{eq:state}).
This system of equations is limited to the non-relativistic regime and assumes that curvature effects are not important --- both assumptions are reasonable as long as the  size of the simulated region is small compared to the radius of curvature and the Hubble length $c/H$ ($c$ is the speed of light and $H$ is the Hubble constant).

It is clear that the evolution equations are equivalent to the fixed coordinate version in the limit in which $a = 1$ and $\dot{a} = 0$.  We include the cosmological terms with the understanding that we are not restricting ourselves to cosmological applications.

%\subsubsection{Particles}

Any collisionless component (e.g. dark matter, stars) is modeled by particles which are governed by Newton's equations in comoving coordinates:
%
\begin{eqnarray}
\frac{d \vecx}{dt} 
    & = & \frac{1}{a} \vecv, 
          \label{eq:dm_position} \\
\frac{d \vecv}{dt} 
    & = & - \frac{\dot{a}}{a} \vecv
          - \frac{1}{a} \nabla \phi, 
          \label{eq:dm_velocity} 
\end{eqnarray}
%
The particles also make a contribution to the potential through their contribution to Poisson's equation.

%\subsubsection{Chemistry}

In addition, we can solve the mass conservation equations for a set of chemical species and reactions.  For any species $i$ with comoving number density $n_i$, these equations have the form:
\begin{equation}
\frac{\partial n_i}{\partial t} 
          + \frac{1}{a} \nabla \cdot (n_i \vecv) = 
        k_{ij}(T) n_i n_j 
      + \Gamma^{ph}_j n_j 
        \label{eq:species_evolution}
\end{equation}
where $k_{ij}$ are the rate coefficients for the two-body reactions
and are usually functions of only the gas species (we will
specifically note the cases where we either include three-body
reactions or have density-dependent rates).  The $\Gamma^{ph}_j$ are
destruction/creation rates due to photoionizations and/or
photodissociations.  Currently the species we can follow include H,
H$^+$, He, He$^+$, He$^{++}$, and $e^-$, as well as additional options
that add H$^-$, H$_2$, H$_2^+$, and  HD, D, and D$^+$.

%\red{
%There is some debate over whether to put in the hydro equations without cosmology, in addition
%to the equations that are already here.  Would this
%be more or less clear?  We're leaning towards saying ``when cosmology is turned off,
% $\dot{a}$ goes to zero and $a$ goes to 1, and the poisson equation changes to blah blah blah'' 
%and assuming our readers can figure it out for themselves.
%We also want to emphasize that the heating and cooling is glossed over in this section, 
%and it's discussed more thoroughly later.
%} %end red
% GB: I agree this would be shorter, and maybe it is a better idea, but I think it's nice to be able
% GB: to include both equations and at the same time make the transformation to comoving coords
% GB: very clear -- this has engendered a lot of confusion in the past and one of the goals of writing
% GB this paper is to answer peoples questions before they ask them.


% GB: I commented out the following.  It has now been incorporated into the above discussion
% GB: (see my comments anove on why I think this is a good way to do it)
%
%\enzo\ solves the equations of ideal gas dynamics
%in a coordinate system that is comoving with the expanding universe:

%\begin{equation}
%\frac{\partial \rho_b}{\partial t} + \frac{1}{a} \vecv_b \cdot \nabla \rho_b =  \label{enzoconserve}
% - \frac{1}{a} \rho_b \nabla \cdot \vecv_b 
%\end{equation}

%\begin{equation}
%\frac{\partial \vecv_b}{\partial t} + \frac{1}{a} (\vecv_b \cdot \nabla) \vecv_b = \label{enzomomentum} 
%-\frac{\dot{a}}{a} \vecv_b - \frac{1}{a \rho_b} \nabla p - \frac{1}{a} \nabla \phi
%\end{equation}

%\begin{eqnarray}
%\frac{\partial E}{\partial t} + \frac{1}{a} \vecv_b \cdot \nabla E =  
%- \frac{\dot{a}}{a} \left( 3 \frac{p}{\rho_b} + \vecv_b^2 \right)  \nonumber \\
%- \frac{1}{a \rho_b} \nabla \cdot (p \vecv_b) \nonumber \\
%- \frac{1}{a} \vecv_b \cdot \nabla \phi + \Gamma - \Lambda
%\label{enzoenergy}
%\end{eqnarray}

%Where $\rho_b$ is the comoving baryon density, $\vecv_b$ is the baryon velocity, $p$ is the pressure, 
%$\phi$ is the modified gravitational potential (in comoving coordinates, which is related to
%the potential in proper coordinates $\Phi$ by $\phi \equiv \Phi + 0.5 $a\"{a}$ \vecx^2$) and 
%$a$ is the ``expansion parameter'' which describes the expansion of a smooth, homogeneous 
%universe as a function of time.  This expansion parameter is related to the redshift:  
%$a \equiv 1/(1+z)$.  All derivatives are in comoving coordinates.  $E$ is the specific
%energy of the gas (total energy per unit mass), and
%$\Gamma$ and $\Lambda$ represent radiative heating and cooling processes as described below.
%  Equations~\ref{enzoconserve}, \ref{enzomomentum}
%and \ref{enzoenergy} represent the conservation of mass, momentum and total (e.g.,
%kinetic plus thermal) fluid energy.

%The equations above are closed with three more equations:

%\begin{equation}
%E = p / [(\gamma - 1) \rho_b] + \vecv^2/2   \label{enzoeos}
%\end{equation}

%\begin{equation}
%\nabla^2 \phi = \frac{4 \pi G}{a} (\rho_b + \rho_{dm} - \rho_0 )
%\label{enzopoisson}
%\end{equation}

%\begin{equation}
%\frac{\ddot{a}}{a} = - \frac{4 \pi G}{3 a^3} (\rho_0 + 3 p_0 / c^2) + \Lambda/3.
%\label{enzocomove}
%\end{equation}

%where $\rho_{dm}$ is the comoving dark matter density, $\rho_0$ is the comoving
%background density 
%($\rho_0 \equiv \Omega_{matter} \rho_{crit}$) and $p_0$ is the background pressure, 
%$\gamma$ is the ratio of specific heats and $\Lambda$ is the cosmological constant.
%Equations~\ref{enzoeos}, \ref{enzopoisson} and~\ref{enzocomove} are the 
%equation of state, Poisson's equation in comoving form and an equation that 
%describes the evolution of the comoving coordinates (i.e. the formula for the 
%expansion of an isotropic, homogeneous universe). 
%All particles in the simulation are governed by Newton's equations in comoving 
%coordinates:

%\begin{equation}
%\frac{d \vecx_{part}}{dt} = \frac{1}{a} \vecv_{part}
%\label{enzopartvel}
%\end{equation}

%\begin{equation}
%\frac{d \vecv_{part}}{dt} = -\frac{\dot{a}}{a} \vecv_{part} - \frac{1}{a} \nabla \phi
%\label{enzopartmom}
%\end{equation}

%Where $\vecx_{part}$ and $\vecv_{part}$ refer to the position and peculiar velocity of any
%particles in the system.  Note that the system of equations~\ref{enzoconserve}-\ref{enzopartmom} 
%is valid only in regimes where relativistic effects are not
%important ($v_b, v_{dm} \ll c$, where c is the speed of light) 
%and where cosmological curvature effects are 
%unimportant, meaning that the simulation volume is much smaller than the radius
%of curvature of the universe, as defined as $r_{hub} \equiv c/H_0$, where $c$ is the
%speed of light and $H_0$ is the Hubble constant.

%\dcc{Fixed punctuation, added 'no mhd, rhd' caveat to Zeus statements.  This has caused
%confusion in the past.}

%\subsubsection{Magnetohydrodynamics}



%\subsubsection{Radiation transport: ray-tracing}

The code can include either a homogeneous radiation background (with possible local self-shielding corrections), or evolve an inhomogeneous radiation field either by directly solving the radiative transfer equation along rays or by solving a set of moment equations derived from the radiative transfer equation.  

The radiative transfer equation in comoving coordinates 
\citep[e.g.,][]{Gnedin97} reads
%
\begin{equation}
  \label{eq:rteqn}
  \frac{1}{c} \; \frac{\partial I_\nu}{\partial t} + 
  \frac{\hat{n} \cdot \nabla I_\nu}{\bar{a}} -
  \frac{H}{c} \; \left( \nu \frac{\partial I_\nu}{\partial \nu} -
  3 I_\nu \right) = -\kappa_\nu I_\nu + j_\nu .
\end{equation}
%
Here $I_\nu \equiv I(\nu, \mathbf{x}, \Omega, t)$ is the radiation
specific intensity in units of energy per time $t$ per solid angle per
unit area per frequency $\nu$.  $H = \dot{a}/a$ is the Hubble
constant, where $a$ is the scale factor.  $\bar{a} = a/a_{em}$ is the
ratio of scale factors at the current time and time of emission.  The
second term represents the propagation of radiation, where the factor
$1/\bar{a}$ accounts for cosmic expansion.  The third term describes
both the cosmological redshift and dilution of radiation.  On the
right hand side, the first term considers the absorption coefficient
$\kappa_\nu \equiv \kappa_\nu(\mathbf{x},\nu,t)$, and the second term
$j_\nu \equiv j_\nu(\mathbf{x},\nu,t)$ is the emission coefficient
that includes any point sources of radiation or diffuse radiation.

%\subsubsection{Radiation transport: moment method}

In the other limit, we solve an equation derived from integrating
the equation of radiative transfer over angles.  In particular, we use a
flux-limited diffusion approximation for cosmological radiative
transfer, with couplings to both the gas energy and chemical number
densities. The equations solved are:
\begin{eqnarray}
  \label{eq:fld_radiation}
  \frac{\partial E_r}{\partial t} + \frac1a \nabla\cdot\left(E_r \vecv \right) &=& 
    \nabla\cdot\left(D\nabla E_r\right) -
    \frac{\dot{a}}{a} E_r - c\kappa E_r + \eta, \\
  \label{eq:fld_heating}
  \frac{\partial e_c}{\partial t} &=& -\frac{2\dot{a}}{a} e_c + \Gamma - \Lambda,
%  \label{eq:fld_chemistry}
%  \partial_t {\mathrm n}_i + 
%    \frac1a \nabla\cdot\left({\mathrm n}_i \vec{v}_b\right) &=& 
%    \alpha_{i,j} {\mathrm n}_e {\mathrm n}_j - {\mathrm n}_i
%    \Gamma_i^{ph}, \quad i=1,\ldots,N_s,
\end{eqnarray}
where $E_r$ is a grey radiation energy density and $e_c$ is the
internal energy correction due to photo-heating and chemical cooling.
%and ${\mathrm n}_i$ correspond to the chemical number densities of
%HI, HII, HeI, HeII and HeIII, and ${\mathrm n}_e$ is the electron
%number density.  
Here, we define the grey radiation energy density
through first assuming a fixed frequency spectrum, i.e.
$E_{\nu}(\nu,\vec{x},t) = \tilde{E}_r(\vec{x},t) \chi(\nu)$, and then
defining the integrated quantity
\begin{equation}
\label{eq:grey_radiation_energy}
   E_r(\vec{x},t) \equiv \int_{\nu_0}^{\infty}
   E_{\nu}(\nu,\vec{x},t)\,\mathrm d\nu \  = \ 
   \tilde{E}_r(\vec{x},t) \int_{\nu_0}^{\infty} \chi(\nu)\,\mathrm d\nu.
\end{equation}
In addition, $D$ in (\ref{eq:fld_radiation}) is the {\em Larsen
square-root flux-limiter} \citep[see][]{Morel2000}, $\kappa$ is the total
opacity, $\eta$ is the field of radiation sources, $\Gamma$ provides the
radiation induced photo-heating, and $\Lambda$ the chemical
cooling.
%, $\alpha_{i,j}$ are the reaction rate coefficients defining
%the interactions between species ${\mathrm n}_i$ and ${\mathrm n}_j$,
%and $\Gamma_i^{ph}$ are the radiation induced photo-ionization rates. 

%\subsubsection{Thermal conduction}

Enzo implements the equations of isotropic heat conduction in a manner
similar to that of \citet{2007ApJ...664..135P}, where we model the
flux of isotropic heat transport:

\begin{equation}
\fcond = -\kappa_{sp} \nabla T
\end{equation}

Where $\kappa_{sp}$ is the Spitzer conduction coefficient
\citep{1962pfig.book.....S} and T is the baryon temperature (with
fluids explicitly assumed to be single-temperature).  Saturation of the heat flux in
high-temperature, low-density regimes (such as the intracluster medium
in galaxy clusters) is taken into account.  Thermal conduction in a
plasma can be strongly affected by the presence of magnetic field
lines, which may strongly suppress heat flow perpendicular to the
magnetic field.  In that case, we allow for heat transport only along
the magnetic field lines in simulations that include this physics, as follows:

\begin{equation}
\fcond= -\kappa_{sp} \hat{\textbf{b}} (\hat{\textbf{b}} \cdot \nabla T)
\end{equation}

Where $\hat{\textbf{b}}$ is the unit vector denoting the direction of the
magnetic field.  In both the isotropic and anisotropic limits, we
assume that conductivity can be suppressed relative to the ideal
Spitzer conductivity, and express this as a multiplicative constant.

% ====================================================

\subsection{Overview of Numerical Methods}
\label{sec.method_overview}

In this section, we briefly describe the numerical methods which are
used to solve the equations outlined in Section~\ref{sec.equations}.  We
proceed through the numerical methods in the same order as will be used in
section~\ref{sec.methods} so that there is a one-to-one correspondence
between each of the following overviews and the complete description
provided in that section.  The goal here is to introduce the reader to
the basic principles of the methods without drowning in detail.

\subsubsection{Structured Adaptive Mesh Refinement}

The primary purpose of the \enzo\ code is its Adaptive Mesh Refinement (AMR)
capability, which allows it to reach extremely large dynamical ranges
with limited computational resources, opening doors previously closed
by finite memory and computational time. Unlike moving mesh methods
\citep{1995ApJS..100..269P,1995ApJS...97..231G} or  
methods that subdivide 
individual cells \citep{Adjerid}, Berger \& Collela's AMR (also referred 
to as \emph{structured} AMR) utilizes an adaptive hierarchy of grid 
patches at varying levels of resolution.  Each rectangular grid patch 
(referred to as a ``grid'') covers some region of space in its 
\emph{parent grid} which requires higher resolution, and can itself 
become the parent grid to an even more highly resolved \emph{child grid}. 

The grid hierarchy begins with the root grid, which covers the entire
domain of interest with a coarse, uniform, Cartesian grid. Then, as
the solution evolves and interesting regions form, finer meshes are
placed below these regions (we use the notation `below' to refer to
finer grids and `above' for coarser grids).  We restrict the ratio
between cell sizes to be an integer, typically 2 or 4, and refer to a
level as all grids with the same cell size.  In order to keep
things as simple as possible, the edges of subgrids must coincide with
the cell edge of its immediate parent (coarser) grid. Additionally,
the hierarchy can be initialized with one or more static grids if a
higher initial resolution is required.

Given the hierarchy at some time $t$, we advance the solution in the
manner of a W-cycle in a multigrid solver.  First, we determine the
maximum time step allowed for the coarsest grid based on a variety of
accuracy and stability criteria and advance the grid by that time
interval, $\Delta t_0$.  We then move down to the next level and
advance all the grids on that level by a timestep $\Delta t_1$
($\Delta t_1 \leq \Delta t_0$), which is the minimum of all the allowed
timesteps for those grids.  If there are more levels, we repeat this
procedure until the bottom level of the hierarchy has been reached.
Once there, we continue advancing the grids on the lowest level until
they have `caught up' to the next highest level above (i.e. $\sum
\Delta t_l = \Delta t_{l-1}$).  This procedure repeats itself until
all grids have been advanced by a total time of $\Delta t_0$.

Since interesting regions on the grid may move, the hierarchy must
adapt itself.  We do this whenever a level has caught up to the
coarser level above it and consists of entirely rebuilding the grids
on that level and below.  This is done by applying the grid refinement
criteria to the grids on that level, flagging zones that require
extra grids.  These criteria depend on the physical problem being
simulated.  We have implemented a number of options, including shock
and steep gradient detectors, but in the problem described here,
employ one based on the mass within a cell, imitating the Lagrangian
nature of the SPH algorithm.  Once a grid has a set of flagged cells,
we run a machine-vision based algorithm \citep{Berger91} to find edges
and determine a good placement of subgrids.  These subgrids must not
overlap one another, must cover all flagged cells and their
neighbouring cells, and be above a preset efficiency threshold, where
the efficiency of a cell is defined as the ratio of flagged cells to
total cells.  Once these new subgrids have been identified, the
solution from the next coarser grid is interpolated (see below) in
order to initialize the values on the new grids.  Finally, any overlap
between these new subgrids and the old ones is identified and the
solution within the regions of overlap is copied to the new subgrids.
The entire procedure just outlined is then repeated on the new grids
and in this way the entire hierarchy (from the original level examined
and below) is rebuilt.

\subsubsection{Godunov PPM method (HD only)}

Four different (magneto)-hydrodynamic methods are implemented in
\enzo: (i) the hydrodynamic-only piecewise parabolic method (PPM)
developed by~\citet{1984JCoPh..54..174C} and extended to cosmology
by~\citet{1995CoPhC..89..149B}; (ii) the MUSCL-like Gudonov scheme
\citep{MUSCL} with or without magnetic fields and Dedner-based
divergence cleaning, described in more detail in
\citet{WangAbelZhang08} and \citet{WangAbel09}; (iii) a constrained
transport staggered MHD scheme \citep{Collins10}, and (iv) the
second-order finite-difference hydrodynamics method described in
\zeus~\citep{Stone92a,Stone92b}.

We begin with the direct-Eulerian PPM implementation.  This is an
explicit, higher-order-accurate version of Godunov's method for ideal
gas dynamics with third order-accurate piecewise parabolic monotonic
interpolation and a nonlinear Riemann solver for shock capturing.  It
advances the hydrodynamic equations in the following steps:
%\begin{enumerate}
% \item 
 (i) Construct monotonic parabolic interpolation of cell average data, for each fluid quantity.
% \item 
(ii) Compute interface states by averaging the parabola over the domain of dependence for each interface
% \item 
(iii) Use interface data to solve the Riemann problem.
 %\item 
 (iv) Difference the interface fluxes to update the cell average quantities.
%\end{enumerate}

This PPM implementation does an excellent job of capturing strong
shocks in a few cells.  Multidimensional schemes are built up by
directional splitting and produce a method that is formally second
order-accurate in space and time and which explicitly conserves mass,
linear momentum, and energy.  A variety of reconstruction methods and
Riemann solvers have been implemented.

As described in \citet{Bryan95}, we modify the method for use in
hypersonic flows when the thermal energy $e$ is extremely small
compared to the total energy $E$.  This is a problem because in the
total energy method, the temperature is computed by subtracting one
large number from another (i.e. the kinetic energy from the total
energy), a situation which can generate large numerical inaccuracies.
We address this situation by also solving the thermal energy equation
and using $e$ from this equation when we expect the error to be large.

%The advantage of the PPM method is that it conserves energy better and is higher-order accurate.  The disadvantage is that it is generally less robust than the ZEUS method.

\subsubsection{Godunov MUSCL (HD) with Dedner divergence cleaning (MHD)}
This solver was developed by Wang \& Abel to attack problems in
magnetic field amplification during the formation of galaxies
\citep{Wang:2009a} and to understand the role proto-stellar jets for
the theory of star formation \citep{Wang:2009b}. It combines the standard
approach of Godunov \citep{Godunov1959} for finite volume techniques
with the method of lines as described as e.g. by
\cite{leveque2002finite} and \cite{toro-1997}. In addition, however,
it implements the hyperbolic divergence cleaning algorithm of
\cite{2002JCoPh.175..645D}. It supports multiple approximate Riemann
solvers and non-ideal equations of states. Consequently,
these suite of solvers can be used for hydro and magneto-hydrodynamic
simulations. It is also this class of solvers as well as a version of
the PPM hydro solver which have been ported to Nvidia's CUDA framework
allowing \enzo to take advantage of modern graphics hardware \citep{Wang:2010}.

\subsubsection{MHD with Constrained Transport (MHD)}

\subsubsection{Second order finite difference method (HD only)}

The last hydrodynamics method we briefly describe is the
finite-difference algorithm used in the \zeus\ code, as given in more
detail in \citet{Stone92a}.  For this reason, we call this the \zeus\
method, although the code is entirely independent of the \zeus\ code,
and only the hydrodynamical algorithm of \zeus\ is implemented in
\enzo -- the MHD and RHD schemes are not.

The ZEUS method uses a staggered mesh such that the velocities are
face-centered, while the density and internal energy quantities are
cell-centered.  It splits the solution up into two steps, the
so-called source step, in which the momentum and energy values are
updated to reflect the pressure and gravity forces, as well as the
effect of an artificial viscosity required for stability.  The second
step (known as the transport step) accounts for the advection of
conserved quantities (mass, momentum and energy) across the grid.

\subsubsection{Gravity}

The current implementation of self-gravity in \enzo\ uses a Fast
Fourier Technique \citep{Hockney88} to solve Poisson's equation on the
top (level 0) grid.  The advantage of using this method is that it
naturally allows both periodic and isolated boundary conditions for
the gravity, choices which are very common in astrophysics and
cosmology.  On subgrids, we interpolate the boundary conditions from
that grid's parent (either the level 0 grid or some other subgrid).
The Poisson equation is then solved using a multigrid technique.  In
\enzo, self-gravity is optional.  Aside from calculating the
gravitational potential from the baryon fields and particles existing
in the simulation, there are also a number of options for static gravitational fields.

\subsubsection{N-body Dynamics}

Collisionless matter (dark matter, stars, etc.) is modeled with
particles that interact with the baryons only via gravity.  These
particles are advanced in a single time step using a drift-kick-drift
algorithm to provide second-order accuracy even in the presence of
changing timesteps.  Since the particles follow the collapse of
structure, they are not adaptively refined.  Nor are there duplicate
sets of particles for each level; instead, each particle is associated
with the highest refined level available at its position in space, and is
moved as the hierarchy is rebuilt.  Thus, a particle has the same
timestep and feels the same gravitational force as a grid at that refinement
level.

Although the particles are fixed in mass once initialized, we can
create them with any initial set of masses and positions.  For example, in
many cosmological simulations, a static subgrid is included from the
beginning in order to improve the initial baryonic mass resolution.
On this subgrid, we also use smaller particles to improve the
collisionless mass resolution.  One particle per initial grid point
seems to provide approximately equal sampling between the dark matter
and gas.

\subsubsection{Chemistry}
\label{sec.ov.chem}

\enzo\ includes the capability of following up to
12 chemical species using a non-equilibrium solver.  The species can
be turned on in sets, with the simplest model including just H, H$^+$,
He, He$^+$, He$^++$, and $e^-$, and more complete models adding first
species important for gas-phase molecular hydrogen formation: $H^-$,
H$_2$ and H$_2^+$, and then $HD$ formation: $HD$, $D$, $D+$.  The
cooling and heating due to these species is also included (see the
next section).  The solution of the rate equations is carried out
using a time-backwards differencing which insures stability; to
maintain accuracy we sub-cycle the rate equations with a shorter
timestep such that the electron and neutral fractions do not change by
more than 10\%.

\subsubsection{Radiative Cooling and Heating}

Enzo may operate in a number of modes with regard to radiative cooling
and heating.  In the simplest mode, with the multi-species flag turned
off (i.e. set to zero), the cooling rate is computed from a simple
temperature-dependent cooling rate, taken from \citet{SW87}.  If
chemistry is turned on, then the code can include cooling from all
species of Hydrogen and Helium (including H$_2$) -- this primordial cooling is
also treated using a backwards-time differencing scheme with
sub-cycling to maintain accuracy.  Finally, we have also recently
added metal cooling based on a set of multi-dimensional lookup tables
computed with the \texttt{Cloudy} code \citep{1998PASP..110..761F} as
described in \citet{2008MNRAS.385.1443S} and
\citet{2011ApJ...731....6S}.  Finally, we note that the cooling and
heating is (almost always) treated in the optically-thin limit,
although the code can also follow radiative transfer in a limited set
of energy bins (see below).

\subsubsection{Homogeneous radiation backgrounds}

The chemical networks and heating rates described in the previous
section can be affected by external radiation fields, and the code
includes a number of pre-set radiation backgrounds that are uniform in
space but can very in time.  These are generally based on the
redshift-dependent rates given in \citet{1996ApJ...461...20H} and
\citet{2012ApJ...746..125H}, but can also include a uniform H$_2$
photo-dissociating background that is either constant in time or
varying as in \citet{WiseAbel05}.

\subsubsection{Radiation transport: ray tracing}

The code includes a photon-conserving radiative transfer algorithm
that is based on an adaptive ray-tracing method that utilizes the
HEALPix pixilation of the sphere \citep{AbelWandelt02}.  Photons are
integrated outwards from sources using an adaptive time-stepping
scheme that preserves accuracy in ionization fronts even in the
optically-thin limit.  This has been integrated with the chemistry and
cooling network to provide ionization and heating rates on a
cell-by-cell basis.  The method is described in more detail in our
main methodology section but is fully detailed and tested in
\citet{Wise11_Moray}.

\subsubsection{Radiation transport: Flux-limited diffusion}

A second option for radiative transfer is a moment-based method that
adds an additional field tracking the radiant energy density
This field is evolved using the flux-limited diffusion method, which
transitions smoothly between streaming (optically thin) and opaque
limits and is coupled to the hydrogen ionization network.  The
resulting set of linear equations is solved using the parallel HYPRE
framework.  Full details for the enzo implementation can be found in
\citet{ReynoldsHayesPaschosNorman2009}.

\subsubsection{Heat Conduction}

Heat conduction, both isotropic and anisotropic, can be included using
an sub-cycled, operator-split method.  The heat fluxes are computed
with simple second-order accurate finite differences and stability is
ensured by restricting the time step, and using flux-limiters where
appropriate.

\subsubsection{Star Formation and Feedback}

A family of simple heuristic methods are used to model the formation
of stars and their feedback of metals and energy into the gas.  These
methods are based on the work of \citet{CO1992}, and involve the
identification of plausible sites of star formation based on a set of
criteria (for example, dense gas with a short cooling time, which is
both collapsing and unstable).  The local star formation rate is
computed using a range of methods, such as a density-dependent method
based on the Schmidt-Kennicutt relation \citep{K89}.  The affected gas
is converted into a star particle over a few dynamical times, and
metals and thermal energy are injected into the region surrounding the
star particle.

\subsubsection{Timestep constraints}

All grids on a given level are advanced with the same timestep.  To
determine this timestep, we calculate, for each cell and for each
physical process (except for the chemistry step, which is sub-cycled),
the minimum $\Delta t$ allowed for stability.  Then the minimum of all
timesteps is taken and the level is advanced with this step.

%%% Local Variables:
%%% mode: latex
%%% TeX-master: "ms"
%%% End: 
