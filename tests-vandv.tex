\subsubsection{The \enzo\ test suite}
\label{sec.tests.suite}

\enzo\ is capable of simulating a large variety of problems types,
with all but a few of these types requiring only a parameter file as an
input.  The most notable exception is the cosmology simulation, which takes
as input initial conditions created by other codes.  The suite of test
problems spans a wide range in complexity.  At one end of this spectrum
are simple problems that utilize only a single component of \enzo\ and
for which analytic solutions exist for comparison with the simulation
results.  At the opposite end are problems that exercise a large
portion of \enzo's machinery in concert.  Together, the available
problem types fully cover all of \enzo's functionality.  This enables
them to serve as a vehicle for verifying that the code's behavior
remains stable over time, on new computing platforms, and after modification of the codebase.

\enzo\ uses an automated testing framework that allows a user to run,
with a single command, a set of test problems and compare the results
against results produced by any other version of the code.  Within the
\enzo\ source distribution, the test problem parameter files are stored
in a nested directory structure organized according to the primary
functionality tested (e.g., hydrodynamics, gravity, cooling, etc.).
Each parameter file is accompanied by a text file containing various
descriptive keywords, such as the machinery tested, the
dimensionality, and the approximate run time.  A test runner script is
responsible for taking as input from the user a series of keywords
that are used to select a subset of all available test problems.  The
test problems are also grouped into three suites: the quick, push, and
full suites, each a superset of the ones named before.  The quick
suite is considered to minimally cover the primary functionality in
\enzo\ and is designed to be run repeatedly during the development
process.  The push suite has slightly increased feature coverage and
is mandated to be run before code changes are accepted into the main
repository.  The full suite consists of all test problems that can be
run with no additional input.  Approximate run times for the quick,
push, and full suite are 15 minutes, 1 hour, and 60 hours,
respectively, on a relatively new desktop computer (circa 2013).

After the test problems are selected by the test runner script, they
are run in succession using either the \enzo\ executable contained
within that distribution or an external executable built from another \enzo\
version and specified by the user.  This allows for any version of the
code to be tested with an identical set of test problems.  After
running the test problem simulations, the test runner then performs a
series of basic analysis tasks using the \texttt{yt} analysis toolkit
\citep{2011ApJS..192....9T, 2011arXiv1112.4482T}.  The default
analysis performed on all test problems includes calculation of
various statistics (such as extrema, mean, and variance) on the fields
present in the output data.  Custom analysis provided by scripts that
accompany the test problem parameter files is run for special cases,
such as when an analytical solution exists that can be compared
against the simulation data.  After the analysis is performed, the
results are compared against a set of gold standard results that are
maintained on a website and downloaded on the fly by the test runner
script.  Alternately, results from any version of \enzo\ can be stored
locally and compared to any other version of the code.  In
Section~\ref{sec.tests.problems}, we describe some of the key test
problems that are used to verify proper behavior.  All of these test
problems, as well as the scripts to generate the figures from them,
are available at \url{http://bitbucket.org/enzo/enzo-method-paper},
the Bitbucket repository for the \enzo\ method paper.  The
\texttt{test\_problems} subdirectory in this repository contains all of
the files necessary to regenerate all of the figures in
Section~\ref{sec.tests.problems}.  We note for completeness that 
the figures in Section~\ref{sec.tests.problems} were generated using \texttt{yt} version 2.5.3 and the
\url{https://bitbucket.org/enzo/enzo-dev} repository of \enzo\ with
changeset \texttt{5d8c412}, which corresponds to \enzo\ 2.3.

\subsubsection{Code comparisons}
\label{sec.tests.compare}

Over the course of its existence, \enzo\ has been involved in numerous
comparisons with other astrophysical codes used for self-gravitating
fluid dynamics.  In general, \enzo\ behaves in a manner similar to other
grid-based (and particularly AMR-based) codes, as we will summarize below.

\enzo\ has been involved in multiple cosmological code comparisons,
including the Santa Barbara Cluster code comparison project
\citep{SantaBarbara}, a large comparison of N-body simulations
\citep{2008CS&D....1a5003H}, as well as several direct comparisons
between \enzo\ and the GADGET SPH code in a variety of
set-ups, including N-body and adiabatic hydrodynamics
\citep{2005ApJS..160....1O,2005MNRAS.364..909V, 2011MNRAS.418..960V}
and simulations of the Lyman-alpha forest \citep{2007MNRAS.374..196R}.
Compared to the other codes involved in these projects, the \enzo\ code typically has a more difficult time resolving
small-scale self-gravitating structures (for an equivalent dark matter
particle count and nominally equivalent force resolution), but does
comparably well as a tree-based code for larger structure, and is
typically superior in terms of resolving fluid features due to its
higher-order (and artificial viscosity-free) PPM hydrodynamics 
solver.  When examining classical test cases such
as the Santa Barbara project \citep{SantaBarbara}, \enzo\ forms galaxy
clusters with very similar density, temperature, and entropy profiles
to other grid-based codes that use Godunov-type hydro methods, which
systematically differed from particle-based codes using SPH in this
code comparison.  Similarly, in galaxy cluster simulations that look
specifically at the properties of cosmological shocks
\citep[e.g.][]{2011MNRAS.418..960V}, \enzo\ produces results that are
similar to other high-order grid-based hydrodynamics codes, and a far
superior performance of \enzo\ is observed (in terms of resolution of
fluid features and shock detection) in low-density regions when
compared to a particle-based code.  In tests of the Lyman-alpha forest
that include radiative cooling and a uniform metagalactic ultraviolet
background, \enzo\ and GADGET provide results on metrics such as the
matter power spectrum that are comparable to within 5\%
\citep{2007MNRAS.374..196R}.

Several comparisons have been made that focus specifically on
hydrodynamics solvers and fluid behavior.  In particular, the work of
\citet{2007MNRAS.380..963A} and \citet{Tasker2008} perform
direct comparisons between several grid- and particle-based
codes for a variety of fluid-centric test problems (including shocked gas clouds,
self-gravitating, translating clouds, and Sedov-Taylor explosions),
and show that \enzo\ is comparable or superior in behavior to the
other grid-based hydrodynamics codes involved in the comparison, and
provide useful information on the sort of practical challenges that a
user of an
AMR code such as \enzo\ may experience.  More specific comparisons, including one
testing the linear and nonlinear growth of the
Kelvin-Helmholz instability \citep{2012ApJS..201...18M}, as well as a
comparison that more
broadly examines Galilean invariance in grid-based codes
\citep{2010MNRAS.401.2463R}, show that \enzo, and in particular its
implementation of the PPM hydrodynamics solver,
converge to the correct solution as expected, and generally provide
less diffusive solutions than lower-order codes, including those that
use artificial viscosity.  Finally, there have been two code comparison
projects that focus on turbulence simulations.  The first studied the
behavior of decaying isothermal supersonic turbulence
\citep{2009A&A...508..541K}, and the second examined supersonic
magnetohydrodynamical turbulence \citep{2011ApJ...737...13K}.  Both
included the \enzo\ code, with the former testing the PPM
hydrodynamics and the latter both the constrained transport MHD
implementation of \citet{Collins10} and the Dedner method of \citet{WangAbelZhang08}.  In both cases, \enzo\ performed similarly to other
grid-based codes that use Godunov-based fluid solvers, and typically
had better effective resolution than particle-based codes when using the
same number of particles as the number of grid cells in the \enzo\ simulation.

Three other comparisons between the \enzo\ code and other simulation
tools have been performed that focus on physics other than gravity and
fluid flow.  The flux-limited diffusion radiation transport scheme was
measured against several test problems by \citet{IlievEtAl2009},
which involved tests with and without analytical solutions.  \enzo\ produced
results similar to both the analytical solutions and results obtained by 
other codes.  We note, however, that there were minor differences throughout
the comparison between various codes, and the majority of the codes differed
in at least a subset of the tests.
\citet{2011ApJ...726...55T} show the result of varying reaction rate
coefficients for the formation of molecular hydrogen via the 3-body
process in both the \enzo\ and GADGET-2 codes, observing similar trends between
the two codes. However, at nominally equivalent resolution
where the particle and cell gas masses are comparable, 
\enzo\ simulations typically displayed a
substantially higher level of gas structure.  This is unsurprising due
to \enzo's higher-order hydro solver.  Finally,
\citet{2012ApJ...744...52P} show the results of comparing \enzo\ in
its non-AMR mode to a smoothed-particle hydrodynamics code, SNSPH, in
the context of common-envelope binary stellar interactions.  The
authors show that the codes display reasonable convergence properties
as a function of simulation resolution, and also agree quite well with
each other. However, the observed mass-loss rates do not agree
particularly well with observations.
