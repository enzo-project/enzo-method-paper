% EJT: small changes only.
% dark matter particles -> massive (dark matter and star) particles

% Changed:  ``Particles from sub-grids are also added in this way'' to ``Particles on sub-grids within the grid's volume are also added to its gravitating field using the same method.'' I expanded because I missed the point and had to clarify with Brian when I read the text. But if it's too long, please just cut it!

% added: ``Each scheme can be selected via a run-time parameter.'' I felt it sounded like a dice roll otherwise, but again, feel free to overule me!

% removed blank reference [REF]

\subsection{Gravity}
\label{sec.gravity}

Solving for the accelerations of the cells and particles on the grid due to self-gravity involves three steps: (i) computing the total gravitating mass, (ii) solving for the gravitational potential field with the appropriate boundary conditions, and (iii) differencing the potential to get the acceleration, and, if necessary, interpolating the acceleration back to the particles. These steps are described in detail below.

First, the massive (dark matter and star) particles are distributed onto the grids using the second-order cloud-in-cell (CIC) interpolation technique \citep{Hockney88} to form a spatially-discretized density field $\rho_{\rm DM}$.  During the CIC interpolation, particle positions are (temporarily) advanced by $0.5 \Delta t v^n$ so that we generate an estimate of the time-centered density field.  Particles on sub-grids within the grid's volume are also added to its gravitating field using the same method. In addition, since the gravitating field for a grid is defined beyond the grid edges (see below), massive particles from sibling grids and sub grids that lie within the entire gravitating field are used.  This step can involve communication.

Next, we add the baryonic grid densities in a similar fashion.  In particular, we treat baryonic cells as virtual CIC particles that are are placed at the grid center but are advanced by $0.5 \Delta t v^n$ in order to approximately time-center the gravitating mass field.  Cells that are covered by further-refined grids are treated in a similar way (i.e., we also use the subgrid cells as virtual CIC particles).  This procedure results in a total gravitating mass field $\rho_{\rm total}^{n+1/2}$.

To compute the potential field from this gravitating mass field on the root grid, we use a a fast Fourier transform. For periodic boundary conditions, we can use either a simple Greens function kernel of $-k^{-2}$, or the finite-difference equivalent \citep{Hockney88}:
\begin{equation}
G(\vec{k}) = - \frac{\Delta x}{2 \left( \sin(k_x \Delta x/2)^2 + \sin(k_y \Delta y/2)^2 + \sin(k_z \Delta z/2)^2 \right) }
\end{equation}
where $k^2 = k_x^2 + k_y^2 + k_z^2$ is the wavenumber in Fourier space and the potential is calculated in k-space as usual with $\tilde{\phi}(k) = G(k) \tilde{\rho}(k)$.  

For isolated boundary conditions, we use the James method \citep{James77}.  In this case, the Greens function is generated in real-space so as to have the correct zero-padding properties and then transformed into the Fourier domain.  In both cases, the potential is then transformed back into the real domain to get potential values at the cell centers.  These are differenced with a two-point centered difference scheme to obtain accelerations at the cell centers (except if we are using the staggered Zeus-like solver, in which case the accelerations are computed at the cell faces to match the velocities).  Particle accelerations are obtained using a (linear) CIC interpolation from the grid.

In order to calculate more accurate potentials on the subgrids, \enzo\ uses a similar but slightly different technique from the root grid.  The generation of the total gravitating mass field is essentially identical, using CIC interpolation for both the particles and baryons, including sub-grids as before.  To compute the potential on subgrids, however, we use the standard seven-point (in three dimensions) second-order finite-difference approximation to Poisson's equation.  Boundary conditions are then interpolated from the potential values on the parent grid.  We use either tri-linear interpolation or a natural second-order spline for this: both methods give similar results but we generally choose to use tri-linear interpolation, which empirically provides a resonable compromise between speed and accuracy. Each scheme can be selected via a run-time parameter. The potential equation on each subgrid is then solved with the given Dirichlet boundary conditions with a multigrid relaxation technique.  This is applied to each subgrid separately.

The region immediately next to the boundary can contain unwanted oscillations \citep[e.g.,][]{Anninos94}, and so we use an expanded buffer zone around the grid, of size three parent grid boundary zones (so typically six refined zones for a refinement factor of 2).  The density is computed in this region and the potential solved, but only the region that overlaps with the active region of the grid itself is used to calculate accelerations.

Simply interpolating the potential without feeding it back to higher levels leads to errors in the potential at more refined levels, due to the build-up of errors during the interpolation of coarse boundary values.  In addition, neighboring subgrids are not guaranteed to generate the same potential values because of the lack of a coherent potential solve across the whole hierarchy. In an attempt to partially alleviate this problem, we allow for an iterative procedure across sibling grids, in which the potential values on the boundary of grids can be updated with the potential in `active' regions of neighboring subgrids.  To prevent overshoot, we average the potential on the boundary and allow for (typically) 4 iterations.  This procedure can help in many cases, but does not necessarily produce a coherent solution across all grids and so does not completely solve the problem; we are working on a slower but more accurate method that does a multigrid solve across the whole grid (Reynolds \etal, in preparation).

At this point it is useful to emphasize that the effective force resolution of an adaptive particle-mesh calculation is approximately twice as coarse as the grid spacing at a given level of resolution.
