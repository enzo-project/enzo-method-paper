\subsection{Gravity}
\label{sec.gravity}

Solving for the accelerations for the cells and particles in a grid due to self-gravity involves three steps: first, computing the total gravitating mass; second, solving for the potential field with the appropriate boundary conditions, and finally, differencing the potential to get the acceleration and, if necessary, interpolating the potential back to the particles. These steps are described in detail below.

First, the dark matter particles are distributed onto the grids using the cloud-in-cell (CIC) interpolation technique \citet{Hockney88} to form a spatially discretized density field $\rho_{\rm DM}$.  When doing the CIC interpolation, particle positions are (temporarily) advanced by $0.5 \Delta t v^n$ so that we generate an estimate of the time-centered density field.  Particles on sub-grids are also added in this way.  In addition, since the gravitating field for a grid is defined beyond the grid edges (see below), dark matter particles from sibling grids and sub grids which lie within the entire gravitating field are used.

Next, we add the baryonic grid densities in a similar fashion.  In particular, we treat baryonic cells as virtual CIC particles which are are placed at the grid center but are advanced by $0.5 \Delta t v^n$ in order to approximately time-center the gravitating mass field.  Cells which are covered by further refined grids are treated in a similar way (i.e. we use the sub grid cells as virtual CIC particles).  This procedure results in a total gravitating mass field $\rho_{\rm total}^{n+1/2}$.

To compute the potential field from this gravitating mass field on the root grid, we use a a fast Fourier transform. For periodic boundary conditions, we can use either a simple Greens function kernel of $-k^{-2}$, or the finite-difference equivalent \citep{Hockney88}:
\begin{equation}
G(\vec{k}) = - \frac{\Delta x}{2 \left( sin(k_x \Delta x/2)^2 + sin(k_y \Delta y/2)^2 + sin(k_z \Delta z/2)^2 \right) }
\end{equation}
where $k^2 = k_x^2 + k_y^2 + k_z^2$ is the wavenumber in Fourier space and the potential is calculated in k-space as usual with $\tilde{\phi}(k) = G(k) \tilde{\rho}(k)$.  

For isolated boundary conditions, we use James method (REF).  In this case, the Greens function is generated in real-space so as to have the correct zero-padding properties and then transformed into the Fourier domain.  In either case, the potential is then transformed back into the real domain to get potential values at the cell centers.  These are differenced with a two-point centered difference to get accelerations at the cell centers (except if we are using the staggered Zeus-like solver, in which case the accelerations are computed at the cell faces to match the velocities).


In order to calculate more accurate potentials on subgrids, \enzo\ re-samples the dark matter density onto individual subgrids using the same CIC method as on the root grid, but at higher spatial resolution (and again adding baryon densities if applicable). Boundary conditions are then interpolated from the potential values on the parent grid.  We use either tri-linear interpolation or a natural second-order spline: both methods seem to give similar results. The potential equation on each subgrid is then solved with the given Dirichlet boundary conditions.  This solution is done with a multigrid relaxation technique (but only applied to each subgrid).

The region immediately next to the boundary can contain unwanted oscillations (Anninos \etal REF), and so we use an expanded buffer zone around the grid, of size three root grid boundary zones (so typically six refined zones).  The density is computed in this region and the potential solved, but only the region which overlaps with the grid itself is used to calculate accelerations.

It is known (REF) that simply interpolating the potential without feeding it back to higher levels leads to errors in the potential at more refined levels because of the build-up of errors during the interpolation of coarse boundary values.  
In addition, neighboring subgrids are not guaranteed to generate the same potential values because of the lack of a coherent potential solve across the whole hierarchy. In an attempt to partially alleviate this problem, we allow for an iterative procedure across sibling grids, in which the potential values on the boundary of grids can be updated with the potential in `active' regions of neighboring subgrids.  To prevent overshoot, we average the potential on the boundary and allow for (typically) 4 iterations.  This procedure can help in many cases, but does not produce a coherent solution across all grids and so does not solve the problem; we are working on a slower but more accurate method which does a multi-grid solve across the whole grid (Reynolds \etal, in preparation).

Forces are computed on the mesh by finite-differencing potential values and are then interpolated to the particle positions, where they are used to update the particle's position and velocity information.  Potentials on child grids are computed recursively and particle positions are updated using the same timestep as in the hydrodynamic equations.  

At this point it is useful to emphasize that the effective force resolution of an adaptive particle-mesh calculation is approximately twice as coarse as the grid spacing at a given level of resolution.  
%The potential is solved in each grid cell; however, the quantity of interest, namely the acceleration, is the gradient of the potential, and hence two potential values are required to calculate this.  In addition, it should be noted that the adaptive particle-mesh technique described here is very memory intensive: in order to ensure reasonably accurate force resolution at grid edges the multigrid relaxation method used in the code requires a layer of ghost zones which is very deep -- typically 6 cells in every direction around the edge of a grid patch.  This greatly adds to the memory requirements of the simulation, particularly because subgrids are typically small (on the order of $12^3 - 16^3$ real cells for a standard cosmological calculation) and ghost zones can dominate the memory and computational 
requirements of the code.

