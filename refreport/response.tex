\documentclass[11pt]{article}

\usepackage{fullpage}
\usepackage{natbib}
\usepackage{hyperref, breakurl, color}
\newcommand{\red}[1]{\textcolor{red}{#1}}
\newcommand{\blue}[1]{\textcolor{blue}{#1}}


\newcommand{\code}[1]{\textsf{#1}}
\newcommand{\enzo}{\code{Enzo}}

\def\apj{ApJ}
\def\apjl{ApJL}
\def\apjs{ApJS}
\def\mnras{MNRAS}
\def\aj{Ast.J.}
\def\nat{Nature}
\def\na{New.A.}
\newcommand{\aap}{A\&A}

\begin{document}

\begin{center} 
\bfseries{
\begin{large}
Response to referee report for manuscript  ApJ/ApJS92163
\end{large}
}
\end{center}


%%%%%%%%%%%%%%%%%%%%%%%%%%%%%%%%%%%%%%%%%%%%%%%%%%%%%%%%%%%%%%%%%%%%%%%%%%%%%

This is the response to the referee's report for ApJ/ApJS manuscript
\#92163, ``Enzo: An Adaptive Mesh Refinement Code for Astrophysics,''
by the Enzo collaboration (Bryan et al.).  We thank the referee for
their feedback, and address each comment separately below.

\section{Test problems}

Regarding test problems, the referee's comment is:

\begin{verbatim}
However, the paper is let down due to the lack of tests presented for
many of the algorithms presented in the paper. In particular, the only
test presented for the MHD algorithm is the Orszag-Tang vortex test,
which 'does not seem to be extremely discriminating for MHD
algorithms' (Stone et al., 2008). In addition, results from some
tests, such as the photo-evaporation test, do not seem to be properly
compared to prior results in the literature (e.g. Jiang et
al. 2012). Finally, no tests appear to be presented for some tests
entirely, e.g. the Chemistry algorithms presented in Section 6.1.
\end{verbatim}

We have modified the test problem section as follows:

\begin{itemize}
\item We have added a section (11.2.10) showing how both \enzo\ MHD
  algorithms perform on the standard Brio \& Wu shock tube test
  problem, which is a standard test of MHD solvers.  Both \enzo\ MHD
  algorithms match the reference solution to high precision.  We retain the
  Orszag-Tang Vortex test as we feel that it is interesting to the
  intended audience.

\item We have added a test of the non-equilibrium primordial chemistry code that
  tests the ionization balance in ionized gas as compared to an
  analytical solution (Section 11.2.13).  As can be seen, \enzo's
  non-equilibrium chemistry solver  reproduces the analytical solution perfectly.

\item The ``Photo-evaporation of a dense clump'' test problem (Section
  11.2.14) has been expanded to include a new figure (Fig. 18) that
  contains much more detail about the evolution of the test problem,
  and comparison to prior results, including the MORAY method paper
  \citep{Wise11_Moray} and the Cosmological Radiative Transfer
  Comparison Project paper \citep{IlievEtAl2009}.

\end{itemize}

We also note that many of the fundamental \enzo\ algorithms are
extensively tested in separate method papers that focus on a single
algorithm/physics package, and a single test problem or small number
of test problems are included in the full \enzo\ method paper as a
demonstration that the algorithms behave sensibly.  In the paper's
section or subsection
describing these algorithm, we cite the full method paper for the
appropriate algorithm and the direct the reader to it for more
information.  These additional algorithm method papers include ones
for both flavors of MHD contained in \enzo\ \citep{Wang:2009a,
Collins10}, the ray-tracing code \citep{Wise11_Moray}, and the
flux-limited diffusion \citep{ReynoldsHayesPaschosNorman2009}.

\section{Paper length and breakdown}

Finally, we would like to address the referee's feedback regarding the
paper's length and current organization.  The referee's comment is:

\begin{verbatim} 
Given the paper's length as it currently stands, it seems appropriate
to split the paper into two. One paper should document the basic
algorithms (e.g. AMR, hydro, MHD and gravity implementations) along
with an appropriate range of tests based on standard tests from the
literature. The second paper should document the microphysics options
(chemistry, radiation, thermal conductivity, etc.).
\end{verbatim}

We respectfully disagree with the referee regarding this point --
splitting the paper in two does not make sense, will not shorten the
overall page count, and will not improve the clarity of presentation. One of the strengths
of the \enzo\ code is the breadth of physics packages that are
available in the publicly-available version -- unlike, for example,
Gadget and Gadget-2 \citep{2001NewA....6...79S, 2005MNRAS.364.1105S},
where the only physics packages that are publicly available are dark
matter and adiabatic hydrodynamics.  Given this, it is logical that
the totality of the code's capabilities be documented in a single
place, for reference by future generations of computational scientists.

Regarding the paper's length, we argue that it is not excessively long
given the topics it covers, and thus does not need to be split into
multiple parts.  At present, the \enzo\ method paper is 62 pages long
in ApJ format.
For comparison:

\begin{itemize}
\item The paper for the open-source FLASH code
\citep{2000ApJS..131..273F} is 62 pages long -- almost identical in
length to the \enzo\ method paper.  At the time of the FLASH method paper's
publication, it covered the full feature set of the FLASH code, which
is comparable to the breadth of topcs covered in the \enzo\ method
paper.

\item The AREPO method paper \citep{2010MNRAS.401..791S} is 60 pages
long, covers a narrower range of topics, and the code is
closed-source.

\item The paper for the open-source Athena code
\citep{2008ApJS..178..137S} is 41 pages long, but only documents the
hydrodynamics and MHD algorithms.  At the time of the code's release,
this paper documented the entire feature set of the publicly-available
code.

\item The paper for the open-source ZEUS-MP 2 code
\citep{2006ApJS..165..188H} is 41 pages long, but again covers a
substantially narrower range of physics modules.  The entirety of the
publicly-available code is documented in this paper.
\end{itemize}

The only \textit{comparable} open-source code that we are aware of
whose method paper is broken up into multiple segments is the series
of three ZEUS-2D papers \citep{1992ApJS...80..753S,
1992ApJS...80..791S, 1992ApJS...80..819S}.  We argue, however, that
this is not a useful comparison, as the three papers were published in
the same issue of the Astrophysical Journal in 1992, and were
published back-to-back as consecutive articles in the same volume.  In
a time when the primary mode of consumption of articles was via paper
journals, this is effectively a single paper (with a length of
approximately 85 pages, no less).

We further note that splitting the paper up into two separate pieces
provides structural challenges, and will serve to increase the total
page count.  Structurally, Section 2 (the ``overview'' section) would
need to be split into two pieces, and the component that discusses the
microphysics will be somewhat confusing on its own, as these modules
act as sink and source terms for the fluid equations that would in
principle be described in a separate paper.  It's not clear where
Section 10 (``Analysis''), Section 12 (``Parallel strategy and
performance''), and Section 13 (``Code development methodology'')
would logically go -- some of these sections apply to all physics
packages in the code (e.g., Section 12), and others to none in
particular (e.g., Section 13).  Similarly, Section 11.1 (``Verifying
and validating the Enzo code'') describes the \enzo\ test suite, which
is used for all aspects of code testing (including both the
fluid/gravity/AMR algorithms \textit{and} the microphysics), and also
for code comparisons that test many different aspects of the code,
often in combination.  In order for these sections to make sense, and
to address test problems that are split between two papers, several of
these sections would need to be either duplicated or require an
extensive amount of cross-referencing between papers Furthermore,
separate introductions, discussions, and conclusions will by necessity
be mutually referential and, in some sense, duplicates of each other.
Together, this will not serve to either decrease page count or
increase clarity for the reader.

\section{Additional changes}

In addition to directly addressing concerns expressed by the referee,
we have fixed additional typos in the paper.  These include:

\begin{enumerate}
\item  We have corrected a typo in the equation describing the Jeans
  length refinement criterion (Section 3.6, Equation 29).

\item We have updated our description of the shear refinement
  criterion (Section 3.6, Equation 30).

\item We have corrected a typo in Table 4, where two photoionization
  rates were accidentally switched in the submitted version.

\item We have added missing acknowledgments.

\item We have updated the mercurial changeset hash to correspond to
  the \enzo\ 2.3 release (which was done slightly after the submission
  of the paper to ApJ).

\end{enumerate}


%%%%%%%%%%%%%%%%%%%%%%%%%%%%%%%%%%%%%%%%%%%%%%%%%%%%%%%%%%%%%%%%%%%%%%%%%%%%%

\newpage

\bibliographystyle{apj}
\bibliography{response}

\end{document}
