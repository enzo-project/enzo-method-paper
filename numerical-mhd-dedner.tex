\subsection{Hydrodynamics and Magnetohydrodynamics: MUSCL with Dedner cleaning}
\label{sec.num.hydro-muscl}

The second method we describe is a MUSCL-based solver than can be used in both HD and MHD modes.  The description here will be very brief both because the ideas are similar to those described in the previous section, and because this implementation has previously
been described in more detail elsewhere \citep{WangAbelZhang08, WangAbel09}.

Much like the PPM solver, we have three basic steps: the first is
reconstruction of the variables, the second is a solution of the
Riemann problem, and the third is is updating the conserved quanties
with the fluxes as written above.
For the reconstruction scheme we have implemented only the simple
piecewise linear reconstruction \citep{1979JCoPh..32..101V, 1985JCoPh..59..264C},
with options for both primitive and conservative variable reconstruction.
The available Riemann solvers are HLL, HLLC, and HLLD, as described earlier.

To more clearly describe the Dedner divergence cleaning modifications, we write the equations of compressible inviscid hydrodynamics can be written in the form of conservation laws as,
\begin{equation}
 \frac{\partial{U}}{\partial{t}} +
 \frac{\partial{F^x}}{\partial{x}} + \frac{\partial{F^y}}{\partial{y}} + \frac{\partial{F^z}}{\partial{z}}= 0, \label{hydro}
\end{equation}
The conserved variable $U$ is given by
\begin{equation}
 U = (\rho, \rho v_x, \rho v_y, \rho v_z, \rho E)^{T},
\end{equation} 
where $\rho$ is density, $v_i$ are the three components of velocity
for $i={x,y,z}$, $E=v^2/2 + e$ denotes the specific total energy and $e$ the
specific internal energy.

For the generalized Lagrange multiplier formulation of the MHD
equations \citep{2002JCoPh.175..645D}, we consider these 
conserved variables
\begin{equation}
 U = (\rho, \rho v_x, \rho v_y, \rho v_z, \rho E+B^2/2, B_x, B_y, B_z, \psi)^{T},
\end{equation} 
where $B_i$ with $i={x,y,z}$ are the three components of magnetic
fields and $\psi$ is the additional scalar field introduced in the GLM
formulation for the divergence cleaning.  The fluxes then are
\begin{eqnarray}
 F^x &=& (\rho v_x, \rho v_x^2+p+B^2/2-B_x^2, \rho v_yv_x-B_yB_x, \cr
 && \rho v_zv_x-B_zB_x, \rho ({v^2\over2} + h)v_x+B^2v_x-B_xB\cdot v, \cr
&& \psi, v_xB_y-v_yB_x, -v_zB_x+v_xB_z, c_h^2B_x)^T,\\
 F^y &=& (\rho v_y, \rho v_xv_y-B_xB_y, \rho v_y^2+p+B^2/2-B_y^2, \cr
 && \rho v_zv_y-B_zB_y, \rho ({v^2\over2} + h)v_y+B^2v_y-B_yB\cdot v, \cr
 && v_yB_z-v_zB_y,\psi,-v_xB_y+v_yB_x, c_h^2B_y)^{T}, \\
 F^z &=& (\rho v_z, \rho v_xv_z-B_xB_z, \rho v_yv_z-B_yB_z, \rho v_z^2+p+B^2/2-B_z^2, \cr
 && \rho ({v^2\over2} + h)v_z+B^2v_z-B_zB\cdot v, \cr
    &&  -v_yB_z+v_zB_y, v_zB_x-v_xB_z,\psi, c_h^2B_z)^{T},
\end{eqnarray}
where $c_h$ is a constant controlling the propagation speed and
damping rate of $\div B$, and $h=e+p/\rho$ denotes the enthalpy.
All quantities are cell-centered.

The method is dimensionally un-split in that the fluxes are computed
for all dimensions first and the conserved quantities are updated in
one step, in contrast to the Strang splitting employed in the other solvers.
Also unlike the other schemes, time-integration is done with a second-order
Runge-Kutta scheme \citep{1988JCoPh..77..439S}.
