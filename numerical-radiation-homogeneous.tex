\subsection{Homogeneous radiation background}
\label{sec.num.rad-homogeneous}

\enzo\ supports the use of a set of spatially uniform (but possibly time-varying) radiation fields that can interface with the chemistry and cooling/heating routines described elsewhere.  Many of these use fits to the H, He and He$^+$ ionizing and photo-ionization heating rates that are of the form
\begin{equation}
\rm{rate} = k_0 * (1+z)^{\alpha} * \exp{\left( \frac{\beta *(z-z_0)^2 }{1 + \gamma * (z+z_1)^2} \right)}
\label{eq:homo_rad_template}
\end{equation}
where the constant coefficients $\alpha$, $\beta$, $\gamma$, $z_0$ and $z_1$ are fits from the literature.  For radiation field types 1-3, we give the coefficients for the photoionization (and photo-heating) rates of H, He, and He$^+$ in Table~\ref{table:homo_coefs}.  Types 1 and 2 are based on \citet{1996ApJ...461...20H} with two different intrinsic quasar spectra slopes, while Type 3 is from \citet{2012ApJ...746..125H}, modified to match the normalized field found in \citet{Kirkman05}.  Type 4 is the same as Type 3 but also adds X-ray Compton heating from \citet{MadauEfstathiou99}, using equations (4) and (11) of that paper.

Homogeneous radiation field types 5 and 6 start with a spectral shape which is then integrated against the appropriate H, He, He$^+$ cross-sections to compute the ionization and photo-ioniziation heating rates (we use 400 bins logarithmically specced from 0.74 eV to $7.24 \times 10^9$ eV).  In particular, Type 5 has a hard, featureless quasar-like power law spectrum $f_{\nu} = f_{\rm HI} \nu^{\alpha_0}$, where $\alpha_0 = 1.5$ by default and the spectrum is normalized at the HI ionization edge.  Type 6 has the same spectrum by is attenuated by a column density of $10^{22}$ cm$^{-3}$ neutral hydrogen.  

Types 8, 9, and 14 have only photo-dissociating Lyman-Werner flux, with Type 8 being constant and Type 9 using the redshift-dependent results of \citet{TrentiStiavelli09}.  Type 14 uses a fit from \citet{WiseAbel05} for the range $6 < z < 50$, constant for $z<6$, and zero for $z > 50$.  Other types are undefined or unused.

\begin{table}
\begin{center}
\caption{Homogeneous radiation field coefficients}
\begin{tabular}{lrrrrrr}
\tableline\tableline
{Element} & {$k_0$} &  {$\alpha$} & {$\beta$} & {$\gamma$} & {$z_0$} & {$z_1$}  \\
\tableline
\multicolumn{7}{c}{Radiation Type 1 \citep{1996ApJ...461...20H} for the case $\alpha_q = 1.5$} \\
\tableline
H ionization & $6.7 \times 10^{-13}$ & 0.43 & 1/1.95 & 0 & 2.3 & 0 \\
He ionization & $6.3 \times 10^{-15}$ & 0.51 & 1/2.35 & 0 & 2.3 & 0 \\
He$^+$ ionization & $3.2 \times 10^{-13}$ & 0.50 & 1/2.00 & 0 & 2.3 & 0 \\
H heating & $4.7 \times 10^{-24}$ & 0.43 & 1/1.95 & 0 & 2.3 & 0 \\
He heating & $8.2 \times 10^{-24}$ & 0.50 & 1/2.00 & 0 & 2.3 & 0 \\
He$^+$ heating & $1.6 \times 10^{-25}$ & 0.51 & 1/2.35 & 0 & 2.3 & 0 \\
\tableline
\multicolumn{7}{c}{Radiation Type 2 \citep{1996ApJ...461...20H} for the case $\alpha_q = 1.8$} \\
\tableline
H ionization & $5.6 \times 10^{-13}$ & 0.43 & 1/1.95 & 0 & 2.3 & 0 \\
He ionization & $3.2 \times 10^{-15}$ & 0.30 & 1/2.60 & 0 & 2.3 & 0 \\
He$^+$ ionization & $4.8 \times 10^{-13}$ & 0.43 & 1/1.95 & 0 & 2.3 & 0 \\
H heating & $3.9 \times 10^{-24}$ & 0.43 & 1/1.95 & 0 & 2.3 & 0 \\
He heating & $6.4 \times 10^{-24}$ & 0.43 & 1/2.10 & 0 & 2.3 & 0 \\
He$^+$ heating & $8.7 \times 10^{-26}$ & 0.30 & 1/2.70 & 0 & 2.3 & 0 \\
\tableline
\multicolumn{7}{c}{Radiation Type 3 modified \citep{2012ApJ...746..125H}} \\
\tableline
H ionization &        $1.04 \times 10^{-12}$ & 0.231 & -0.6818 & 0.1646 & 1.855 & 0.3097 \\
He ionization &       $1.84 \times 10^{-14}$ & -1.038 & -1.1640 & 0.1940 & 1.973 & -0.6561 \\
He$^+$ ionization & $5.79 \times 10^{-13}$ & 0.278 & -0.8260 & 0.1730 & 1.973 & 0.2880 \\
H heating &            $8.86 \times 10^{-25}$ & -0.0290 & -0.7055 & 0.1884 & 2.003 & 0.2888 \\
He heating &         $5.86 \times 10^{-24}$ & 0.1764 & -0.8029 & 0.1732 & 2.088 & 0.1362 \\
He$^+$ heating &   $2.17 \times 10^{-25}$ & -0.2196 & -1.070 & 0.2124 & 1.1782 & -0.9213 \\
\end{tabular}
%\tablecomments{}
\label{table:homo_coefs}
\end{center}
\end{table}