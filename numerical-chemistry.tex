\subsection{Chemistry}
\label{sec.num.chemistry}

While it is often safe to assume that species (both chemical and
ionization) within a gas can be treated as being in equilibrium, in
some regimes that are found in astrophysics this is not true.  For
example, the cooling and collapse of primordial gas in Population III
star formation is dominated by molecular hydrogen, which in the
absence of dust forms via an inefficient pair of collisional processes
that depend heavily on the local population of free electrons.  As a
result, when modeling primordial star formation, it is critical to
follow the non-equilibrium evolution of the chemical species of
hydrogen, including molecular hydrogen and deuterium.

The primordial non-equilibrium chemistry routines used in \enzo\ were
first described by Abel et al. and Anninos et
al. \citep{abel97,anninos97}, but have been extended since then with
updated reaction rates and the inclusion of deuterium species
\citep{2009PhDT.........5T}.  These routines follow the
non-equilibrium chemistry of a gas of primordial composition with 12
total species: H, \Hp, He, \Hep, \Hepp, \Hm, \HHp, \HH, e$^-$, D, \Dp,
and HD.  (A separate, but related, module follows the radiative
heating and cooling of the gas from atomic and molecular line
excitation, recombination, collisional excitation, free-free
transitions, Compton scattering of the cosmic microwave background, as
well as several models for a metagalactic UV background that heat the
gas via photoionization and discussion -- see
Section~\ref{sec.num.cooling} for more details).  Depending on user
parameters, one can follow simply atomic species (H, \Hp, He, \Hep,
\Hepp, and e$^-$), include those relevant for molecular hydrogen
formation (\HH, \HHp, and \Hm), and further include deuterium and its
species (D, \Dp, and HD).  \red{A total of 9 kinetic equations are
  solved, including 29 kinetic and radiative processes, for the 9
  species mentioned above.  See Table~\ref{table.collisional} for the
  collisional processes and Table~\ref{table.radiative} for the
  radiative processes solved.}

The chemical reaction equation network is technically challenging to
solve due to the huge range of reaction timescales involved -- the
characteristic creation and destruction timescales of the various
species and reactions can differ by many orders of magnitude.  As a
result, the set of rate equations is extremely stiff, and an explicit
scheme for integration of the rate equations can be exceptionally
costly if small enough timesteps are taken to keep the network stable.
This makes an implicit scheme much more preferable for such a set of
equations.  However, an implicit scheme typically require an iterative
procedure to converge, and for large networks (such as this one) an
implicit method can be very time-consuming, making it undesirable for
a large, three-dimensional simulation.

\enzo\ solves the rate equations using a method based on a backwards
differencing formula (BDF) in order to provide a stable and accurate
solution. This technique is optimized by taking the chemical
intermediaries \Hm and \HHp, which have large rate coefficients
and low concentrations, and grouping them into a separate category of
equations.  Due to their fast reactions these species are very
sensitive to small changes in the more abundant species.  However, due
to their low overall concentrations, they do not significantly
influence the abundance of species with higher concentrations.
Therefore, reactions involving these two species can be decoupled from
the rest of the network and treated independently.  In fact, \Hm~and
\HHp~are treated as being in equilibrium at all times, independent
of the other species and the hydrodynamic variables.  This allows a
large speedup in solution as the BDF scheme is then applied only to
the slower 7- or 10-species network (depending on whether deuterium is
included or not) on timescales closer to those required by the
hydrodynamics of the simulation.  Even so, the accuracy and stability
of the scheme is maintained by subcycling the rate solver over a
single hydrodynamic timestep.  These subcycle timesteps are determined
so that the maximum fractional change in the electron concentration is
limited to no more than $10\%$ per timestep.  Given this subcycling,
the method is not truly implicit over an entire hydrodynamical
timestep, and thus not truly a BDF method -- however, it uses the
formalism and is implicit over subcycle timesteps.


It is important to note the regime in which this model is valid.
According to \citet{abel97} and \citet{anninos97},
the reaction network is valid for temperatures between $10^0 - 10^8$
K.  The original model discussed in these two references is only valid
up to $n_H \sim 10^4$~cm$^{-3}$.  However, addition of the 3-body
\HH~formation process (equation 20 in Table~\ref{table.collisional})
allows correct modeling of the chemistry of the gas up until the point
where collisionally-induced emission from molecular hydrogen becomes
an important cooling process, which occurs at $n_{\rm H} \sim
10^{14}$~cm$^{-3}$.  A further concern is that the optically thin
approximation for radiative cooling breaks down, which occurs before
$n_{\rm H} \sim 10^{16} - 10^{17}$~cm$^{-3}$.  Beyond this point,
modifications the cooling function that take into account the
non-negligible opacity in the gas must be made, as discussed by
\citet{2004MNRAS.348.1019R}, and was put into \enzo\ for the work
published in \citep{2009Sci...325..601T,2009PhDT.........5T}.  Even
with these modifications, a completely correct description of the
cooling of this gas will require some form of radiation transport,
which will greatly increase the cost of the simulations.

%---------------- table of collisional processes

\begin{table}
\begin{center}
{\bfseries Collisional Processes}\\[1ex]
\begin{tabular}{llllllll}
(1) & H & + & e$^-$ & $\rightarrow$ & H$^+$ &+& 2e$^-$ \\
(2) & H$^+$ &+ &e$^-$ & $\rightarrow$ & H &+ &$\gamma$ \\
(3) & He &+& e$^-$ & $\rightarrow$ & He$^+$ &+& 2e$^-$  \\
(4) & He$^+$ &+& e$^-$ & $\rightarrow$ & He &+ &$\gamma$  \\
(5) & He$^{+}$ &+& e$^-$ & $\rightarrow$ & He$^{++}$ &+& 2$e^-$  \\
(6) & He$^{++}$ &+& e$^-$ & $\rightarrow$ & He$^+$ &+& $\gamma$ \\
\hline
(7) & H &+& e$^-$ &$\rightarrow$& H$^-$ &+& $\gamma$  \\
(8) & H$^-$ &+& H &$\rightarrow$ & H$_2$ & +& e$^-$ \\
(9) & H &+ &H$^+$ &$\rightarrow$ &H$_2^+$ &+ &$\gamma$ \\
(10) & H$_2^+$ &+ &H &$\rightarrow$ &$H_2$ &+ &$H^+$ \\
(11) & H$_2$ &+ &H$^+$ &$\rightarrow$ &H$_2^+$ & +& H \\
(12) & H$_2$ &+ &e$^-$ & $\rightarrow$ & 2H & + & e$^-$  \\
(13) & H$_2$ & + & H & $\rightarrow$ & 3H &   &      \\
(14) & H$^-$ & + & e$^-$ & $\rightarrow$ & H & + & 2e$^-$ \\
(15) & H$^-$ & + & H & $\rightarrow$ & 2H & + & e$^-$ \\ 
(16) & H$^-$ & + & H$^+$ & $\rightarrow$ & 2H & & \\
(17) & H$^-$ & + & H$^+$ & $\rightarrow$ & H$_2^+$ & + & e$^-$ \\
(18) & H$_2^+$ & + & e$^-$ & $\rightarrow$ & 2H & & \\
(19) & H$_2^+$ & + & H$^-$ & $\rightarrow$ & H$_2$ & + & H  \\
(20) & 2H & + & H$_2$ & $\rightarrow$ & 2H$_2$ &  &   \\
(21) & 2H & + & H & $\rightarrow$ & H$_2$ & + & H  \\
(22) & H$_2$ & + & H$_2$ & $\rightarrow$ & H$_2$ & + & 2H  \\
(23) & 3H & & & $\rightarrow$ & H$_2$ & + & H \\
\hline
(24) & D & + & e$^-$ & $\rightarrow$ & D$^+$ &+& 2e$^-$ \\
(25) & D$^+$ &+ &e$^-$ & $\rightarrow$ & D &+ &$\gamma$ \\
(26) & H$^+$ &+ &D & $\rightarrow$ & H &+ &D$^+$ \\
(27) & H &+ &D$^+$ & $\rightarrow$ & H$^+$ &+ &D \\
(28) & H$_2$ &+ &D$^+$ & $\rightarrow$ & HD &+ &D$^+$ \\
(29) & HD &+ &H$^+$ & $\rightarrow$ & H$_2$ &+ &D$^+$ \\
(30) & H$_2$ &+ &D & $\rightarrow$ & HD &+ &H \\
(31) & HD &+ &H & $\rightarrow$ & H$_2$ &+ &D \\


\end{tabular}
\caption[]{Collisional processes solved in the Enzo nonequilibrium
primordial chemistry routines.}
\label{table.collisional}
\end{center}
\end{table}



\begin{table}
\begin{center}
{\bfseries Radiative Processes}\\[1ex]
\begin{tabular}{llllllll}
(32) & H & + & $\gamma$ & $\rightarrow$ & H$^+$ & + & e$^-$ \\
(33) & He & + & $\gamma$ & $\rightarrow$ & He$^+$ & + & e$^-$ \\
(34) & He$^+$ & + & $\gamma$ & $\rightarrow$ & He$^{++}$ & + & e$^-$ \\
(35) & H$^-$ & + & $\gamma$ & $\rightarrow$ & H & + & e$^-$ \\
(36) & H$_2$ & + & $\gamma$ & $\rightarrow$ & H$_2^+$ & + & e$^-$ \\
(37) & H$_2^+$ & + & $\gamma$ & $\rightarrow$ & H & + & H$^+$ \\
(38) & H$_2^+$ & + & $\gamma$ & $\rightarrow$ & 2H$^+$ & + & e$^-$ \\
(39) & H$_2$ & + & $\gamma$ & $\rightarrow$ & H$_2^*$ & $\rightarrow$ & 2H \\
(40) & H$_2$ & + & $\gamma$ & $\rightarrow$ & 2H &  & 
\end{tabular}
\caption[]{Radiative processes solved in the Enzo nonequilibrium
primordial chemistry routines. \red{Missing rates for HD destruction/ionization?}}
\label{table.radiative}
\end{center}
\end{table}

%%% Local Variables: 
%%% mode: latex
%%% TeX-master: "ms"
%%% End: 
