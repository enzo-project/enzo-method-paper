\subsection{Chemistry}
\label{sec.num.chemistry}

While it is often safe to assume that species (both chemical and ionization)
within a gas can be treated as being in equilibrium, in some regimes that are
found in astrophysics this assumption leads to considerable error.  For
example, the cooling and collapse of primordial gas in Population III star
formation is dominated by molecular hydrogen, which in the absence of dust
forms via an inefficient pair of collisional processes that depend heavily on
the local, highly non-equilibrium population of free electrons.  As a result,
when modeling primordial star formation, it is critical to follow the
non-equilibrium evolution of the chemical species of hydrogen, including
molecular hydrogen and deuterium.

The primordial non-equilibrium chemistry routines used in \enzo\ were first
described by Abel et al. and Anninos et al. \citep{abel97,anninos97}, but have
been extended since then with updated reaction rates and the inclusion of
deuterium species \citep{2009PhDT.........5T}.  These routines follow the
non-equilibrium chemistry of a gas of primordial composition with 12 total
species: H, \Hp, He, \Hep, \Hepp, \Hm, \HHp, \HH, e$^-$, D, \Dp, and HD.  Enzo
also computes the radiative heating and cooling of the gas from atomic and
molecular line excitation, recombination, collisional excitation, free-free
transitions, Compton scattering of the cosmic microwave background, as well as
several models for a metagalactic UV background that heat the gas via
photoionization and photodissociation; see Section~\ref{sec.num.cooling} for
more details.  The chemical and thermal states of the gas can be updated either
at the same hydrodynamical timestep (i.e., de-coupled and operator split) or
through the same subcycling system (i.e., a coupled chemical and thermal
system).  The default behavior of \enzo\ is to couple these two systems at
subcycles of the hydrodynamic timestep; this results in updates to both the
chemical and thermal states of the gas (which also inform the temperature, the
reaction rate coefficients and the cooling functions of the gas) on timescales
that faster than those of the gas dynamics.

Input parameters to \enzo\ govern the chemical species that are updated during
the course of the simulation.  This can include only the atomic species (H,
\Hp, He, \Hep, \Hepp, and e$^-$), those species relevant for molecular hydrogen
formation (\HH, \HHp, and \Hm), and can further include deuterium and its
species (D, \Dp, and HD).  A total of 9 kinetic equations are solved, including
29 kinetic and radiative processes, for the 12 species mentioned above.  See
Table~\ref{table.collisional} for the collisional processes and
Table~\ref{table.radiative} for the radiative processes solved.

The chemical reaction equation network is technically challenging to solve due
to the huge range of reaction timescales involved.  The characteristic times
for creation and destruction of the various species and reactions can differ by
many orders of magnitude and are often very sensitive to the chemical and
thermal state of the gas.  This makes a fully-implicit scheme, with convergence
criteria and error tolerance, strongly preferable for such a set of equations.
However, most implicit schemes require an iterative procedure to converge, and
for large networks (such as this one) an iterative, fully-implicit method can
be very time-consuming and computationally costly for a relatively small
increase in accuracy.  At this present time, this makes fully-implicit methods
somewhat undesirable for a large, three-dimensional simulation.

\enzo\ solves the rate equations using a method based on a semi-implicit
formulation in order to provide a stable, positive definite and first-order
accurate solution.  The update discretization splits chemical changes into
formation components and destruction components and updated with a mixed set of
time states, as described in \citet{anninos97}.  The formation components of
species $S_i$ are computed at the current subcycle time, where the contribution
of species $S_i$ to its own destruction components are computed at the updated
time; all other contributions to the destruction component are computed at the
current time.  This mixed state improves accuracy and ensures species values
are positive definite, and is equivalent to one Jacobi iteration of an implicit
Euler solve.  This technique is optimized by taking the chemical intermediaries
\Hm and \HHp, which have large rate coefficients and low concentrations, and
grouping them into a separate category of equations.  Due to their fast
reactions, these species are very sensitive to small changes in the more
abundant species and are at all times in astrophysical calculations close to
equilibrium values.  Attempting to resolve their formation and destruction
times would necessitate extremely small timesteps, and while these species
provide necessary channels to facilitate the formation of molecular hydrogen at
low densities, they do not otherwise significantly influence the abundance of
species with higher concentrations.  Therefore, reactions governing these two
species can be decoupled from the rest of the network and treated independently
through analytic solutions for equilibrium values.  This allows a large speedup
in solution as the timestepping scheme is applied only to the slower 7- or
10-species network (depending on whether deuterium is included or not), which
will be much closer to the overall hydrodynamic timestep of the simulation.

Even so, the accuracy and stability of the scheme is maintained by subcycling
the rate solver within a single hydrodynamic timestep.  These subcycle
timesteps are determined so that the estimated fractional change in the
electron concentration is limited to no more than $10\%$ per timestep;
additional criteria may be applied based on the expected change from radiative
cooling, chemical heating from the formation of molecular hydrogen.

It is important to note the regime in which this model is valid.
According to \citet{abel97} and \citet{anninos97},
the reaction network is valid for temperatures between $10^0 - 10^8$
K.  The original model discussed in these two references is only valid
up to $n_H \sim 10^4$~cm$^{-3}$.  However, addition of the 3-body
\HH~formation process (equation 20 in Table~\ref{table.collisional})
allows correct modeling of the chemistry of the gas up until the point
where collisionally-induced emission from molecular hydrogen becomes
an important cooling process, which occurs at $n_{\rm H} \sim
10^{14}$~cm$^{-3}$.  A further concern is that the optically thin
approximation for radiative cooling eventually breaks down, which occurs before
$n_{\rm H} \sim 10^{16} - 10^{17}$~cm$^{-3}$ in gas of primordial composition.  Beyond this point,
modifications to the cooling function that take into account the
non-negligible opacity in the gas must be made, as discussed by
\citet{2004MNRAS.348.1019R}, and was put into \enzo\ for the work
published in \citep{2009Sci...325..601T,2009PhDT.........5T}.   The formation
of molecular hydrogen as catalyzed by dust was recently added to \enzo\ to
enable studies of low-metallicity gas, as well as the inclusion of appropriate
timestepping criteria to account for the input of ionizing radiation.  Even
with these modifications, a completely correct description of the cooling of
primordial gas at very high densities requires some form of radiation
transport, which will greatly increase the cost of the simulations.
Furthermore, at very high densities, the stiffness of the molecular hydrogen
reaction rates may require better than a first-order accurate solution; as
such, the transition to this regime will likely necessitate a fully-implicit,
iterative solver.

%---------------- table of collisional processes

\begin{table}
\begin{center}
{\bfseries Collisional Processes}\\[1ex]
\begin{tabular}{llllllll}
(1) & H & + & e$^-$ & $\rightarrow$ & H$^+$ &+& 2e$^-$ \\
(2) & H$^+$ &+ &e$^-$ & $\rightarrow$ & H &+ &$\gamma$ \\
(3) & He &+& e$^-$ & $\rightarrow$ & He$^+$ &+& 2e$^-$  \\
(4) & He$^+$ &+& e$^-$ & $\rightarrow$ & He &+ &$\gamma$  \\
(5) & He$^{+}$ &+& e$^-$ & $\rightarrow$ & He$^{++}$ &+& 2$e^-$  \\
(6) & He$^{++}$ &+& e$^-$ & $\rightarrow$ & He$^+$ &+& $\gamma$ \\
\hline
(7) & H &+& e$^-$ &$\rightarrow$& H$^-$ &+& $\gamma$  \\
(8) & H$^-$ &+& H &$\rightarrow$ & H$_2$ & +& e$^-$ \\
(9) & H &+ &H$^+$ &$\rightarrow$ &H$_2^+$ &+ &$\gamma$ \\
(10) & H$_2^+$ &+ &H &$\rightarrow$ &$H_2$ &+ &$H^+$ \\
(11) & H$_2$ &+ &H$^+$ &$\rightarrow$ &H$_2^+$ & +& H \\
(12) & H$_2$ &+ &e$^-$ & $\rightarrow$ & 2H & + & e$^-$  \\
(13) & H$_2$ & + & H & $\rightarrow$ & 3H &   &      \\
(14) & H$^-$ & + & e$^-$ & $\rightarrow$ & H & + & 2e$^-$ \\
(15) & H$^-$ & + & H & $\rightarrow$ & 2H & + & e$^-$ \\ 
(16) & H$^-$ & + & H$^+$ & $\rightarrow$ & 2H & & \\
(17) & H$^-$ & + & H$^+$ & $\rightarrow$ & H$_2^+$ & + & e$^-$ \\
(18) & H$_2^+$ & + & e$^-$ & $\rightarrow$ & 2H & & \\
(19) & H$_2^+$ & + & H$^-$ & $\rightarrow$ & H$_2$ & + & H  \\
(20) & 2H & + & H$_2$ & $\rightarrow$ & 2H$_2$ &  &   \\
(21) & 2H & + & H & $\rightarrow$ & H$_2$ & + & H  \\
(22) & H$_2$ & + & H$_2$ & $\rightarrow$ & H$_2$ & + & 2H  \\
(23) & 3H & & & $\rightarrow$ & H$_2$ & + & H \\
\hline
(24) & D & + & e$^-$ & $\rightarrow$ & D$^+$ &+& 2e$^-$ \\
(25) & D$^+$ &+ &e$^-$ & $\rightarrow$ & D &+ &$\gamma$ \\
(26) & H$^+$ &+ &D & $\rightarrow$ & H &+ &D$^+$ \\
(27) & H &+ &D$^+$ & $\rightarrow$ & H$^+$ &+ &D \\
(28) & H$_2$ &+ &D$^+$ & $\rightarrow$ & HD &+ &D$^+$ \\
(29) & HD &+ &H$^+$ & $\rightarrow$ & H$_2$ &+ &D$^+$ \\
(30) & H$_2$ &+ &D & $\rightarrow$ & HD &+ &H \\
(31) & HD &+ &H & $\rightarrow$ & H$_2$ &+ &D \\


\end{tabular}
\caption[]{Collisional processes solved in the Enzo nonequilibrium
primordial chemistry routines.}
\label{table.collisional}
\end{center}
\end{table}



\begin{table}
\begin{center}
{\bfseries Radiative Processes}\\[1ex]
\begin{tabular}{llllllll}
(32) & H & + & $\gamma$ & $\rightarrow$ & H$^+$ & + & e$^-$ \\
(33) & He & + & $\gamma$ & $\rightarrow$ & He$^+$ & + & e$^-$ \\
(34) & He$^+$ & + & $\gamma$ & $\rightarrow$ & He$^{++}$ & + & e$^-$ \\
(35) & H$^-$ & + & $\gamma$ & $\rightarrow$ & H & + & e$^-$ \\
(36) & H$_2$ & + & $\gamma$ & $\rightarrow$ & H$_2^+$ & + & e$^-$ \\
(37) & H$_2^+$ & + & $\gamma$ & $\rightarrow$ & H & + & H$^+$ \\
(38) & H$_2^+$ & + & $\gamma$ & $\rightarrow$ & 2H$^+$ & + & e$^-$ \\
(39) & H$_2$ & + & $\gamma$ & $\rightarrow$ & H$_2^*$ & $\rightarrow$ & 2H \\
(40) & H$_2$ & + & $\gamma$ & $\rightarrow$ & 2H &  & 
\end{tabular}
\caption[]{Radiative processes solved in the Enzo nonequilibrium
primordial chemistry routines. \red{Missing rates for HD destruction/ionization?}}
\label{table.radiative}
\end{center}
\end{table}

%%% Local Variables: 
%%% mode: latex
%%% TeX-master: "ms"
%%% End: 
