\subsection{Chemistry}
\label{sec.num.chemistry}

While it is often safe to assume that species (both chemical and
ionization) within a gas can be treated as being in 
equilibrium, in some regimes that are found in astrophysics this is not true.  For example, the cooling and collapse of primordial gas in Population III
star formation is dominated by molecular hydrogen, which in the
absence of dust forms via an inefficient pair of collisional processes
that depend heavily on the local population of free electrons.  
As a
result, when modeling primordial star formation, it is critical to
follow the non-equilibrium evolution of the chemical species of
hydrogen, including molecular hydrogen and deuterium.

The primordial non-equilibrium chemistry routines used in \enzo\ were
first described by Abel et al. and Anninos et
al. \citep{abel97,anninos97}, but have been extended since then with
updated reaction rates and the inclusion of deuterium species.  These 
routines follow the non-equilibrium chemistry of a 
gas of primordial composition with 12 total species:  
$H$, $H^+$, $He$, $He^+$, $He^{++}$, $H^-$, $H_2^+$, $H_2$, $e^-$,
$D$, $D^+$, and $HD$.  (A separate, but related, module follows the
radiative heating and cooling of the gas from atomic and molecular line excitation,
recombination, collisional excitation, free-free transitions, Compton
scattering of the cosmic microwave background, as well as several
models for a metagalactic UV background that heat the gas via
photoionization and discussion -- see Section~\ref{sec.num.cooling}
for more details).  Depending on user parameters, one can follow
simply atomic species ($H$, $H^+$, $He$, $He^+$, $He^{++}$,  and
$e^-$), include those relevant for molecular hydrogen formation
($H_2$, $H_2^+$, and $H^-$), and further include deuterium and its
species ($D$, $D^+$, and $HD$).  
\red{A total of 9 kinetic equations are solved, including 29 kinetic and radiative 
processes, for the 9 species mentioned above.  See Table~\ref{table.collisional} 
for the collisional processes and Table~\ref{table.radiative} for the 
radiative processes solved.}

The chemical reaction equation network is technically challenging to solve due to 
the huge range of reaction timescales involved -- the characteristic creation
and destruction timescales of the various species and reactions can differ by 
many orders of magnitude.  As a result, the set of rate equations is extremely 
stiff, and an explicit scheme for integration of the rate equations can be 
exceptionally costly if small enough timesteps are taken to keep the network
stable.  This makes an implicit scheme much more preferable for such a set of 
equations.  However, an implicit scheme typically require an iterative 
procedure to converge, and for large networks (such as this one) an implicit
method can be very time-consuming, making it undesirable for a large, three-dimensional
simulation.

\enzo\ solves the rate equations using a method based on a backwards differencing 
formula (BDF) in order to provide a stable and accurate solution.  This technique is
optimized by taking the chemical intermediaries $H^-$ and $H_2^+$, which have 
large rate coefficients and low concentrations, and grouping them into a separate
category of equations.  Due to their fast reactions these species are very sensitive
to small changes in the more abundant species.  However, due to their low overall
concentrations, they do not significantly influence the abundance of species with
higher concentrations.  Therefore, reactions involving these two species can be
decoupled from the rest of the network and treated independently.  In fact, $H^-$ 
and $H_2^+$ are treated as being in equilibrium at all times, independent of 
the other species and the hydrodynamic variables.  This allows a large speedup
in solution as the BDF scheme is then applied only to the slower 7- or
10-species network (depending on whether deuterium is included or not)
on timescales closer to those required by the hydrodynamics of the simulation.
Even so, the accuracy and stability of the scheme is maintained by subcycling the 
rate solver over a single hydrodynamic timestep.  These subcycle timesteps are 
determined so that the maximum fractional change in the electron concentration is
limited to no more than $10\%$ per timestep.

It is important to note the regime in which this model is valid.  According to Abel et al. and
Anninos et al. \citep{abel97,anninos97}, the reaction network is valid for temperatures
between $10^0 - 10^8$ K.  The original model discussed in these two references is only
valid up to $n_H \sim 10^4$~cm$^{-3}$.  However, addition of the 3-body $H_2$ formation
process (equation 20 in Table~\ref{table.collisional}) allows 
correct modeling of the chemistry of the gas up
until the point where collisionally-induced emission from molecular hydrogen becomes an important
cooling process, which occurs at $n_H \sim 10^{14}$~cm$^{-3}$.  A further concern is that
the optically thin approximation for radiative cooling breaks down, which
occurs before $n_H \sim 10^{16} - 10^{17}$~cm$^{-3}$.  Beyond this point, 
modifications the cooling function that take into account the non-negligible
opacity in the gas must be made, as discussed by \citet{2004MNRAS.348.1019R}. 
Even with these modifications, a completely correct description of the cooling of
this gas will require some form of radiation transport, which will greatly 
increase the cost of the simulations.

\red{Several processes are neglected.  The deuterium atom and its processes are 
completely ignored, which may have some effect.  Recent work shows that HD is
a more effective coolant than previously thought~\citep{2005MNRAS.361..850L}.  However,
the fractional abundance of HD is so low that under circumstances relevant to
the formation of a Population III star in an un-ionized region it should be
sub-dominant.  However, the enhanced electron fraction in fossil 
HII regions (as discussed later in this thesis) could result in the HD molecule
becoming a dominant cooling mechanism at relatively low ($\sim$ few hundred K)
temperatures, and could potentially cool the gas down to below $100$~K, which can
enhance fragmentation and could have important consequences for the IMF of 
primordial stars forming in a relic HII region~\citep{2005MNRAS.364.1378N}.}

Aside from deuterium, the chemical reactions involving lithium 
are also neglected.  According to \citet{1998A&A...335..403G}, these are 
not important for the density and temperature regimes explored 
by the simulations discussed in this thesis.  However, at higher densities it 
is possible that there are regimes where lithium can be an important coolant.



%---------------- table of collisional processes

\begin{table}
\begin{center}
{\bfseries Collisional Processes}\\[1ex]
\begin{tabular}{llllllll}
(1) & H & + & e$^-$ & $\rightarrow$ & H$^+$ &+& 2e$^-$ \\
(2) & H$^+$ &+ &e$^-$ & $\rightarrow$ & H &+ &$\gamma$ \\
(3) & He &+& e$^-$ & $\rightarrow$ & He$^+$ &+& 2e$^-$  \\
(4) & He$^+$ &+& e$^-$ & $\rightarrow$ & He &+ &$\gamma$  \\
(5) & He$^{+}$ &+& e$^-$ & $\rightarrow$ & He$^{++}$ &+& 2$e^-$  \\
(6) & He$^{++}$ &+& e$^-$ & $\rightarrow$ & He$^+$ &+& $\gamma$ \\
(7) & H &+& e$^-$ &$\rightarrow$& H$^-$ &+& $\gamma$  \\
(8) & H$^-$ &+& H &$\rightarrow$ & H$_2$ & +& e$^-$ \\
(9) & H &+ &H$^+$ &$\rightarrow$ &H$_2^+$ &+ &$\gamma$ \\
(10) & H$_2^+$ &+ &H &$\rightarrow$ &$H_2$ &+ &$H^+$ \\
(11) & H$_2$ &+ &H$^+$ &$\rightarrow$ &H$_2^+$ & +& H \\
(12) & H$_2$ &+ &e$^-$ & $\rightarrow$ & 2H & + & e$^-$  \\
(13) & H$_2$ & + & H & $\rightarrow$ & 3H &   &      \\
(14) & H$^-$ & + & e$^-$ & $\rightarrow$ & H & + & 2e$^-$ \\
(15) & H$^-$ & + & H & $\rightarrow$ & 2H & + & e$^-$ \\ 
(16) & H$^-$ & + & H$^+$ & $\rightarrow$ & 2H & & \\
(17) & H$^-$ & + & H$^+$ & $\rightarrow$ & H$_2^+$ & + & e$^-$ \\
(18) & H$_2^+$ & + & e$^-$ & $\rightarrow$ & 2H & & \\
(19) & H$_2^+$ & + & H$^-$ & $\rightarrow$ & H$_2$ & + & H  \\
(20) & 3H & & & $\rightarrow$ & H$_2$ & + & H
\end{tabular}
\caption[]{Collisional processes solved in the Enzo nonequilibrium
primordial chemistry routines.}
\label{table.collisional}
\end{center}
\end{table}



\begin{table}
\begin{center}
{\bfseries Radiative Processes}\\[1ex]
\begin{tabular}{llllllll}
(21) & H & + & $\gamma$ & $\rightarrow$ & H$^+$ & + & e$^-$ \\
(22) & He & + & $\gamma$ & $\rightarrow$ & He$^+$ & + & e$^-$ \\
(23) & He$^+$ & + & $\gamma$ & $\rightarrow$ & He$^{++}$ & + & e$^-$ \\
(24) & H$^-$ & + & $\gamma$ & $\rightarrow$ & H & + & e$^-$ \\
(25) & H$_2$ & + & $\gamma$ & $\rightarrow$ & H$_2^+$ & + & e$^-$ \\
(26) & H$_2^+$ & + & $\gamma$ & $\rightarrow$ & H & + & H$^+$ \\
(27) & H$_2^+$ & + & $\gamma$ & $\rightarrow$ & 2H$^+$ & + & e$^-$ \\
(28) & H$_2$ & + & $\gamma$ & $\rightarrow$ & H$_2^*$ & $\rightarrow$ & 2H \\
(29) & H$_2$ & + & $\gamma$ & $\rightarrow$ & 2H &  & 
\end{tabular}
\caption[]{Radiative processes solved in the Enzo nonequilibrium
primordial chemistry routines.}
\label{table.radiative}
\end{center}
\end{table}
