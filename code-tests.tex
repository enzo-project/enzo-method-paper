
%%%%%%%%%%%%%%%%%%%%%%%%%%%%%%%%%%%%%%%%%%%%%%%%%%%%%%%%%%%%%%%%%%%%%%%%%%%%%
\section{Code tests}
\label{sec.tests}

%%%%%%%%%%%%%%%%%%%%%%%%%%%%%%%%%%%%%%%%%%%%%%%%%%%%%%%%%%%%%%%%%%%%%%%%%%%%%
\subsection{Verifying and validating the \enzo\ code}
\label{sec.tests.vandv}

\red{(Britton)} Description of test suite and test runner.

\red{(Brian)} Refer to all code comparisons that Enzo has been part of.

%%%%%%%%%%%%%%%%%%%%%%%%%%%%%%%%%%%%%%%%%%%%%%%%%%%%%%%%%%%%%%%%%%%%%%%%%%%%%
\subsection{Representative test problems}
\label{sec.tests.problems}

General structure for each test problem:  Outline how the test problem is constructed (initial and boundary conditions), 
the analytical solution, why we have in the paper (what does it break, or what flaw does it expose (not in enzo of course)),
a plot showing how well enzo solves said test problem, and a brief description of the plot and how awesome enzo is.

%%%%%%%%%%%%%%%%%%%%%%%%%%%%%%%%%%%%%%%
\subsubsection{Sod Shock Tube}
\label{sec.tests.sodshock}
\red{(Greg)}
Problem type 1.  AMR version.  Tests the hydro.
This is also problem 7.2 in the FLASH method paper.

%%%%%%%%%%%%%%%%%%%%%%%%%%%%%%%%%%%%%%%
\subsubsection{Wave pool}
\label{sec.tests.wavepool}
\red{(Greg)}
Problem type 2, with AMR.  Tests reflections of waves at grid boundaries.

%%%%%%%%%%%%%%%%%%%%%%%%%%%%%%%%%%%%%%%
\subsubsection{Shock pool}
\label{sec.tests.shockpool}
\red{(Greg)}
Problem type 3, with AMR.  Tests passage of shock through a refinement boundary.

%%%%%%%%%%%%%%%%%%%%%%%%%%%%%%%%%%%%%%%
\subsubsection{Double mach reflection}
\label{sec.tests.doublemach}
\red{(Brian)}
Problem type 4.  This is one of Alexei's test problems.  Tests
boundary conditions in hydro.

%%%%%%%%%%%%%%%%%%%%%%%%%%%%%%%%%%%%%%%
\subsubsection{Sedov Explosion}
\label{sec.tests.sedov}
\red{(Elizabeth)}
Problem type 7.  One of Alexei's test problems.  
Also problem 7.4 in the FLASH method paper.
This tests the hydro in the strong-shock limit.

%%%%%%%%%%%%%%%%%%%%%%%%%%%%%%%%%%%%%%%
\subsubsection{Point source gravity test}
\label{sec.test.gravitypointsource}
\red{(Greg)}
This is the TestGravity (problem 23) test problem.  It tests gravity around a point source, using fixed AMR.

%%%%%%%%%%%%%%%%%%%%%%%%%%%%%%%%%%%%%%%
\subsubsection{Orbit Test}
\label{sec.test.testorbit}
\red{(Greg)}
This is the TestOrbit problem (problem 29 for GB).  This tests gravity and particle integration (no hydro).

%%%%%%%%%%%%%%%%%%%%%%%%%%%%%%%%%%%%%%%
\subsubsection{Self-Similar infall test}
\label{sec.tests.infall}
\red{(Greg)}
Problem type 24.  This is a test based on Bertschinger's 1985 3D self-similar infall
solution, and tests gravity + hydro.

%%%%%%%%%%%%%%%%%%%%%%%%%%%%%%%%%%%%%%%
\subsubsection{Zel`Dovich Pancake}
\label{sec.tests.pancake}
\red{(Brian)}
Problem type 20, unigrid and AMR both.  Tests gravity + hydro.

%%%%%%%%%%%%%%%%%%%%%%%%%%%%%%%%%%%%%%%
\subsubsection{Cosmology simulations}
\label{sec.tests.}
\red{(Brian)}
Talk about cosmology code comparisons, where Enzo is good/weak.

%%%%%%%%%%%%%%%%%%%%%%%%%%%%%%%%%%%%%%%
\subsubsection{Representative MHD test}
\label{sec.tests.mhd}
\red{(Dave)}
One representative MHD test.  What's the best/hardest one?

%%%%%%%%%%%%%%%%%%%%%%%%%%%%%%%%%%%%%%%
\subsubsection{1-zone free-fall test}
\label{sec.tests.freefall}
\red{(Britton)}
One zone free fall test.  Tests cooling, chemistry.

%%%%%%%%%%%%%%%%%%%%%%%%%%%%%%%%%%%%%%%
\subsubsection{Photo-evaporation of a dense clump}
\label{sec.tests.raytracing}
\red{(John)}
Photo-evaporation of a dense clump.  Tests ray tracing, specifically
the shadowing effects of the clump, and the hydrodynamic response of
the clump.

%%%%%%%%%%%%%%%%%%%%%%%%%%%%%%%%%%%%%%%
\subsubsection{Representative FLD test}
\label{sec.tests.fld}
\red{(Dan)}
One test for the implicit FLD: best/hardest one.


