
%%%%%%%%%%%%%%%%%%%%%%%%%%%%%%%%%%%%%%%%%%%%%%%%%%%%%%%%%%%%%%%%%%%%%%%%%%%%%
\section{Code tests}\label{sec.tests}

\red{
\begin{enumerate}
\item Things that need to be addressed:  Can we refine by non-powers of two?  What test problems to do?  Order of test problems?
\item Some of the test problems are motivated by the FLASH code method paper \citep{flash_method}, and are referred to there.  Others are motivated by the GADGET-2 method paper \citep{gadget_2_method}, and are referred to as such.
\item In this paper we want to make sure to refer to the Santa Barbara Cluster comparison project \citep{SantaBarbara} and to our ENZO/GADGET-2 comparison project~\citep{2005ApJS..160....1O}.
\end{enumerate}
}

\red{
General structure for each test problem:  Outline how the test problem is constructed (initial and boundary conditions), 
the analytical solution, why we have in the paper (what does it break, or what flaw does it expose (not in enzo of course)),
a plot showing how well enzo solves said test problem, and a brief description of the plot and how awesome enzo is.
}

\red{*** WE NEED SUGGESTIONS FOR CODE TESTS***}

\subsection{Advection problem}\label{sec.tests.advect}
Problem 7.1 in the flash method paper~\citep{flash_method}.

\subsection{Sod Shock Tube}\label{sec.tests.sodshock}
Problem type 1.  Both unigrid and AMR.  Tests the hydro.
This is also problem 7.2 in the FLASH method paper.

\subsection{Wave pool}\label{sec.tests.wavepool}
Problem type 2, unigrid and fixed AMR 1D mostly.  Tests reflections of waves at boundaries.

\subsection{Shock pool}\label{sec.tests.shockpool}
Problem type 3, unigrid and AMR 2D.  Tests passage of shock through a refinement boundary.

\subsection{Interacting Blast-Wave problem}\label{sec.tests.interblast}
Problem 7.5 in the FLASH method paper.


\subsection{Double mach reflection}\label{sec.tests.doublemach}
Problem type 4.  This is one of Alexei's test problems: 
http://cosmos.ucsd.edu/~akritsuk/renzo/mach/mach.html

This tests the hydro.

\subsection{Sedov Explosion}\label{sec.tests.sedov}
Problem type 7.  One of Alexei's test problems:
http://cosmos.ucsd.edu/~akritsuk/renzo/sedov/sedov.html
Also problem 7.4 in the FLASH method paper.
This tests the hydro.

\subsection{Noh problem}\label{sec.tests.noh}
Problem type 9.  One of Alexei's test problems:
http://cosmos.ucsd.edu/~akritsuk/renzo/noh/noh.html
This tests the hydro.

\subsection{Radiating shock problem}\label{sec.test.radshock}
In~\citet{anninos97} -- tests the energy equation.  Followon
to Noh problem.  Original reference is cited in this paper.

\subsection{Point source gravity test}\label{sec.test.gravitypointsource}
This is the TestGravity (problem 23) test problem.  It tests gravity around a point source, using fixed AMR.

\subsection{Orbit Test}\label{sec.test.testorbit}
This is the TestOrbit problem (problem 29 for GB).  This tests gravity and particle integration (no hydro).

\subsection{Pressureless Sphere Collapse}\label{sec.test.pressurelesscollapse}
This is in the zeus paper and in the kronos test paper -- it tests mostly gravity and advection.  Test problem 21.

\subsection{Collapse of an adiabatic gas sphere}\label{sec.tests.adiasphere}
I think this is problem type 27, and also is problem 6.2 in the
GADGET-2 method paper~\citep{gadget_2_method}.



\subsection{Self-Similar infall test}\label{sec.tests.infall}
Problem type 24.  This is a test based on Bertschinger's 1985 3D self-similar infall
solution, and tests gravity + hydro.

\subsection{Zel`Dovich Pancake}\label{sec.tests.pancake}
Problem type 20, unigrid and AMR both.  Tests gravity + hydro.

\subsection{Santa Barbara Cluster comparison}\label{sec.tests.santabarbara}
This is the classic test -- we can mostly just point to the enzo results in that paper.  Maybe update them?

\subsection{Dark matter halo mass function and clustering}\label{sec.tests.dm}
Standard problem, also problem 6.4 in GADGET-2 method paper.  Should this be here, or just a reference to another paper?

