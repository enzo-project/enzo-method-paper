\section{Introduction}\label{sec.intro}

Due to the high spatial and temporal dynamical ranges involved,
astrophysical and cosmological phenomena present a taxing challenge for
simulators. To tackle such situations, a number of numerical
techniques have been developed that can be broadly split into the
gridless, Lagrangian methods and grid-based, Eulerian schemes. The
most commonly used is an example of the first type known as Smoothed
Particle Hydrodynamics \citep[SPH;][]{Lucy77, SPH}. It has achieved
much success, particularly in regimes dominated by gravity. However,
its development to include an increasing number of sought-after
physical processes is still at a relatively early stage when compared
with the effort put into the latter type of Eulerian grid-based
hydrodynamic schemes \citep[e.g.,][]{laney-1998, toro-1997,
Woodward84}.

Despite this invested expertise, the Eulerian solvers in their
original form have a serious drawback: they do not provide an easy
method of adaptively increasing the spatial and temporal resolution in
small volumes of the simulation. Such flexibility is essential for
following physical processes such as gravitational instability. A
solution to this problem was first proposed by \citet{Berger89} in the
Computational Fluid Dynamics (CFD) community, and became known as
Structured Adaptive Mesh Refinement (SAMR). The principle is to
adaptively add and modify additional, finer meshes (``grids'') over
regions that require higher resolution. In addition, it is possible to
add other, more advanced physics including -- for the AMR
implementation in the astrophysics code presented in this paper --
comoving coordinates, self-gravity, radiative cooling, chemistry, heat
conduction, collisionless fluids, magnetohydrodynamics, radiation
transport, star formation and a range of other physical effects.
\newpage

There have been a number of numerical methods described in the
astronomical literature that contain elements of SAMR or have a
similar aim. For example, the N-body solver developed by
\citet{Villumsen89} used non-adaptive meshing to increase the
resolution in pre-selected regions. This static approach was later
used extensively when applied to hydrodynamics
\citep[e.g.,][]{Ruffert94, Anninos94}. Adding adaptivity is a more
recent enhancement, and there are now a number of N-body codes that
possess this feature, both with and without hydrodynamics
\citep{Couchman91, Jessop94, Suisalu95, Splinter96, Gelato97, ART97,
Truelove98, flash_method, MLAPM01, Yahagi01, RAMSES, Quilis04,
Ziegler05, Zhang06, Astrobear09, Pluto-amr, GAMER, Nyx}. Of these,
perhaps the most comparable and widely used are FLASH
\citep{flash_method}, which uses grid blocks of fixed size, and RAMSES
\citep{RAMSES} and ART \citep{ART97}, both of which refine individual
cells. It is also possible to deform the grid to obtain high
resolution \citep[e.g.,][]{Gnedin95, Xu97, Pen98}, and more recently a
few grid codes have adopted an unstructured approach based on a moving
Voronoi mesh \citep{Arepo10, Tess11}.

In this paper, we present \enzo; a structured adaptive mesh refinement
(SAMR) code. Originally developed for cosmological hydrodynamics,
\enzo\ has since been used on a wide variety of problems.  It has
grown to become a general tool for astrophysical fluid dynamics and is
intended to be efficient, accurate and easily extended to include new
capabilities.  Although many of the components of the \enzo\ code have
been described in previous publications \citep{1995CoPhC..89..149B,
BryanThesis96, Bryan97a, Bryan97b, Norman99, BryanCompSci99, Bryan01,
Oshea04, 2007arXiv0705.1556N,
WangAbelZhang08,ReynoldsHayesPaschosNorman2009,Collins10,Wise11_Moray},
there has previously been no systematic and complete description of
the code.  In this paper we provide that description, filling in many
previous omissions and showing the code's performance for a wide
variety of test problems.

The \enzo\ code has been extensively used over the last two decades in
a wide variety of problems, resulting in the publication of more than
100 peer-reviewed papers based on results obtained with the code. The
variety of astrophysical systems that \enzo\ has been used for include
galaxies \citep{2003ApJ...587...13T, 2012MNRAS.425..641L,
  2012ApJ...749..140H, 2013MNRAS.432.1989S}, galaxy
clusters \citep{Loken02, Xu11, Skillman13}, the interstellar medium
\citep{Slyz05}, the intergalactic medium \citep{Fang01,
2011ApJ...731....6S}, cooling flows \citep{Li12, 2013ApJ...763...38S},
turbulence \citep{Kritsuk04, 2007ApJ...665..416K, 2009JPhCS.180a2020K, Collins11}, the
formation of the first stars \citep{ABN02, 2007ApJ...654...66O,
2009Sci...325..601T, Xu08}, and the formation of stars in our own Galaxy
\citep{Collins11, 2011ApJ...727L..20K, Collins12a}.

Numerical simulations of astronomical phenomena now play a key role,
along with observations and analytic theory, in pushing forward our
understanding of the cosmos \citep[e.g.,][]{DecadalSurvey01,
DecadalSurvey10}.  But along with this role comes responsibility.  We
believe that those developing simulation tools must fulfill two key
obligations: the first is to make those tools available to the
community as a whole, much in the way that astronomical data are now
regularly made publicly available.  The second is to document, test
and refine those methods so that they can be critically evaluated and
expanded upon by others.  Our public release of the \enzo\ code represents our
attempt to meet the first of these obligations; this paper represents
our attempt to meet the second.

The structure of this paper is as follows.  In
Section~\ref{sec.overview}, we first provide a top-level overview of
the code method and structure.  This is designed to give a broad-brush
picture of the equations solved by \enzo\ and the methods used to
solve them.  Next, in Sections~\ref{sec.amr}
through~\ref{sec.num.analysis}, we describe the methods we use in
detail, reserving some of the longer descriptions of particular
components for the appendix in order to not interrupt the flow of the
paper.  The \enzo\ testing framework and code tests are described in
Section~\ref{sec.tests}.  The parallelism strategy and scaling results
are described in Section~\ref{sec.parallel}.  Finally, we discuss the
code's development methodology (which is, as far as we know, unique in
the astrophysics community) in Section~\ref{sec.development}.
