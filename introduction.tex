
\section{Introduction}\label{sec.intro}

The numerical simulation of astronomical phenomena now plays a key role, along with observations and analytic theory, in pushing forward our understanding of the cosmos.
%\citep[e.g.,][]{DecadalSurvey01}.  
But we believe that with this role comes responsibility.  Those developing simulation tools must fulfill two key obligations: the first is to make those tools available to the community as a whole, much in the way that astronomical data is usually made publicly available.  The second is document, test and refineme those methods so that they can be critically evaluated and expanded by others.  This paper represents our attempt to meet our obligations with respect to the \enzo\ code.

\enzo\ is a structured adaptive mesh refinement (SAMR) code that was originally developed for cosmological hydrodynamics and has been used on a wide variety of problems.  It has been expanded as a general tool for astrophysical fluid dynamics and is intended to be efficient, accurate and easily expanded.  Although many of the components of the code have been described in previous publications \citep{1995CoPhC..89..149B, BryanThesis96, Bryan97a, Bryan97b, Norman99, BryanCompSci99, Bryan01, Oshea04}, there has been no systematic and complete description of the code.  In this paper we provide that description, filling in many gaps and showing the result of the code on a wide variety of test problems.

There are many diverse phenomena in astrophysics and cosmology which require a high spatial and temporal dynamical range. There are a number of numerical techniques which have been used to
model such problems.  The most common is a gridless, fully Lagrangian method call Smoothed Particle Hydrodynamcs \citep[SPH;][]{Lucy77, SPH} which has been very successful, particularly for those situations dominated by gravity.  Its development, however, is still at a relatively early stage when
compared to the effort put into Eulerian, grid-based hydrodynamic schemes \citep{Woodward84}. 

Therefore, we would like to make use of this expertise; unfortunately, none of these schemes provide an easy way to adaptively and dynamically increase the resolution in small volumes, as required by, for example, gravitational instability.  We have instead, adapted an idea first suggested by \citet{Berger89} in the Computational Fluid Dynamics (CFD) community, which has become known as Structured Adaptive Mesh Refinement (SAMR).  The principle is to adaptively add (and modify) additional, finer meshes, or patches, over regions which require higher resolution.  Since each patch has a simple geometry (here, cartesian) the hyperbolic fluid-dynamical equations can be solved on that patch with a well-developed `off-the-shelf' numerical method.  In addition, it will be relatively straightforward to add other advanced physics (e.g. MHD, radiation transport beyond flux limited diffusion, etc.). This paper is devoted to describing how we have implemented this idea for astrophysics, including the addition of self-gravity and collisionless fluids.

There have been a few other numerical methods described in the astronomical literature which contain elements of this idea, or have a similar aim.  For example, an N-body solver \cite{Villumsen89} used
non-adaptive meshing to increase the resolution in pre-selected.  This static approach, applied to hydrodynamics, has been used extensively \citep[e.g.,]{Ruffert94, Anninos94}.  Adding adaptivity is a more recent enhancement, and there are now a number of N-body codes which possess this feature, both with and without  hydrodynamics  \cite{Couchman91, Jessop94, Suisalu95, Splinter96, Gelato97, ART97, Truelove98, flash_method, MLAPM01,  Yahagi01, RAMSES, Quilis04, Ziegler05, Zhang06}. It is also possible to deform the grid to obtain high resolution \citep[e.g.,][]{Gnedin95, Xu97, Pen98}.

The \enzo\ code has been widely used over the lasts decade in a wide variety of problems in topics ranging from star formation to turbulence to large-scale structure.  More than 50 peer-reviewed papers have been published based on results obtained with the code.

The structure of the code is as follows.  In section~\ref{sec.overview}, we first provide a top-level overview of the code method and structure.  This is designed to give a first broad-brush picture of the equations we aim to solve and the methods used to solve them.  Next, in section~\ref{sec.methods}, we describe the methods we use in detail, reserving some of the longer descriptions of particular components for the appendix in order not to interrupt the flow of the paper.  The code tests are described in section~\ref{sec.tests} and the parallelism strategy and scaling results in section~\ref{sec.performance}.

%\red{
%\begin{enumerate}
%  \item Motivate why we're writing the paper now: (because there isn't one, lots of people use it for all sorts of things, given that it's publicly available we should really have a method paper and some general proof that the code works, and a list of what it does at this point in time)
%  \item Why have the \enzo\ code at all?  Many diverse phenomena in astrophysics and cosmology where you need a high spatial and temporal dynamical range.  Regular (i.e. monolithic) grids don't have the required range.  While SPH is the most commonly used tool, and has been quite successful, the lack of a grid structure complicates the usage of advanced hydrodynamical techniques and other advanced physics (e.g. full solution of the MHD equations, radiation transport beyond flux limited diffusion, etc.).  There has been 50(ish) years of advancement in the fields of grid-based solutions to these methods, and the use of adaptive mesh on structured grids allows us to take advantage of this for a wide range of astrophysical and cosmological applications.  
%  \item List the subjects that the code has been used for, citing appropriate papers.  Emphasize heavily landmark contributions (Greg's 1998 galaxy cluster paper and various Motl/Burns/Hallman cluster things, various lyman alpha forest stuff, Pop III papers, alexei's turbulence stuff, etc.).  This code has been used for approximately 10 years, and is publicly available.  List papers by non-collaborators that have already used the code.
%  \item Explain structure of the paper.  Emphasize that we pass through the enzo code in multiple levels of (increasing) detail: The ``physical equations'' section provides bare-bones ``what we're solving'' information, the ``overview of numerical methods'' section briefly explains (i.e. no pseudocode or finite difference stencils) how the various important subsystems of the code work, and the appendices of the paper have the real nitty-gritty details that most people don't want to know, but a select few will find highly useful.
%\end{enumerate}
%} %end red

