
\section{Introduction}\label{sec.intro}


Many phenomena in astrophysics and cosmology require a high spatial and temporal dynamical range, and there are a number of numerical techniques that have been used to model such problems.  At the current time, the most common is a gridless, fully Lagrangian method call Smoothed Particle Hydrodynamcs \citep[SPH;][]{Lucy77, SPH} which has been very successful -- particularly for those situations dominated by gravity, where systems evolve over large spatial and temporal dynamic ranges.  Its development, however, is still at a relatively early stage when compared to the effort put into Eulerian, grid-based hydrodynamic schemes \citep[e.g.,][]{laney-1998, toro-1997, Woodward84}. 

Therefore, we would like to make use of the expertise that has been invested in Eulerian solvers; unfortunately, as originally developed, none of these schemes provide an easy way to adaptively and dynamically increase the spatial and temporal resolution in small volumes, as required by, for example, gravitational instability.  We have instead adopted an idea first suggested by \citet{Berger89} in the Computational Fluid Dynamics (CFD) community, which has become known as Structured Adaptive Mesh Refinement (SAMR).  The principle is to adaptively add and modify additional, finer meshes (``patches'') over regions that require higher resolution.  Since each patch has a simple geometry (Cartesian, in our case) the hyperbolic fluid-dynamical equations can be solved on that patch with a well-developed `off-the-shelf' numerical method.  In addition, it is relatively straightforward to add other advanced physics (e.g. magnetohydrodynamics, radiation transport, chemistry, etc.). This paper is devoted to describing how we have implemented this idea for astrophysics, including the addition of comoving coordinates, self-gravity, radiative cooling, chemistry, heat conduction, collisionless fluids, MHD, radiation transport, star formation and a range of other physical effects.

There have been many other numerical methods described in the astronomical literature which contain elements of this idea, or have a similar aim.  For example, an N-body solver \citep{Villumsen89} used non-adaptive meshing to increase the resolution in pre-selected regions.  This static approach, applied to hydrodynamics, has been used extensively \citep[e.g.,][]{Ruffert94, Anninos94}.  Adding adaptivity is a more recent enhancement, and there are now a number of N-body codes that possess this feature, both with and without  hydrodynamics  \citep{Couchman91, Jessop94, Suisalu95, Splinter96, Gelato97, ART97, Truelove98, flash_method, MLAPM01,  Yahagi01, RAMSES, Quilis04, Ziegler05, Zhang06, Astrobear09, Pluto-amr, GAMER, Nyx}.  Of these, perhaps the most comparable and widely used are FLASH \citep{flash_method}, which uses grid blocks of fixed size, and RAMSES \citep{RAMSES} and ART \citep{ART97}, both of which refine individual cells.   It is also possible to deform the grid to obtain high resolution \citep[e.g.,][]{Gnedin95, Xu97, Pen98}, and more recently a few grid codes have adopted an unstructured approach based on a moving Voronoi mesh \citep{Arepo10, Tess11}.

\enzo\ is a structured adaptive mesh refinement (SAMR) code that was originally developed for cosmological hydrodynamics and has been used on a wide variety of problems.  It has been expanded as a general tool for astrophysical fluid dynamics and is intended to be efficient, accurate and easily expanded.  Although many of the components of the code have been described in previous publications \citep{1995CoPhC..89..149B, BryanThesis96, Bryan97a, Bryan97b, Norman99, BryanCompSci99, Bryan01, Oshea04, 2007arXiv0705.1556N}, there has been no systematic and complete description of the code.  In this paper we provide that description, filling in many gaps and showing the code's performance for a wide variety of test problems.

The \enzo\ code has been extensively used over the last two decades in a wide variety of problems, and more than 100 peer-reviewed papers have been published based on results obtained with the code.  The code has previously been used to study a wide range of astrophysical systems, including galaxy clusters \citep{Loken02}, the interstellar medium \citep{Slyz05}, the intergalactic medium \citep{Fang01}, cooling flows \citep{Li12}, turbulence \citep{Kritsuk04}, and the formation of the first stars \citep{ABN02}.

Finally, we note that the numerical simulation of astronomical phenomena now plays a key role, along with observations and analytic theory, in pushing forward our understanding of the cosmos \citep[e.g.,][]{DecadalSurvey01, DecadalSurvey10}.   But along with this role comes responsibility.  We believe that those developing simulation tools must fulfill two key obligations: the first is to make those tools available to the community as a whole, much in the way that astronomical data is now regularly made publicly available.  The second is document, test and refine those methods so that they can be critically evaluated and expanded by others.  This paper represents one part of our attempt to meet our obligations with respect to the \enzo\ code.

The structure of the code is as follows.  In Section~\ref{sec.overview}, we first provide a top-level overview of the code method and structure.  This is designed to give a first broad-brush picture of the equations we aim to solve and the methods used to solve them.  Next, in Section~\ref{sec.methods}, we describe the methods we use in detail, reserving some of the longer descriptions of particular components for the appendix in order not to interrupt the flow of the paper.  The \enzo\ testing framework and code tests are described in Section~\ref{sec.tests}.  The parallelism strategy and scaling results are described in Section~\ref{sec.parallel}.  Finally, we discuss the code's development methodology (which is as far as we know unique in the astrophysics community) in Section~\ref{sec.development}.

