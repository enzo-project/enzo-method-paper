% EJT: general changes
% Enzo to \enzo\
% changed ``some Lagrangian fluid trajectories'' to ``a fixed number of Lagrangian fluid trajectories.'' Easier to understand.
% changed ``some evolution'' to ``a period of evolution'' ... I basically don't like 'some'. It sounds vague.
% rephrased to remove ``e.g.'' as they read slightly awkwardly
% moved definition of T_1 and T_2 slightly higher in shock text to where they first appear.

\section{Analysis}
\label{sec.num.analysis}

\subsection{Inline analysis with yt}

Detailed analysis of simulation results requires both the tools to ask
sophisticated questions of the data and the ability to process vast quantities of
data at high time-cadence.  As simulations grow in size and complexity, storing data
for post-processing simply becomes intractable.  To cope with this, we have
instrumented \enzo\ with the ability to conduct analysis during the course of a
simulation.  This enables analysis with extremely high time cadence (as often
as every subcycle of the finest refinement level), without
attempting to write an entire checkpoint output to disk.  The current mechanism
for conducting analysis in \enzo\ during the course of the simulation utilizes
the same computional resources as are used by the simulation itself by transferring their usage 
from \enzo\ to the analysis routines; this is often
referred to as \textit{in situ} analysis or visualization.  Utilizing a
dynamically-scheduled second set of computing resources, often referred to as
\textit{co-scheduled} analysis or visualization, provides greater flexibility and overall
throughput at the expense of simplicity.

We expose \enzo's mesh geometry and fluid quantities to the analysis platform
\texttt{yt} \citep{2011ApJS..192....9T, 2011arXiv1112.4482T}.  At compile time,
\enzo\ is (dynamically or statically) linked against the Python and NumPy
libraries necessary to create proxy objects exposing the mesh geometry,
fluid quantities and particle arrays as NumPy arrays.  This information is then
passed to a special handler inside \texttt{yt}.  \texttt{yt} interprets the
mesh and fluid information and, without saving data to disk, constructs a native representation of the in-memory state
of the simulation that appears identical to an on-disk simulation output.
\enzo\ then executes a user-provided analysis script, which is able to access the
in-memory simulation object.  Once the analysis script has returned control to
\enzo, the simulation proceeds.  This process can occur at either the top of the
main ``EvolveHierarchy'' loop or at the end of a timestep at the finest level,
and the frequency with which it is called is adjustable by a run-time
parameter.  During the course of conducting analysis, the simulation is halted
until the conclusion of the analysis.

Most analysis operations that can be performed on data sets that reside on
disk can be performed on in-memory data sets in \texttt{yt}.  This includes
projections (i.e., line integrals, both on- and off-axis), slices, 1-, 2-, and
3-D fluid phase distributions, calculation of derived quantities and arbitrary
data selection.  As of version 2.5 of \texttt{yt},  the Rockstar phase-space
halo finder \citep{2013ApJ...762..109B} can be executed through \texttt{yt} on
in-memory \enzo\ data, and so can the Parallel HOP halo finder
\citep{1998ApJ...498..137E, 2010ApJS..191...43S}.  Operations that currently cannot be conducted on
in-memory datasets are those that require spatial decomposition of
data.  For 
instance, calculating marching cubes on a data object with \texttt{yt} is a
fully local operation and can be conducted \textit{in situ}.  However,
calculating topologically-connected sets requires a spatial decomposition of
data and thus cannot be conducted \textit{in situ}.  This prohibition extends
to halo finding operations other than Rockstar, most multi-level
parallelism operations, and volume rendering.

Where microphysical solvers or other operator-split physics calculations can be
done in Python, \texttt{yt} can serve as a driver for these calculations.  A
major feature set that is currently being developed is to pass structured
(i.e., non-fluid) information back from \texttt{yt} into \enzo.  For instance,
this could be the result of semi-analytic models of the growth and evolution of
star clusters, galaxy particle feedback parameters that have been influenced by
merger-tree analysis, or even spectral energy distributions that are calculated
within \texttt{yt} and provided to \enzo\ as input into radiative transfer
calculations.  Future versions will include this, as well as the ability to
dynamically allocate computional resources to \texttt{yt} such that the simulation
may proceed asynchronously with analysis (\textit{co-scheduled} analysis).
With this functionality will also come the ability to dynamically partition
data, such that spatially-decomposed operations such as volume rendering become
feasible during the course of a simulation.

\subsection{Tracer particles}

One of the inherent drawbacks of a grid-based fluid method is the
inability to follow the evolution of a single parcel of fluid as it
travels through the simulation volume.  To address this, \enzo\ has the
capability to introduce Lagrangian ``tracer particles'' into a
calculation either at the beginning of the simulation or when
restarting the calculation.  These tracer particles are put into the
simulation in a rectangular solid volume with uniform, user-specified
spacing.  Each particle's position and velocity is updated over the
course of a single timestep $\Delta t$ as follows:

\begin{eqnarray}
\label{eqn.drifttrace}
x^{n+1/2} & = & x^n + (\Delta t/2) v^{\rm interp,n} \nonumber \\
v^{n+1} & = & v^{\rm interp,n+1} \\
x^{n+1} & = & x^{n+1/2} + (\Delta t/2) v^{n+1} \nonumber
\end{eqnarray}

This is essentially a drift-kick-drift particle update from time $n$
to time $n+1$ -- however, instead of computing an acceleration at the
half-timestep $t^n + \Delta t/2$ (as is done for massive particles --
see Equation~\ref{eqn.driftkick}), the particle's velocity is updated
both at the beginning of each timestep and at the half-timestep
by linearly interpolating the cell-centered baryon velocity to the
position of the particle ($v^{\rm interp}$), and assigning it to that value.

\enzo\ saves tracer particle data at user-specified intervals,
independent of the intervals at which regular data sets are written.  The data written out
typically includes the tracer particle's unique ID and position, as
well as the velocity, density,
and temperature of the gas at its location, but is easily extensible to output any grid-based
quantity that a user requires.  This capability has been used quite
effectively in several papers, including \citet{2010ApJ...715.1575S}
and \citet{2012ApJ...748...12S}. It is crucial to keep in mind that
tracer particles model a fixed number of Lagrangian fluid trajectories. The fluid
on the grid, however, models the motion of all the mass and represents
the average quantity in a grid cell's volume. Consequently, after a period of
evolution, tracer particles -- even if they initially had the same
density distribution as the gas -- will not have the same density
distribution as the fluid. For example, they tend to accumulate at stagnation points
of the flow, and care has to be taken in using these particles
appropriately. Tracer particles are very useful in studies such as the variety of
histories of the hydrodynamic quantities in Lagrangian fluid elements
and when evaluating complex chemical and cooling models in regard to the
simpler ones used in the actual numerical evolution.  

\subsection{Shock finding}

Identification of shocks and their pre- and post-shock conditions can be
accomplished through a combination of either temperature or velocity jumps with
dimensionally split or unsplit search methods.  The primary method used in
\enzo~is the dimensionally unsplit temperature jump method described in detail
in \citet{2008ApJ...689.1063S}.  We briefly outline the method here.

For every cell, we first determine whether it satisfies the following conditions
necessary to be flagged as a shock:

\begin{eqnarray}
\div \myvec{v} < 0,\\
\grad T \cdot \grad S > 0,\\
T_2 > T_1,\\
\rho_2 > \rho_1,
\end{eqnarray}

where $\myvec{v}$ is the velocity field, $T$ is the temperature, $\rho$ is the
density, and $S=T/\rho^{\gamma-1}$ is the entropy.  $T_2$ and $T_1$ are the post-shock~(downstream)
and pre-shock~(upstream) temperatures, respectively. An optional temperature floor
may be chosen, which is useful for situations such as cosmological simulations without a radiation
background where underdense gas in the intergalactic medium (IGM) can cool adiabatically to
unphysically low temperatures.

Once a cell is flagged, the local temperature gradient is calculated, which is 
then used to traverse cells parallel to the gradient to search for the first
pre- and post-shock cell that do not satisfy the above conditions.  If during
the search a cell satisfying the conditions is found to have a more convergent
flow, that cell is marked as the center, and the search is started again. Using the 
temperature values from each of these cells, the Mach number is then solved
using the Rankine-Hugoniot temperature jump conditions:
\begin{equation}
\frac{T_2}{T_1} = \frac{(5\Mach^2 - 1)(\Mach^2 + 3)}{16\Mach^2},
\end{equation}
where $\Mach$ is the upstream Mach number.  

Shock finding in the context of AMR is applied grid-by-grid.  If a search for
pre/post-shock cells goes outside the bounds of the grid ghost zones, the
search is stopped for that particular shocked cell. In most situations this is
adequate since the hydrodynamic shock is captured in fewer than the number of
ghost zones.  Shock finding can be run either upon data output or at each step
in the evolution of an AMR level if the Mach and pre/post-shock quantities are
needed for additional physics modules.

