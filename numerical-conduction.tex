\subsection{Thermal conduction}
\label{sec.num.conductions}

\enzo\ implements the equations of isotropic heat conduction in a manner
similar to that of \citet{2007ApJ...664..135P}.  
The isotropic flux of heat is given by:
\begin{equation}
\myvec{F} = -\kappa_{\rm sp} \grad T
\end{equation}
where $\kappa_{\rm sp}$ is the Spitzer conduction coefficient, $4.6 \times 10^{-7}$~T$^{5/2}$ erg
s$^{-1}$~cm$^{-1}$~K$^{-1}$ \citep{1962pfig.book.....S}, here we are using a
value for the Coulomb logarithm, $\log \Lambda = 37.8$, that is
appropriate for the intracluster medium \citep{1988xrec.book.....S}.
It is quite possible that the local heat flux computed in this way 
can become unphysically large in
the high-temperature, low-density cluster regime when using this
formulation; therefore, we take into account the saturation of the heat
flux \citep{1977ApJ...211..135C} at a maximum level of
\begin{equation}
F_{sat} \simeq 0.4 n_e k_b T \left( \frac{2 k_b T}{\pi m_e} \right)^{1/2}.
\end{equation}
To ensure a smooth transition between
the Spitzer and saturated regimes, we define an effective
conductivity using the formalism of \citet{1988xrec.book.....S}
\begin{equation}
\kappa_{eff} = \frac{\kappa_{\rm sp}}{1 + 4.2 \lambda_e / \ell_T} \; ,
\end{equation}
where $\lambda_e$ is the electron mean free path and $\ell_T \equiv T
/ |\grad T|$ is the characteristic length scale of the local
temperature gradient.  We also assume that the conductivity of the
plasma can be described in terms of an effective conductivity, which
can be expressed as a fraction f$_{\rm sp}$ of the Spitzer conductivity
(where f$_{\rm sp} \leq 1.0$ are considered physical values).  This takes
into account physical processes below the resolution limit of the
simulation (such as tangled magnetic fields) that can suppress heat
transport.

Thermal conduction in a plasma can be strongly affected by the
presence of magnetic field lines, which may suppress heat
flow perpendicular to the magnetic field.  In that case, we allow for
heat transport only parallel to the magnetic field lines in
magnetohydrodynamic simulations.  Mathematically, this is treated as:
\begin{equation}
\myvec{F} = -\kappa_{\rm sp} \myvec{b} (\myvec{b} \cdot \grad T)
\end{equation}
where $\myvec{b}$ is the unit vector denoting the direction of the
magnetic field.  As with the isotropic thermal conduction, we allow a
multiplicative factor f$_{\rm sp}$ to take into account the possible
behavior of magnetic fields below the resolution limit of the
simulation.

Both isotropic and anisotropic thermal conduction in \enzo\ are treated
in an operator-split manner.  Furthermore, within the heat transport
module, transport along the x, y, and z directions are computed in a
directionally-split fashion, with heat flux along each direction
calculated at the + and - faces of the cell using the arithmetic mean
of the cell-centered temperature in cells $n$ and $n+1$ or $n-1$ and
$n$, respectively (empirically, this is more stable than taking the
geometric mean of the cell-centered temperatures).  The addition of
transport along magnetic field lines requires the calculation of
cross-terms in the temperature derivatives at cell faces, which can
result in spurious oscillations in the temperature field in regions
where the temperature gradient is strong in more than one spatial
direction.  Controlling these oscillations requires the addition of a
flux limiter for calculations of the temperature field.  In this case,
we choose the monotonized central difference flux limiter
\citep{1977JCoPh..23..263V}, which serves to maintain numerical
stability without sacrificing substantial speed or accuracy.
