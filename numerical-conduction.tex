\subsection{Thermal conduction}
\label{sec.num.conductions}

We've got both kinds of conduction: isotropic AND anisotropic.  How
great is that?



Enzo implements the equations of isotropic heat conduction in a manner
similar to that of \citet{2007ApJ...664..135P}, where we model the
flux of isotropic heat transport:

\begin{equation}
\mathbb{F} = -\kappa_{sp} \nabla T
\end{equation}

Where $\kappa_{sp}$ is the Spitzer conduction coefficient, $4.6 \times 10^{-7}$~T$^{5/2}$ erg
s$^{-1}$~cm$^{-1}$~K$^{-1}$ \citep{1962pfig.book.....S}, and using a
value for the Coulomb logarithm, $\log \Lambda = 37.8$, that is
appropriate for the intracluster medium \citep{1988xrec.book.....S}.
It is quite possible that the local heat flux can become unphysical in
the high-temperature, low-density cluster regime when using this
formulation; thus, we take into account the saturation of the heat
flux at a maximum level of

\begin{equation}
j_{sat} \simeq 0.4 n_e k_b T \left( \frac{2 k_b T}{\pi m_e} \right)^{1/2}
\end{equation}

\citep{1977ApJ...211..135C}, and to ensure a smooth transition between
the Spitzer regime and the saturated regime we define an effective
conductivity using the formalism of \citet{1988xrec.book.....S}

\begin{equation}
\kappa_{eff} = \frac{\kappa_{sp}}{1 + 4.2 \lambda_e / \ell_T} \; ,
\end{equation}

where $\lambda_e$ is the electron mean free path and $\ell_T \equiv T
/ |\nabla T|$ is the characteristic length scale of the local
temperature gradient.

It is important to note that the conductivity can be strongly affected
by the presence of magnetic fields, which may strongly suppress heat
flow.  The strength and confuction of magnetic fields in the
intracluster medium is poorly understood, and it is unclear to what
extent these magnetic fields will suppress conduction.  For the
purposes of the work described in this paper, we assume that the
conductivity of the plasma can be described in terms of an effective
conductivity, which can be expressed as a fraction f$_{Sp}$ of the
Spitzer conductivity (where f$_{Sp} \leq 1.0$ are considered physical
values).


